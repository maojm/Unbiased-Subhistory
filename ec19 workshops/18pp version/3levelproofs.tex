%!TEX root = main.tex

\section{Proof of Theorem~\ref{thm:3level}}
\label{app:3level}

\subsection{High-probability events}
\label{sec:3level-events}

The following lemmas can be derived from combining Lemma~\ref{lem:t1runs} and union bound.

\begin{lemma}[Concentration of first-level number of pulls.]\label{3levelw1}
  Let $W_1$ be the event that for all groups $s\in [\NG]$ and arms
  $a\in \{1, 2\}$, the number of arm $a$ pulls in the $s$-th
  first-level group is in the range of
  $$
  \left[\fdpN  T_1- \GdT \sqrt{T_1\log(T)}, \fdpN  T_1 + \GdT \sqrt{T_1\log(T)}\right],
  $$
  where $\fdpN $ is the expected number of arm $a$ pulls in a $\ALGG$ run
  of length $\GdT$. Then $\Pr[W_1] \geq 1- \frac{4\NG}{T^2}$.
\end{lemma}

\begin{proof}[Proof of Lemma~\ref{3levelw1}]
  For the $s$-th first-level group, define $W_1^{a,s}$ to be the event
  that the number of arm $a$ pulls in the $s$-th first-level group is
  between $\fdpN T_1- \GdT \sqrt{T_1\log(T)}$ and
  $\fdpN T_1 + \GdT \sqrt{T_1\log(T)}$. By Lemma~\ref{lem:t1runs}
\[
\Pr[W_1^{a,s}] \geq 1-2\exp(-2\log(T)) \geq 1-2/T^2.
\]
Let $W_1$  be the intersection of all these events (i.e.
$W_1 = \bigcap_{a,s}W_1^{a,s}$). By union bound, we have
\[
\Pr[W_1] \geq 1- \frac{4\NG}{T^2}.
\]
\end{proof}

To state the events, it will be useful to think of a
hypothetical reward tape $\cT^1_{s, a}$ of length $T$ for each
group $s$ and arm $a$, with each cell independently sampled from
$\cD_a$.  The tape encodes rewards as follows: the $j$-th time arm $a$
is chosen by the group $s$ in the first level, its reward is taken
from the $j$-th cell in this arm's tape. The following result
characterizes the concentration of the mean rewards among all
consecutive pulls among all such tapes, which follows from Chernoff
bound and union bound.

\begin{lemma}[Concentration of empirical means in the first level]\label{3levelw2}
  For any $\z_1, \z_2\in [T]$ such that $\z_1 < \z_2$, $s\in [\NG]$, and
  $a\in \{1,2\}$, let $W_2^{s,a,\z_1,\z_2}$ be the event that the mean
  among the cells indexed by $\z_1, (\z_1+1), \ldots, \z_2$ in the tape
  $\cT^1_{a, s}$ is at most $\sqrt{\frac{2\log(T)}{\z_2-\z_1+1}}$ away
  from $\mu_a$.  Let $W_2$ be the intersection of all these events
  (i.e.  $W_2 = \bigcap_{a,s,\z_1,\z_2} W_2^{s,a,\z_1,\z_2}$). Then
  \[
    \Pr[W_2] \geq 1- \frac{4\NG}{T^2}.
  \]
\end{lemma}


\begin{proof}[Proof of Lemma~\ref{3levelw2}]
  By Chernoff bound,
\[
\Pr[W_2^{s,a,\z_1,\z_2}] \geq 1 - 2\exp(-4\log(T)) \geq 1- 2/T^4.
\]
By union bound, we have
\[
\Pr[W_2] \geq 1- \frac{4\NG}{T^2}.
\]
\end{proof}

Our policy also relies on the anti-concentration of the empirical
means in the first round. We show that for each arm $a\in \{1, 2\}$,
there exists a group $s_a$ such that the empirical mean of $a$ is
slightly above $\mu_a$, while the other arm $(3 - a)$ has empirical
mean slightly below $\mu_{(3-a)}$. This event is crucial for inducing
agents in the second level to explore both arms when the their
mean rewards are indistinguishable after the first level.


\begin{lemma}[Co-occurence of high and low deviations in this first level]\label{3levelw4}
  For any group $s\in [\NG]$, any arm $a$, let $\tilde\mu_{a,s}$ be the
  empirical mean of the first $\fdpN  T_1$ cells in tape $\cT^1_{a, s}$.
  Let $W_3^{s,a,\text{high}}$ be the event
  $\tilde\mu_{a, s} \geq \mu_a + 1/\sqrt{\fdpN  T_1}$ and let
  $W_3^{s,a,\text{low}}$ be the event that
  $\tilde\mu_{a, s} \leq \mu_a - 1/\sqrt{\fdpN  T_1}$.  Let $W_3$ be the
  event that for every $a\in \{1, 2\}$, there exists a group
  $s_a\in [\NG]$ in the first level such that both $W_3^{s_a,a,\text{high}}$
  and $W_3^{s_a,3-a,\text{low}}$ occur. Then
  \[
    \Pr[W_3]\geq 1 -2 /T.
  \]
\end{lemma}



\begin{proof}[Proof of Lemma~\ref{3levelw4}]
By Berry-Esseen Theorem and
  $\mu_a \in [1/3,2/3]$, we have for any $a$,
\[
\Pr[W_3^{s,a,high}] \geq (1-\Phi(1/2)) - \frac{5}{\sqrt{\fdpN T_1}} > 1/4.
\]
The last inequality follows when $T$ is larger than some constant.
Similarly we also have
\[
\Pr[W_3^{s,a,low}] > 1/4.
\]
Since $W_3^{s,a,high}$ is independent with $W_3^{s,3-a,low}$, we have
\[
\Pr[W_3^{s,a,high} \cap W_3^{s,3-a,low}] =\Pr[W_3^{s,a,high}] \cdot  \Pr[W_3^{s,3-a,low}]>(1/4)^2 = 1/16.
\]
Notice that $(W_3^{s,a,high} \cap W_3^{s,3-a,low})$ are independent
across different $s$'s. By union bound, we have
\[
\Pr[W_3] \geq 1- 2(1-1/16)^\NG \geq 1 -2 /T.
\]
\end{proof}



Lastly, we will condition on the event that the empirical means of
both arms are concentrated around their true means in any prefix of
their pulls. This guarantees that the policy obtains an accurate
estimate of rewards for both arms after aggregating all the data in
the first two levels.
% We will again state the event using a hypothetical reward tape
% $\cT^2_a$ for each arm $a$: the $j$-th time arm $a$ is chosen in the
% two levels, its reward is taken from the $j$-th cell in $\cT^2_a$.


%\swcomment{might want to swap out $W_3$ and $W_4$} \nicomment{yes please!}
%\jmcomment{let's follow the order of levels}

\begin{lemma}[Concentration of empirical means in the first two
  levels]\label{3levelw3}
  With probability at least $1 - \frac{4}{T^3}$, the following event
  $W_4$ holds: for all $a\in \{1, 2\}$ and $\z \in [N_{T, a}]$, the
  empirical means of the first $\z$
  %\nicomment{what is $t$ here; you mean $m$? in which case please replace $m$ with $t$ in the conditioning and bound}
  arm $a$ pulls is at most
  $\sqrt{\frac{2\log(T)}{\z}}$ away from $\mu_a$, where $N_{T, a}$ is
  the total number of arm $a$ pulls by the end of $T$ rounds.
\end{lemma}


\begin{proof}[Proof of Lemma~\ref{3levelw3}]
  For any arm $a$, let's imagine a hypothetical tape of length $T$,
  with each cell independently sampled from $\cD_a$. The tape encodes
  rewards of the first two levels as follows: the $j$-th time arm $a$
  is chosen in the first two levels, its reward is taken from the
  $j$-th cell in the tape. Define $W_4^{a,\z}$ to be the event that the
  mean of the first $t$ pulls in the tape is at most
  $\sqrt{\frac{2\log(T)}{\z}}$ away from $\mu_a$. By Chernoff bound,
\[
\Pr[W_4^{a,\z}] \geq 1 - 2\exp(-4\log(T)) \geq 1- 2/T^4.
\]
By union bound, the intersection of all these events has probability
at least:
\[
\Pr[W_4] \geq 1- \frac{4}{T^3}.
\]
\end{proof}


Let $W = \bigcap_{i=1}^4 W_i$ be the intersection of all 4
events.  By union bound, $W$ occurs with probability $1-O(1/T)$. Note
that the regret conditioned on $W$ not occurring is at most
$O(1/T) \cdot T = O(1)$, so it suffices to bound the regret conditioned on $W$.








\subsection{Case Analysis}
\label{sec:3level-case}

Now we assume the intersection $W$ of events $W_1,\cdots,W_4$ happens. We will
first provide some helper lemmas for our case analysis.

\begin{lemma}
  For the $s$-th first-level group and arm $a$, define
  $\bar{\mu}_a^{1,s}$ to be the empirical mean of arm $a$ pulls in
  this group. If $W$ holds, then
  \[
    |\bar{\mu}_a^{1,s} - \mu_a| \leq \sqrt{\frac{4\log(T)}{\fdpN T_1}}.
  \]
\end{lemma}

\begin{proof}
  The events $W_1$ and $W_2^{a,s,1,\z}$ for
  $\z = \fdpN T_1- \GdT \sqrt{T_1\log(T)},...,\fdpN T_1 + \GdT
  \sqrt{T_1\log(T)}$ together imply that
\[
|\bar{\mu}_a^{1,s} - \mu_a| \leq \sqrt{\frac{2\log(T)}{\fdpN T_1- \GdT \sqrt{T_1\log(T)}}} \leq \sqrt{\frac{4\log(T)}{\fdpN T_1}}.
\]
The last inequality holds when $T$ is larger than some constant.
\end{proof}


\begin{lemma}
  For each arm $a$, define $\bar{\mu}_a$ to be the empirical mean of
  arm $a$ pulls in the first two levels. If $W$ holds, then
  \[
    |\bar{\mu}_a - \mu_a| \leq \sqrt{\frac{4\log(T)}{\NG \fdpN T_1}} .
  \]
Furthermore, if there are at least $T_2$ pulls of arm $a$ in the first two levels,
\[
|\bar{\mu}_a-\mu_a| \leq \sqrt{\frac{2\log(T)}{T_2}}.
\]
\end{lemma}

\begin{proof}
The events $W_1$ and $W_4^{a,\z}$ for $\z \geq  (\fdpN T_1- \GdT \sqrt{T_1\log(T)})\NG$ together imply that
  \[
    |\bar{\mu}_a - \mu_a| \leq \sqrt{\frac{2\log(T)}{\NG\left(\fdpN T_1- \GdT \sqrt{T_1\log(T)}\right)}} \leq \sqrt{\frac{4\log(T)}{\NG \fdpN T_1}} .
\]
The last inequality holds when $T$ is larger than some constant.
\end{proof}


\begin{lemma}\label{lem:luck}
  For the $s$-th first-level group and arm $a$, define
  $\bar{\mu}_a^{1,s}$ to be the empirical mean of arm $a$ pulls in
  this group. For each $a \in \{1,2\}$, there exists a group $s_a$
  such that
\[
\bar{\mu}_a ^{1,s_a} > \mu_a + \frac{1}{4\sqrt{\fdpN T_1}} \quad \mbox{and, } \quad
\bar{\mu}_{3-a} ^{1,s_a} < \mu_{3-a}   - \frac{1}{4\sqrt{\fdpN[3-a] T_1}}.
\]
\end{lemma}




\begin{proof}% [Proof of Lemma~\ref{lem:luck}]
  For each $a \in \{1,2\}$, $W_3$ implies that there exists $s_a$ such
  that both $W_3^{s_a,a,high}$ and $W_3^{s_a,3-a,low}$ happen.  The
  events $W_3^{s_a,a,high}$, $W_1$, $W_2^{s_a,a,\z, \fdpN T_1}$
  for $\z = \fdpN T_1- \GdT \sqrt{T_1\log(T)}+1, ...,\fdpN T_1-1$ and
  $W_2^{s_a,a,\fdpN T_1,\z}$ for
  $\z= \fdpN T_1,...,\fdpN T_1+ \GdT \sqrt{T_1\log(T)}$ together imply that
\begin{align*}
\bar{\mu}_a ^{1,s_a} &\geq \mu_a + \left(\fdpN T_1 \cdot \frac{1}{\sqrt{\fdpN T_1}} - \GdT \sqrt{T_1\log(T)} \cdot \sqrt{\frac{2\log(T)}{ \GdT \sqrt{T_1\log(T)}}} \right) \cdot \frac{1}{\fdpN T_1+ \GdT \sqrt{T_1\log(T)}} \\
&> \mu_a + \frac{1}{4\sqrt{\fdpN T_1}}.
\end{align*}
The second to the last inequality holds when $T$ is larger than some constant.
Similarly, we also have
\[
\bar{\mu}_{3-a} ^{1,s_a} < \mu_{3-a}   - \frac{1}{4\sqrt{\fdpN[3-a] T_1}}.
\]
This completes the proof.
\end{proof}


Now we proceed to the case analysis.


\begin{proof}[Proof of Lemma~\ref{3levelbigcase} (Large gap case)]
  Observe that for any group $s$ in the first level, the empirical
  means satisfy
\[
\bar{\mu}_1^{1,s} - \bar{\mu}_2^{1,s} \geq \mu_1 -\mu_2 - \sqrt{\frac{4\log(T)}{\fdpN[1]T_1}} - \sqrt{\frac{4\log(T)}{\fdpN[2]T_1}} \geq  \sqrt{\frac{4\log(T)}{\fdpN[1]T_1}} + \sqrt{\frac{4\log(T)}{\fdpN[2]T_1}}.
\]


For any agent $t$ in the $s$-th second-level group, by Assumption \ref{ass:embehave}, we have
\begin{align*}
\hat{\mu}_1^t - \hat{\mu}_2^t &>\bar{\mu}_1^{1,s} - \bar{\mu}_2^{1,s} - \frac{\estC}{\sqrt{\fdpN[1]T_1/2}} - \frac{\estC}{\sqrt{\fdpN[2]T_1/2}}\\
&\geq  \sqrt{\frac{4\log(T)}{\fdpN[1]T_1}} + \sqrt{\frac{4\log(T)}{\fdpN[2]T_1}}- \frac{\estC}{\sqrt{\fdpN[1]T_1/2}} - \frac{\estC}{\sqrt{\fdpN[2]T_1/2}} > 0
\end{align*}
Therefore, we know agents in the $s$-th second-level group will all pull arm 1.

Now consider the agents in the third level group. Recall $\bar{\mu}_a$
is the empirical mean of arm $a$ in the history they see. We have
\[
\bar{\mu}_1 - \bar{\mu}_2 \geq \mu_1 -\mu_2 - \sqrt{\frac{4\log(T)}{\NG\fdpN[1]T_1}} - \sqrt{\frac{4\log(T)}{\NG\fdpN[2]T_1}} \geq  \sqrt{\frac{4\log(T)}{\fdpN[1]T_1}}
+ \sqrt{\frac{4\log(T)}{\fdpN[2]T_1}}.
\]
Similarly as above, by Assumption \ref{ass:embehave}, we know
$\hat{\mu}_1^t - \hat{\mu}_2^t > 0$ for any agent $t$ in the third
level. Therefore, the agents in the third-level group will all pull
arm 1.  \OMIT{Therefore the expected regret is at most
  $\NG \GdT T_1 = O(T^{4/7} \log^{6/7}(T))$.}
\end{proof}


\begin{proof}[Proof of Lemma~\ref{3levelmedium} (Medium gap case)]
  % $2\left(\sqrt{\frac{4\log(T)}{\NG\fdpN[1]T_1}} +
  %   \sqrt{\frac{4\log(T)}{\NG\fdpN[2]T_1}}\right) \leq \mu_1-\mu_2 <
  % 2\left(\sqrt{\frac{4\log(T)}{\fdpN[1]T_1}} +
  %   \sqrt{\frac{4\log(T)}{\fdpN[2]T_1}}\right)$.
  Recall $\bar{\mu}_a$ is
  the empirical mean of arm $a$ in the first two levels. We have
\[
\bar{\mu}_1 - \bar{\mu}_2 \geq \mu_1 -\mu_2 - \sqrt{\frac{4\log(T)}{\NG\fdpN[1]T_1}} - \sqrt{\frac{4\log(T)}{\NG\fdpN[2]T_1}} \geq  \sqrt{\frac{4\log(T)}{\NG\fdpN[1]T_1}}
+ \sqrt{\frac{4\log(T)}{\NG\fdpN[2]T_1}}.
\]
For any agent $t$ in the third level, by Assumption \ref{ass:embehave}, we have
\begin{align*}
\hat{\mu}_1^t - \hat{\mu}_2^t &>\bar{\mu}_1 - \bar{\mu}_2 - \frac{\estC}{\sqrt{\NG\fdpN[1]T_1/2}} - \frac{\estC}{\sqrt{\NG\fdpN[2]T_1/2}}\\
&\geq  \sqrt{\frac{4\log(T)}{\NG\fdpN[1]T_1}} + \sqrt{\frac{4\log(T)}{\NG\fdpN[2]T_1}}- \frac{\estC}{\sqrt{\NG\fdpN[1]T_1/2}} - \frac{\estC}{\sqrt{\NG\fdpN[2]T_1/2}}\\
 &> 0.
\end{align*}
So we know agents in the third-level group will all pull arm 1. \OMIT{Therefore the expected regret is at most
\[
(\NG \GdT T_1 + \NG T_2) \cdot 2\left(\sqrt{\frac{4\log(T)}{\fdpN[1]T_1}}
+ \sqrt{\frac{4\log(T)}{\fdpN[2]T_1}}\right) = O(T^{4/7} \log^{6/7}(T))
\]
}
\end{proof}


\begin{proof}[Proof of Lemma~\ref{3levelsmallcase} (Small gap case)]
  % $ 3\sqrt{\frac{2\log(T)}{T_2}} < \mu_1-\mu_2 <
  % 2\left(\sqrt{\frac{4\log(T)}{\NG\fdpN[1]T_1}} +
  %   \sqrt{\frac{4\log(T)}{\NG\fdpN[2]T_1}}\right)$.
  In this case, we need both arms to be pulled at least $T_2$ rounds
  in the second level. For every arm $a$, consider the $s_a$-th
  second-level group, with $s_a$ given by Lemma~\ref{lem:luck}. We
  have
\begin{align*}
\bar{\mu}_a^{1,s_a} - \bar{\mu}_{3-a}^{1,s_a} &> \mu_a + \frac{1}{4\sqrt{\fdpN T_1}} -\mu_{3-a} +\frac{1}{4\sqrt{\fdpN[3-a]T_1}} \\
&> \frac{1}{4\sqrt{\fdpN[1]T_1}}+ \frac{1}{4\sqrt{\fdpN[2]T_1}} - 2\left(\sqrt{\frac{4\log(T)}{\NG\fdpN[1]T_1}}
+ \sqrt{\frac{4\log(T)}{\NG\fdpN[2]T_1}}\right) \\
&\geq \frac{1}{8\sqrt{\fdpN[1]T_1}}+ \frac{1}{8\sqrt{\fdpN[2]T_1}}.
\end{align*}
For any agent $t$ in the $s_a$-th second-level group, by Assumption \ref{ass:embehave}, we have
\begin{align*}
\hat{\mu}_a^t - \hat{\mu}_{3-a}^t &>\bar{\mu}_a^{1,s_a} - \bar{\mu}_{3-a}^{1,s_a} - \frac{\estC}{\sqrt{\fdpN[1]T_1/2}} - \frac{\estC}{\sqrt{\fdpN[2]T_1/2}}\\
&\geq   \frac{1}{8\sqrt{\fdpN[1]T_1}}+ \frac{1}{8\sqrt{\fdpN[2]T_1}}- \frac{\estC}{\sqrt{\fdpN[1]T_1/2}} - \frac{\estC}{\sqrt{\fdpN[2]T_1/2}}\\
 &> 0.
\end{align*}
So we know agents in the $s_a$-th second-level group will all pull arm $a$. Therefore in the first two levels, both arms are pulled at least $T_2$ times. Now consider the third-level. We have
\[
\bar{\mu}_1 - \bar{\mu}_2  \geq \mu_1 -\mu_2 - 2\sqrt{\frac{2\log(T)}{T_2}} \geq \sqrt{\frac{2\log(T)}{T_2}}.
\]
Similarly as above, by Assumption \ref{ass:embehave}, we know $\hat{\mu}_1^t - \hat{\mu}_2^t > 0$ for any agent $t$ in the third level. So we know agents in the third-level group will all pull arm 1.\OMIT{
Therefore the expected regret is at most
\[
(\NG \GdT T_1 + \NG T_2) \cdot 2\left(\sqrt{\frac{4\log(T)}{\NG\fdpN[1]T_1}}
+ \sqrt{\frac{4\log(T)}{\NG\fdpN[2]T_1}}\right) \leq O(T^{4/7} \log^{6/7}(T))
\]
}
\end{proof} 
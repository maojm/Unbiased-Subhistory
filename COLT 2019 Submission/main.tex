\documentclass[anon,12pt]{colt2019}
\usepackage{fullpage}
\usepackage{mathrsfs}
\usepackage{float}
% ======    PACKAGES
\usepackage{slivkins-setup}
%,slivkins-theorems}
%\usepackage{amsmath, amsfonts, amssymb, amsthm, amsbsy, amscd, bm, bbm}
%\usepackage{amsbsy,amscd, bm, bbm}
\usepackage{kpfonts}
\usepackage{array}
%\usepackage{booktabs}
\usepackage{graphicx}
%\usepackage[small,bf]{caption}
%\setlength{\captionmargin}{30pt}
%\usepackage{subcaption}
%\captionsetup[sub]{margin=10pt,font=small}
\usepackage{color}
\usepackage{ifthen}
\usepackage{xspace}
\usepackage[noend]{algorithmic}
\usepackage{algorithm}
%\usepackage[colorlinks,citecolor={black},urlcolor={black},linkcolor={black}]{hyperref}
\usepackage{url}
%\usepackage{tocbibind}
\usepackage{enumerate}
\usepackage{mdframed}
\usepackage{comment}
\usepackage{tikz}
\usetikzlibrary{decorations.pathreplacing}
\usepackage{cleveref}

\newtheorem{assumption}[theorem]{Assumption}
\newtheorem{construction}[theorem]{Construction}
\newtheorem{claim}[theorem]{Claim}
\DeclareMathOperator*{\argmin}{argmin}

% a very useful package for edits and comments, from David Kempe (USC)
%\usepackage{color-edits}
%\usepackage[suppress]{color-edits}  % use this to suppress the package
%\addauthor{as}{red}      % as for Alex
%\addauthor{jm}{blue}     % jm for Jieming
%\addauthor{ni}{magenta}     % ni for Nicole
%\addauthor{sw}{brown}     % sw for Steven
% e.g. for Alex, provides \asedit{}, \ascomment{} and \asdelete{}.


%%%%%%%%%%%%%%%%%%%%%%%%%%%%%%%%%%
% \newtheorem{theorem}{Theorem}[section]
% \newtheorem{corollary}[theorem]{Corollary}
% \newtheorem{conjecture}[theorem]{Conjecture}
% \newtheorem{proposition}[theorem]{Proposition}
% \newtheorem{definition}[theorem]{Definition}
% \newtheorem{lemma}[theorem]{Lemma}
% \newtheorem{remark}{Remark}[section]
% \newtheorem{claim}{Claim}[section]
% \newtheorem{example}{Example}[section]


\newcommand{\term}[1]{\ensuremath{\mathtt{#1}}\xspace}
\newcommand{\thread}{\term{thread}}
\newcommand{\ALGG}{full-disclosure path }
\newcommand{\ALGPG}{\term{Parallel Greedy}}

% Alex's notation!
\newcommand{\SubH}[1]{\mathcal{H}_{#1}} % subhistory
\newcommand{\AnonSubH}[1]{H^{\texttt{anon}}_{#1}} % subhistory

% new notation for FDPs (full-disclosure paths)
\newcommand{\fdp}{\term{FDP}}
\newcommand{\fdpL}{L^\fdp_K} % sufficient length of greedy path
\newcommand{\fdpP}{p^\fdp_K} % success prob for greedy path
\newcommand{\fdpN}[1][a]{N^\fdp_{K,#1}}
    % Exp #pulls for arm $a$ in greedy path of length L_K.

% new notation for low-deviation assumption on agents' reward estimates
\newcommand{\estN}{N_{\term{est}}}
\newcommand{\estC}{C_{\term{est}}}

% notation for 3level section
\newcommand{\conf}[1]{\mathtt{conf}\left(#1\right)}
\newcommand{\event}[1]{\ensuremath{\mathtt{event}_{#1}}\xspace}
\newcommand{\cG}{\mathcal{G}}

% Greedy notation
\def\GdT{L^\fdp_K} % sufficient length of greedy path
\def\GdP{p^\fdp_K} % success prob for greedy path

\def\2LEVEL{two-level policy}

\newcommand{\DT}{\mD} % Alex' notation for distribution over types.


%number of groups
\def\NG{\sigma}

\def\DKL{\textbf{D}_{\term{KL}}}
\def\D{\mathbb{D}}
\def\E{\mathbb{E}}
\def\reg{\mathrm{Reg}}
\def\A{\mathcal{A}}
\def\M{\mathcal{M}}
\def\S{\mathcal{S}}
\def\X{\mathcal{X}}
\def\EX{\term{EX}}
\def\Ds{*}
\def\EE{\mathcal{E}}
\def\varTheta{\bold{\Theta}}
\def\varOmega{\bold{\Omega}}
\def\cD{\mathcal{D}}
\def\cT{\mathcal{T}}
\def\Ber{\emph{Ber} }

\def\pmn{\hat{\mu}}
\def\emn{\bar{\mu}}
%\def\q_a{\fdpN}
%\def\ell{l}

%index on the tape
\def\z{\tau}
\title{Incentivizing Exploration with Unbiased Histories}
\usepackage{times}

% Authors with different addresses:
\coltauthor{
\Name{Nicole Immorlica} \Email{nicimm@microsoft.com}\\
\addr Microsoft Research, Cambridge, MA. 
\and
\Name{Jieming Mao} \Email{maojm517@gmail.com}\\
\addr University of Pennsylvania, Philadelphia, PA.
\and
\Name{Aleksandrs Slivkins}  \Email{slivkins@microsoft.com}\\
\addr Microsoft Research, New York, NY.
\and
\Name{Zhiwei Steven Wu} \Email{zsw@umn.edu}\\
\addr University of Minnesota, Minneapolis, MN.
}
\begin{document}
\begin{titlepage}
\maketitle

\thispagestyle{empty}
\begin{abstract}
In a social learning setting, there is a set of actions, each of which has a payoff that depends on a hidden state of the world. A sequence of agents each chooses an action with the goal of maximizing payoff given estimates of the state of the world.  A disclosure policy tries to coordinate the choices of the agents by sending messages about the history of past actions.  The goal of the algorithm is to minimize the regret of the action sequence.

In this paper, we study a particular class of disclosure policies that use messages, called {\em unbiased subhistories}, consisting of the actions and rewards from by a subsequence of past agents, where the subsequence is chosen ahead of time. One trivial message of this form contains the full history; a disclosure policy that chooses to use such messages risks inducing herding behavior among the agents and thus has regret linear in the number of rounds.  Our main result is a disclosure policy using unbiased subhistories that obtains regret $\tilde{O}(\sqrt{T})$.  We also exhibit simpler policies with higher, but still sublinear, regret.  These policies can be interpreted as dividing a sublinear number of agents into constant-sized focus groups, whose histories are then fed to future agents.
\end{abstract}
\begin{keywords}%
  incentivizing exploration, multi-armed bandit, social learning
\end{keywords}
\end{titlepage}
\section{Introduction}
\label{sec:intro}



\nicomment{a proposed abstract since i don't have the token on main.tex:}

In a social learning setting, there is a set of actions, each of which has a payoff that depends on a hidden state of the world.  A sequence of agents each choose an action with the goal of maximizing payoff given estimates of the state of the world.  A recommendation algorithm tries to coordinate the choices of the agents by sending messages about the history of past actions.  The goal of the algorithm is to minimize the regret of the action sequence.

In this paper, we study a particular class of recommendation algorithms that use messages, called {\em unbiased subhistories}, consisting of the actions and rewards chosen by a subsequence of past agents.  One trivial message of this form contains the full history; a recommendation algorithm that chooses to use such messages risks inducing herding behavior among the agents and thus has regret linear in the number of rounds.  Our main result is a recommendation algorithm using unbiased subhistories that obtains regret $\tilde{O}(\sqrt{T})$.  We also exhibit simpler policies with higher, but still sublinear, regret.  These policies can be interpreted as dividing a sublinear number of agents into constant-sized focus groups, whose histories are then fed to future agents.

\nicomment{and now for the intro to the intro.}

In {\em incentivized exploration}, a recommendation algorithm presents a sequence of agents 

In social learning, a sequence of individuals choose actions based on observations of past choices.  Incentivized exploration algorithms coordinate social learning, hiding inducing sufficient exploration





\ascomment{Nicole, it is all yours ..}

Recommendation systems are ubiquitous, providing recommendations for movies (\eg  Netflix), products (\eg  Amazon), restaurants (\eg  Yelp), vacations (\eg Tripadvisor), etc.  A typical recommendation system elicits user feedback about their experiences, and aggregates this feedback in order to provide better recommendations in the future. Thus, each user she consumes information from the previous users (indirectly, \eg via recommendations), and produces new information (\eg  a review) that benefits future users. This dual role creates a three-way tension between exploration, exploitation, and users' incentives.
A social planner would balance ``exploration" of insufficiently known alternatives and ``exploitation" of the information acquired so far. Designing algorithms to trade off these two objectives is a well-researched subject in machine learning and operations research. However, a given user who decides to ``explore" typically suffers all the downside of this decision, whereas the upside (improved recommendations) is spread over many users in the future. Therefore, users' incentives are skewed in favor of exploitation. As a result, observations may be collected at a slower rate, and may suffer from selection bias (\eg  ratings of a particular movie may mostly come from people who like this type of movies). Moreover, in some natural but idealized models (\eg  \cite{Kremer-JPE14,ICexploration-ec15}), there are simple  examples when optimal recommendations are never found because the corresponding actions are never taken.

Thus, we have a problem of \emph{incentivizing exploration}. Providing monetary incentives can be financially or technologically unfeasible, and relying on voluntary exploration can lead to selection biases. A recent line of work, started by \cite{Kremer-JPE14}, relies on the inherent \emph{information asymmetry} between the recommendation system and a user. These papers posit a simple model, termed \emph{Bayesian Exploration} in \cite{ICexplorationGames-ec16}. The recommendation system is a ``principal" that interacts with self-interested ``agents" who arrive one by one. Each agent needs to make a decision: take an action from a given set of alternatives. The principal sends a \emph{message} to the agent according to some ``disclosure policy", \eg issues a recommendation. Then the agent chooses an action, and both the agent and the principal observe the outcome.  Crucially, the principal does not control the agent's decision. The problem is to design the disclosure policy for the principal that learns over time to issue messages so as to incentivize the agents to balance exploration and exploitation in a socially optimal way.
A single round of this model is a version of a well-known ``Bayesian Persuasion game" \cite{Kamenica-aer11}.

\ascomment{ ... up to here.}

\xhdr{Our scope.} Prior work on incentivizing exploration, with or without monetary incentives, achieves much progress (more on this in ``related work"), but relies heavily on the standard assumptions of Bayesian rationality and the ``power to commit" (\ie users trust that the principal actually implements the policy that it claims to implement). However, these assumptions appear quite problematic in the context of recommendation systems. In particular, much of the prior work suggests disclosure policies that merely recommend an action to each agent, without any other supporting information, and moreover recommend exploratory actions to some randomly selected users. This works out extremely well in theory, but it is very unclear whether users would trust the principal to implement the stated policy and, even if they do, whether they would react to it rationally. Several issues are in play: to wit, whether the principal intentionally uses a different disclosure policy than the claimed one (\eg because its incentives are not quite aligned with the users'), whether the principal correctly implements the policy that it wants to implement, whether the users trust the principal to make correct inferences on their behalf, and whether they find it acceptable that they may be singled out for exploration. Furthermore, regardless of how the users react to such disclosure policies, they may prefer not to be subjected to them, and leave the system.

We strive to design disclosure policies which mitigate these issues and (still) incentivize a good balance between exploration and exploitation. While some assumptions on human behavior are unavoidable, we are looking for a class of disclosure policies for which we can make plausible behavioral assumptions. Then we arrive at a concrete mathematical problem: design policies from this class so as to optimize performance, \ie  the induced explore-exploit tradeoff. Our goal in terms of performance is to approach the performance of the social planner.

\xhdr{Our model.}
For the sake of intuition, let us revisit the \emph{full-disclosure policy} that reveals the full history of observations from the previous users. We interpret it as the ``gold standard": we posit that users would trust such policy, even if they cannot verify it. Unfortunately, the full-disclosure policy is not good for our purposes, essentially because rational users would \emph{exploit} rather than \emph{explore}. However, what if a disclosure policy reveals the outcomes for every other agent, rather than the outcomes for all agents? We posit that users would trust such policy, too. Given a large volume of data, we posit that users would not be too unhappy with having access to only a fraction this data. A crucial aspect of our intuition here is that the ``subhistory" revealed to a given user comes from a subset of previous users that is chosen in advance, without looking at what happens during the execution. In particular, the subhistory is not ``biased", in the sense that the disclosure policy cannot subsample the observations in favor of a particular action.

With this intuition in mind, we define the class of \emph{unbiased-subhistory policies}: disclosure policies that reveal, to each arriving agent $t$, a subhistory  consisting of the outcomes for a subset $S_t$ of previous agents, where $S_t$ is chosen ahead of time. Further, we impose a transitivity property: if $t' \in S_t$, for some previous agent $t'$, then $S_{t'}\subset S_t$. So, agent $t$ has all information that agent $t'$ had at the time she chose her action. In particular, agent $t$ does not need to second-guess which message has caused agent $t'$ to make choose that action.

Following much of the prior work on incentivizing exploration, we do not attempt to model heterogenous agent preferences and non-stationarity. Formally, we assume that the expected reward of taking a given action $a$, denoted $\mu_a$, is the same for all agents, and does not change over time. Then the crucial parameter of interest, for a given action $a$, are the number of samples $N_a$ and the empirical mean reward $\bar{\mu}_a$ in the observed subhistory. We consider a flexible model of agent response: for each action $a$ an agent forms an estimate $\hat{\mu}_a$ of the mean reward $\mu_a$, roughly following $\bar{\mu}_a$ but taking into account the uncertainty due to a small number of samples, and chooses an action with a largest reward estimate. We allow the reward estimates to be arbitrary otherwise, and not known to the principal.

\xhdr{Regret.} We measure the performance of a disclosure policy in terms of \emph{regret}, a standard notion from the literature on multi-armed bandits. Regret is defined as the difference in the total expected reward between the best fixed action and actions induced by the policy. Regret is typically studied as a function of the time horizon $T$, which in our model is the number of agents. For multi-armed bandits, $o(T)$ regret bounds are deemed non-trivial, and $O(\sqrt{T})$ regret bounds are optimal in the worst case. Regret bounds that depend on a particular problem instance are also considered. A crucial parameter then is the \emph{gap} $\Delta$, the difference between the best and second best expected reward. One can achieve $O(\tfrac{1}{\Delta}\; \log T)$ regret rate, without knowing the $\Delta$.

\xhdr{Our results and discussion.}
Our main result is a transitive, unbiased-subhistory policy which attains near-optimal $\tilde{O}(\sqrt{T})$ regret rate for a constant number of actions. This policy also obtains the optimal instance-dependent regret rate
    $\tilde{O}(\tfrac{1}{\Delta})$
for problem instances with gap $\Delta$, without knowing the $\Delta$ in advance. In particular, we match the regret rate achieved for incentivizing exploration with unrestricted disclosure policies \cite{ICexploration-ec15-working}.

The main challenge is that the agents still follow exploitation-only behavior, just like they do for the full-disclosure policy, albeit based only on a portion of history. A disclosure policy controls the flow of information (who sees what), but not the \emph{content} of that information. 

The first step is to obtain any substantial improvement over the full-disclosure policy. We accomplish this with a relatively simple policy which runs the full-disclosure policy ``in parallel" on several disjoint subsets of agents,  collects all data from these runs and discloses it to all remaining agents. While any single run of the full-disclosure policy may get stuck on a suboptimal arm, having these parallel runs ensure that sufficiently many of them will ``get lucky" and provide some exploration. This simple policy achieves $\tilde{O}(T^{2/3})$ regret. Conceptually, it implements a basic bandit algorithm that explores uniformly for a pre-set number of rounds, then picks one arm for exploitation and stays with it for the remaining rounds. We think of this policy  as having two ``levels": Level 1 contains the parallel runs, and Level 2 is everything else. 

The next step is to implement \emph{adaptive exploration}, when exploration schedule is adapted to previous observations. This is needed to improve over the $\tilde{O}(T^{2/3})$ regret. We upgrade the simple two-level policy with a middle level, where agents receive the data collected in some (but not all) runs from the first level. Essentially, this new level provides exploration only if the gap $\Delta$ is small (and therefore more exploration is needed).

The main result extends this construction to multiple levels, connected in fairly intricate ways. \ascomment{add more}.



\ascomment{stable-ish text up to here.}


\xhdr{Related work.}
The problem of incentivizing exploration via information asymmetry was introduced in \cite{Kremer-JPE14}, under the Bayesian rationality and the (implicit) power-to-commit assumptions. The original problem -- essentially, a version of our model with unrestricted disclosure policies -- was largely resolved in \cite{Kremer-JPE14} and the subsequent work \citep{ICexploration-ec15,ICexplorationGames-ec16}. Several extensions were considered: to contextual bandits \citep{ICexploration-ec15}, to repeated games
\citep{ICexplorationGames-ec16}, and to social networks \citep{Bahar-ec16}.

Several other papers study related, but technically different models: same model with time-discounted utilities \citep{Bimpikis-exploration-ms17}; a version with monetary incentives \citep{Frazier-ec14} and moreover with heterogenous agents \citep{Kempe-colt18}; a version with a continuous information flow and a continuum of agents \citep{Che-13}; coordination of costly exploration decisions when they are separate from the ``payoff-generating" decisions \citep{Bobby-Glen-ec16,Annie-ec18-traps,Liang-ec18}. Scenarios with long-lived, exploring agents and no principal to coordinate them have been studied in \citep{Bolton-econometrica99,Keller-econometrica05}.

Full-disclosure policy, and closely related ``greedy" (exploitation-only) algorithm in multi-armed bandits, have been a subject of a recent line of work \citep{Sven-aistats18,kannan2018smoothed,bastani2017exploiting,externalities-colt18}.
A common theme is that the greedy algorithm performs well in theory, under  substantial assumptions on heterogeneity of the agents. Yet, it suffers $\Omega(T)$ regret in the worst case.%
\footnote{This is a well-known folklore result; \eg see \citep{CompetingBandits-itcs18} for a concrete example.}
\citet{Baransi-ecml14} use the ``greedy" algorithm on sub-sampled subhistories to achieve optimal regret rate for two arms. However, their result is insufficient for our motivation, as the subhistories are not \emph{unbiased} or \emph{transitive}, in the sense defined earlier. In particular, the sub-sampling scheme depends on the history, which may distort agents' incentives.

Exploration-exploitation tradeoff received much attention over the past decades, usually under the rubric of ``multi-armed bandits"; see  \citep{Bubeck-survey12% ,Gittins-book11
} for background.
Exploration-exploitation problems with incentives issues arise in other scenarios, including dynamic pricing
    \citep{KleinbergL03,BZ09,BwK-focs13},
dynamic auctions
    \citep{AtheySegal-econometrica13,DynPivot-econometrica10,Kakade-pivot-or13},
pay-per-click ad auctions
    \citep{MechMAB-ec09,DevanurK09,Transform-ec10-jacm},
and human computation
    \citep{RepeatedPA-ec14,Ghosh-itcs13,Krause-www13}.



%!TEX root = main.tex
\section{Model and Preliminaries}
\label{sec:model}
We study the multi-armed bandit problem in a social learning context, in which a principal faces a sequence of $T$ myopic agents. There is a set $\A$ of $K$ possible actions, a.k.a. \emph{arms}. At each round $t\in [T]$, a new agent $t$ arrives, receives a message $m_t$ from the principal, chooses an arm $a_t\in \A$, and collects a reward $r_t\in \{0,1\}$ that is immediately observed by the principal. The reward from pulling an arm $a\in \A$ is drawn independently from Bernoulli distribution $\cD_a$ with an unknown mean $\mu_a$. An agent does not observe anything from the previous rounds, other than the message $m_t$. The problem instance is defined by (known) parameters $K,T$ and the (unknown) tuple of mean rewards, $(\mu_a:\,a\in\A)$. We are interested in \emph{regret}, defined as
\begin{align}
\textstyle
  \reg(T)
  %= \reg(a_1, \ldots , a_T)
  = T \max_{a\in \A} \mu_a -
  \sum_{t\in [T]} \E[\mu_{a_t}].
\end{align}
The principal chooses messages $m_t$ according to an online algorithm called \emph{disclosure policy}, with a goal to minimize regret. We assume that mean rewards are bounded away from $0$ and $1$, to ensure sufficient entropy in rewards. For concreteness, we posit that
    $\mu_a\in [\tfrac13,\tfrac23]$.

\xhdr{Unbiased subhistories.}
The \emph{subhistory} for a subset of rounds $S\subset [T]$ is defined as 
\begin{align} \label{eq:model-subhistory}
    \SubH{S} = \left\{\; (s,a_s,r_s):\;s\in S \;\right\}.
\end{align}
Accordingly, $\SubH{[t-1]}$ is called the \emph{full history} at time $t$. 
The \emph{outcome} for agent $t$ is the tuple $(t,a_t,r_t)$.


We focus on disclosure policies of a particular form, where the message in each round $t$ is $m_t = \SubH{S_t}$ for some subset $S_t\subset [t-1]$. We assume that the subset $S_t$ is chosen ahead of time, before round $1$ (and therefore does not depend on the observations $\SubH{t-1}$). Such message is called \emph{unbiased subhistory}, and the resulting disclosure policy is called an \emph{unbiased-history policy}.

Further, we focus on disclosure policies that are \emph{transitive}, in the following sense:
\[ t'\in S_t \Rightarrow S_{t'}\subset S_t
    \qquad \text{for all rounds $t,t'\in [T]$}. \]
In words, if an agent observes the outcome for some previous agent, then she observes the entire message revealed to that agent.

A transitive unbiased-history policy can be represented as an undirected graph, where nodes correspond to rounds, and any two rounds $t'<t$ are connected if and only if $t'\in S_t$ and there is no intermediate round $t''$ with
    $t'\in S_{t''}$ and $t''\in S_t$.
This graph is henceforth called the \emph{information flow graph} of policy $\pi$, or \emph{info-graph} for short. We assume that it is common knowledge.

\xhdr{Agents' behavior.} Let us define agents' behavior in response to an unbiased-history policy. We posit that each agent $t$ uses its observed subhistory $m_t$ to form a reward estimate $\hat{\mu}_{t,a} \in [0,1]$ for each arm $a\in \A$, and chooses an arm with a maximal estimator. (Ties are broken according to an arbitrary rule that is the same for all agents.) The basic model is that $\hat{\mu}_{t,a}$ is the sample average for arm $a$ over the subhistory $m_t$, as long as it includes at least one sample for $a$; else, $\hat{\mu}_{t,a}\geq \tfrac13$.

We allow a much more permissive model, characterized by the following assumption:%
\footnote{In what follows, we make this assumption without further notice.}

%If an arm is not pulled in $H_t$, we will let $\bar{\mu}_a^t = 0$.

\begin{assumption}\label{ass:embehave}
Reward estimates are close to empirical averages. Let $N_{t,a}$ and $\bar{\mu}_{t,a}$ denote the number of pulls and the empirical mean reward of arm $a$ in subhistory $m_t$. Then for some absolute constant $N_0\in \N$ and $C_0=\tfrac{1}{16}$, and for all agents $t\in [T]$ and arms $a\in\A$ it holds that
\[
\text{if}\; N_{t,a} \geq N_0 
\quad\text{then}\quad
    \left|\hat{\mu}^t_a - \bar{\mu}^t_a \right| <
		\frac{C_0}{\sqrt{N_{t,a}}}.
\]
Also, 
    $\hat{\mu}^t_a\geq\tfrac13$ if $N_{t,a}=0$.
(NB: we make no assumption if $1\leq N_{t,a}<N_0$.)
\end{assumption}

\OMIT{ %%%%%%%%%%
The first assumption ensures that the reward distributions have sufficient entropy to induce natural exploration. We choose the bounded range $[1/3, 2/3]$ for the simplicity of our analysis, and it can further relaxed to $[1/C, (C-1)/C]$ for any constant $C$. The second assumption says that the estimates computed by the agents are well-behaved, and are close to the empirical estimates given by the sub-history, provided that the number of observations is sufficiently large.

We model agents with heterogeneity in their arm selections. In particular, there is an unknown distribution over the set of agent estimators satisfying Assumption~\ref{ass:embehave}.  Each agent $t$ indepedently draws an estimator from this distribution, uses it to calculate the mean reward estimates $\hat{\mu}_a^t$ for every arm $a$, and then chooses the arm $a_t$ with the highest estimate.
% \swcomment{why independently? can be be
%   adversarial?} \jmcomment{Actually current proofs don't work when it's adversarial. It's fairly annoying for the anti-concentration argument. In the proof, we need to use independent to show that the amount of agents pulling arm $a$ in the first level is concentrated to $T_1 q_a$.}

\begin{remark}
We emphasize that the agents we consider in this paper are {\em frequentists}.  Thus their estimators, which determine their behavior, take samples as inputs and not priors.  The estimators satisfying Assumption~\ref{ass:embehave} include that of the natural greedy frequentist, who always pulls the arm with the highest empirical mean.
\end{remark}
} %%%%%%%

\xhdr{Connection to multi-armed bandits.}
The special case when each message $m_t$ is an arm, and the $t$-th agent always chooses this arm, corresponds to a standard multi-armed bandit problem with IID rewards. Thus, regret in our problem can be directly compared to regret in the bandit problem with the same mean rewards $(\mu_a:\,a\in\A)$. Following the literature on bandits, we define the \emph{gap parameter} $\Delta$ as the difference between the largest and second largest mean rewards.%
\footnote{Formally, the second-largest mean reward is
    $\max_{a\in\A:\mu(a)<\mu^*} \mu(a)$,
where $\mu^* = \max_{a\in\A} \mu(a)$.
}
The gap parameter is not known to the principal (in our problem), or to the algorithm (in the bandit problem). Optimal regret rates for bandits with IID rewards are as follows \cite{bandits-ucb1,bandits-exp3,Lai-Robbins-85}: 
\begin{align}\label{eq:model-OptRegret}
\reg(T) \leq O\left(\min\left( 
    \sqrt{KT\log T)},\; \tfrac{1}{\Delta} \log T
    \right)\right).
\end{align}
This regret bound can only be achieved using \emph{adaptive exploration}: \ie when exploration schedule is adapted to the observations. A simple example of \emph{non}-adaptive exploration is the \emph{explore-then-exploit} algorithm which samples arms uniformly at random for the first $N$ rounds, for some pre-set number $N$, then chooses one arm and sticks with it till the end. More generally, 
\emph{exploration-separating} algorithms have a property that in each round $t$, either the choice of an arm does not depend on the observations so far, or the reward collected in this round is not used in the subsequent rounds. Any such algorithm suffers from $\Omega(T^{2/3})$ regret in the worst case.%
\footnote{The first explicit reference we are aware of is \cite{MechMAB-ec09,DevanurK09}, but this fact has been known in the community for much longer.}

\xhdr{Preliminaries.}
Throughout the paper, we will use the standard concentration and anti-concentration inequalities: respectively, Chernoff Bounds and Berry-Esseen Theorem. The former states that that a sum of independent random variables converges to its expectation quickly. The latter states that the CDF of an appropriately scaled average of IID random variables converges to the CDF of the standard normal distribution pointwise. In particular, the average strays far enough from its expectation with some guaranteed probability. The theorem statements are moved to Appendix~\ref{sec:prelim}.

We use $O_K(\cdot)$ notation to hide the dependence on parameter $K$, and $\tilde O(\cdot)$ notation to hide polylogarithmic factors. We denote $[T]= \{1,2 \LDOTS T\}$.



%%% Local Variables:
%%% mode: latex
%%% TeX-master: "main"
%%% End:


%%!TEX root = main.tex
\section{\ALGG: Revealing Full History}

\swedit{Before we describe our algorithms, we will first analyze a
  simple policy \ALGG that fully discloses the history over all
  previous rounds, and lets the agents make the ``greedy'' choices
  across all rounds. Even though the algorithm does not guarantee low
  regret, it will be used as a subroutine in our main algorithms.  In
  particular, we show that because of the randomness in the rewards,
  with constant probability, \ALGG will incentivize the agents to
  play each arm at least once.}

\begin{lemma}
\label{lem:greedy}
Under Assumption \ref{ass:embehave}, there exist two constants $\GdT = O_K(1)$
and $\GdP = \Omega_K(1)$ such that for any arm $a$, with probability
at least $\GdP$, \ALGG of $\GdT$ rounds pulls this arm at least
once.\footnote{The notations $O_k$ and $\Omega_K$ hide the dependence
  on the number of arms $K$.}
\end{lemma}

\begin{proof}
  Fix any arm $a$. Let $\GdT = (K-1) \cdot c_T + 1$ and
  $\GdP = (1/3)^{\GdT}$. \swedit{We will condition on the event that
    all realized rewards in $\GdT$ rounds are 0, which occurs with
    probability at least $\GdP$ under Assumption~\ref{ass:embehave}.}
  In this case, we want to show that arm $a$ is pulled at least
  once. We prove by contradiction. Suppose arm $a$ is not pulled. By
  the pigeonhole principle, we know that there is some other arm $a'$
  that is pulled at least $c_T + 1$ rounds. Let $t$ be the round in
  which arm $a'$ is pulled exactly $c_T + 1$ times. By Assumption
  \ref{ass:embehave}, we know
  \[
    \hat{\mu}_{a'}^t \leq 0 + c_m / \sqrt{c_T} < 1/3. 
  \]
  On the other hand, we have
  $\hat{\mu}_a^t \geq 1/3 > \hat{\mu}_{a'}^t$. This contradicts with
  the fact that in round $t$, arm $a'$ is pulled, instead of arm $a$.
  \swdelete{ In addition, we know that this case happens with
    probability at least $(1/3)^{\GdT} = \GdP$ as each arm's mean is
    at most $2/3$. To sum up, we know that with probability at least
    $\GdP$, \ALGG of $\GdT$ rounds pulls arm $a$ at least once.}
   \jmcomment{Agree with these changes.}
\end{proof}


\swedit{In our policies presented in the next sections, we will use
  \ALGG of $T_G$ rounds as a subroutine for initial exploration. For
  any $a$, let $q_a$ be the expected number of arm $a$ pulls in one
  run of \ALGG $T_G$.}


%%% Local Variables:
%%% mode: latex
%%% TeX-master: "main"
%%% End:


%!TEX root = main.tex
\section{Warm-up: full-disclosure paths}

We first consider a disclosure policy that reveals the full history in each round $t$, \ie $m_t=\SubH{t-1}$; we call it the \emph{full-disclosure policy}. The info-path for this policy is a simple path. We use this policy as a ``gadget" in our constructions. Hence, we formulate it slightly more generally:
  
\begin{definition}
A subset of rounds $S\subset [T]$ is called a \emph{full-disclosure path} in the  info-graph $G$ if the induced subgraph $G_S$ is a simple path, and it connects to the rest of the graph only through the terminal node $\max(S)$, if at all.
\end{definition}
  
We prove that for a constant number of arms, with constant probability, a full-disclosure path of constant length suffices to sample each arm at least once. We will build on this fact throughout.

\begin{lemma}\label{lem:greedy}
There exist numbers $\fdpL>0$ and $\fdpP>0$ that depend only on $K$, the number of arms, with the following property. Consider an arbitrary disclosure policy, and let $S\subset [T]$ be a full-disclosure path in its info-graph, of length $|S|\geq \fdpL$. Under Assumption \ref{ass:embehave}, with probability at least $\fdpP$, subhistory $\SubH{S}$ contains at least once sample of each arm $a$.
\end{lemma}

\begin{proof}[Proof Sketch]
\ascomment{Jieming, pls add a brief proof sketch if you can.}
\end{proof}


We provide a simple disclosure policy based on full-disclosure paths. The policy follows the ``explore-then-exploit'' paradigm. The ``exploration phase" comprises the first $N = T_1\cdot \fdpL$ rounds, and consists of $T_1$ full-disclosure paths of length $\fdpL$ each, where $T_1$ is a parameter. In the ``exploitation phase", each agent $t>N$ receives the full subhistory from exploration, \ie $m_t = \SubH{[N]}$. The info-graph for this disclosure policy is shown in Figure~\ref{fig:2level}.

\begin{figure}[H]
\centering
\begin{tikzpicture}
 \filldraw[fill=blue!20!white]
 (0,2)--(10,2)--(10,3)--(0,3)--cycle;
  \filldraw[fill=red!20!white]
  (0,0)--(1,0)--(1,1)--(0,1)--cycle;
  \draw (0.5,1)--(5,2);
  \filldraw[fill=red!20!white]
  (1,0)--(2,0)--(2,1)--(1,1)--cycle;
  \draw (1.5,1)--(5,2);
  \filldraw[fill=red!20!white]
  (2,0)--(3,0)--(3,1)--(2,1)--cycle;
  \draw(2.5,1)--(5,2);
  \filldraw[fill=red!20!white]
  (3,0)--(4,0)--(4,1)--(3,1)--cycle;
  \draw(3.5,1)--(5,2);
  \filldraw[fill=red!20!white]
  (9,0)--(10,0)--(10,1)--(9,1)--cycle;
  \draw(9.5,1)--(5,2);
  \node at(5,0.5){$\cdots$};
  \node at(6,0.5){$\cdots$};
  \node at(7,0.5){$\cdots$};
  \node at(8,0.5){$\cdots$};
  \node at(5,2.5){$T -T_1 \cdot \fdpL$ rounds};
  \node at(0.5,0.5){$\fdpL$};
  \node at(1.5,0.5){$\fdpL$};
  \node at(2.5,0.5){$\fdpL$};
  \node at(3.5,0.5){$\fdpL$};
  \node at(9.5,0.5){$\fdpL$};
  \node at(-1,0.5){\textbf{Level 1}};
  \node at(-1,2.5){\textbf{Level 2}};
  \draw[->] (11,0)--(11,3);
  \node at(11.5,1.5)[ rotate=90]{Time};

  \draw [decorate,decoration={brace,amplitude=10pt},xshift=0pt,yshift=0pt] (10,-0.2) -- (0,-0.2) node [black,midway,yshift=-0.6cm] 
  {$T_1$ full-disclosure paths of length $\fdpL$ each};
\end{tikzpicture}

\caption{Info-graph for the 2-level policy. }
%Each red box in level 1 corresponds to a path connecting a set of
%  $T_G$ agents. The entire history in level 1 is then aggregated and
%  shown to each agent in level 2.}
\label{fig:2level}
\end{figure}


It is useful to think of the info-graph as having two levels (corresponding to exploration and exploitation). Accordingly, we call this policy the \emph{\2LEVEL}. We show that it incentivizes the agents to perform non-adaptive exploration, and achieves a regret rate of  $\tilde O_K(T^{2/3})$. The key idea is that using many full-disclosure paths ``in parallel" ensures that sufficiently many samples of each arm are collected during exploration.

\begin{theorem}\label{thm:2level}
The \2LEVEL with parameter $T_1 = T^{2/3}\,\log(T)^{1/3}$ achieves regret
\[ \reg(T) \leq O_K\left( T^{2/3}\, \log(T)^{1/3} \right).\]
\end{theorem}

\begin{remark}
For a constant $K$, the number of arms, we match the optimal regret rate for non-adaptive multi-armed bandit algorithms. If the gap parameter $\Delta$ is known to the principal, then (for an appropriate tuning of parameter $T_1$) we can achieve regret 
  $\reg(T) \leq O_K(\log(T) \cdot \Delta^{-2})$.
\end{remark}

To simplify presentation of the proofs, we will make the following assumption.%
\footnote{This assumption can be removed, without any conceptual difficulty, but at the cost of a somewhat lengthier proofs.}
 

\begin{assumption}\label{ass:simplifying}
In each round $t$, the estimates $\hat{\mu}_{t,a}$ depend only the multiset 
    $\left\{\; (a_s,r_s):\;s\in S_t \;\right\}$,
called \emph{anonymized subhistory}. Each agent forms its estimates according to the same \emph{estimate function}: a function $f$ from subhistories to $[0,1]^K$, so that the estimate vector
        $(\hat{\mu}_{t,a}:\, a\in\A)$
equals $f(m_t)$. 
\end{assumption}

Among other things, this assumption allows us to the define the expected number of samples of a given arm $a$ collected by a full-disclosure path $S$ of length $\fdpL$ (\ie present in the subhistory $\SubH{S}$. Indeed, this number, denoted $\fdpN$, is the same for all such paths. Then,

\begin{lemma}\label{lem:t1runs}
Suppose the info-graph contains $T_1$ full-disclosure paths of $\fdpL$ rounds each. Let $N_a$ be the number of samples of arm $a$ collected by all paths. Then with probability at
  least $1-\delta$, for all $a\in \A$,
  \[
    \left| N_a - \fdpN T_1\right| \leq \fdpL\cdot \sqrt{T_1 \log(2K/\delta) / 2}.
  \]
\end{lemma}






%%% Local Variables:
%%% mode: latex
%%% TeX-master: "main"
%%% End:


%!TEX root = main.tex

\section{Adaptive exploration with a three-level disclosure policy}
\label{sec:3level}

The two-level policy from the previous section implements the explore-then-exploit paradigm using a basic design with parallel full-disclosure paths. The next challenge is to implement \emph{adaptive exploration}, and go below the $T^{2/3}$ barrier. We accomplish this using a construction that adds a middle level to the info-graph. This construction also provides intuition for the main result, the multi-level construction presented in the next section. For simplicity, we assume $K=2$ arms.

For the sake of intuition, consider the framework of bandit algorithms with limited adaptivity \cite{Perchet2015BatchedBP}. Suppose a bandit algorithm outputs a distribution $p_t$ over arms in each round $t$, and the arm $a_t$ is then drawn independently from $p_t$. This distribution can change only in a small number of rounds, called \emph{adaptivity rounds}, that need to be chosen by the algorithm in advance. A single round of adaptivity corresponds to explore-then-exploit paradigm. Our goal here is to implement one extra adaptivity round, and this is what the middle level accomplishes.


\begin{construction}
The \emph{three-level policy} is defined as follows. The info-graph consists of three levels: the first two correspond to \emph{exploration}, and the third implements \emph{exploitation}. Like in the two-level policy, the first level consists of multiple full-disclosure paths of length $\fdpL$ each, and each agent $t$ in the exploitation level sees full history from exploration  (see Figure~\ref{fig:3level}).

The middle level consists of $\NG$ disjoint subsets of $T_2$ agents each, called \emph{second-level groups}. Each second-level group $G$ has the following property:
\begin{align}\label{eq:group-defn}
\text{all nodes in $G$ are connected to the same nodes outside of $G$, but not to one another.}
\end{align}

%\jmcomment{$N_2$ is also used in Lemma 3.6. We need to find another notation. And I think it should be something without subscript 2. It's generally a property of the whole construction. How about $\NG$.}

The full-disclosure paths in the first level are also split into $\NG$ disjoint subsets, called \emph{first-level groups}. Each first-level group consists of $T_1$ full-disclosure paths, for the total of $T_1\cdot \NG\cdot \fdpL$ rounds in the first layer. There is a 1-1 correspondence between first-level groups $G$ and second-level groups $G'$, whereby each agent in $G'$ observes the full history from the corresponding group $G$. More formally, agent in $G'$ is connected to the last node of each full-disclosure path in $G$. In other words, this agent receives message
    $\SubH{S}$,
where $S$ is the set of all rounds in $G$.
\end{construction}

\begin{figure}[t]
\centering
\begin{tikzpicture}
 \filldraw[fill=green!20!white]
 (0,4)--(10,4)--(10,5)--(0,5)--cycle;
 \foreach \x in {0,3,8}
 {
 \filldraw[fill=blue!20!white]
 (\x+0,2)--(\x+2,2)--(\x+2,3)--(\x+0,3)--cycle;
 \draw (\x+1,3)--(5,4);
 \filldraw[fill=red!20!white]
 (\x+0,0)--(\x+2,0)--(\x+2,1)--(\x+0,1)--cycle;
 \draw (\x+1,1)--(\x+1,2);
 \draw(\x+0.33,1)--(\x+1,2);
 \draw(\x+1.66,1)--(\x+1,2);
 %\draw(\x+1.4,1)--(\x+1,2);
 %\draw(\x+1.8,1)--(\x+1,2);
 \draw(\x+0.66,0)--(\x+0.66,1);
 \draw(\x+1.33,0)--(\x+1.33,1);
% \draw (\x+1.2,0)--(\x+1.2,1);
 %\draw(\x+1.6,0)--(\x+1.6,1);
 \node at(\x+1,2.5){$T_2$ rounds};
 \node at(\x+0.33, 0.5){$\GdT$};
 %\node at(\x+0.6, 0.5){$\cdot$};
 \node at(\x+1.0, 0.5){$\cdots$};
 %\node at(\x+1.4, 0.5){$\cdot$};
 \node at(\x+1.66, 0.5){$\GdT$};
 %\node at(\x+1,0.5){$T_1 \cdot \GdT$};
 \draw [decorate,decoration={brace,amplitude=10pt},xshift=0pt,yshift=0pt] (\x+2,-0.2) -- (\x+0,-0.2) node [black,midway,yshift=-0.6cm] {$T_1$ paths};
 }
  \node at(5,4.5){all remaining rounds};
  \node at (6,0.5){$\cdots$};
  \node at (7,0.5){$\cdots$};
  \node at (6,2.5){$\cdots$};
  \node at (7,2.5){$\cdots$};
  \node at(-1,0.5){\textbf{Level 1}};
  \node at(-1,2.5){\textbf{Level 2}};
  \node at(-1,4.5){\textbf{Level 3}};
  \draw[->] (11,0)--(11,5);
  \node at(11.5,2.5)[ rotate=90]{Time};

  \draw [decorate,decoration={brace,amplitude=10pt,aspect=0.33},xshift=0pt,yshift=0pt] (10,1.8) -- (0,1.8) node [black,pos=0.33,xshift = 0cm,yshift=-0.6cm] {$\NG$ groups};

\end{tikzpicture}
\caption{Info-graph for the three-level policy. Each red box in level 1 corresponds to $T_1$ full-disclosure paths of length $\GdT$ each.}
\label{fig:3level}
\end{figure}

The key idea is as follows. Consider the gap parameter $\Delta = |\mu_1-\mu_2|$. If it is is large, then each first-level group produces enough data to determine the best arm with high confidence, and so each agent in the upper levels chooses the best arm. If $\Delta$ is small, then due to \emph{anti-concentration} each arm gets ``lucky" within  at least once first-level group, in the sense that it appears much better than the other arm based on the data collected in this group (and therefore this arm gets explored by the corresponding second-level group). To summarize, the middle level exploits if the gap parameter is large, and provides some more exploration if it is small.

\begin{theorem}
\label{thm:3level}
For two arms, the three-level policy
achieves regret
\[ \reg(T) \leq O\left( T^{4/7}\, \log T \right).\]
This is achieved with parameters
    $T_1 = T^{4/7}\log^{-1/7}(T)$,
    $\NG = 2^{10}\log(T)$, and
    $T_2 = T^{6/7}\log^{-5/7}(T)$.
\end{theorem}

Let us sketch the proof of this theorem; the full proof can be found in Sections~\ref{sec:3level-events} and~\ref{sec:3level-case}.

\xhdr{The ``good events".}
We establish four ``good events" each of which occurs with high probability.
\begin{description}
\item[(\event{1})] \emph{Exploration in Level 1:} Every first-level group collects at least $\Omega(T_1)$ samples of each arm.
\item[(\event{2})] \emph{Concentration in Level 1:} Within each first-level group, empirical mean rewards of each arm $a$ concentrate around $\mu_a$.
\item[(\event{3})] \emph{Anti-concentration in Level 1:} For each arm, some first-level subgroup collects data which makes this arm look much better than its actual mean and other arms look worse than their actual means.
\item[(\event{4})] \emph{Concentration in prefix:}
The empirical mean reward of each arm $a$ concentrates around $\mu_a$ in any prefix of its pulls. (This ensures accurate reward estimates in exploitation.)
\end{description}

The analysis of these events applies Chernoff Bounds to a suitable version of ``reward tape" (see the definition of ``reward tape" in Section~\ref{sec:model}). For example, \event{2} considers a reward tape restricted to a given first-level group.

\xhdr{Case analysis.}
We now proceed to bound the regret conditioned on the four ``good events". W.l.o.g., assume $\mu_1 \geq \mu_2$. We break down the regret analysis into four cases, based on the magnitude the gap parameter $\Delta = \mu_1-\mu_2$. As a shorthand, denote
    $\conf{n} = \sqrt{\log(T)/n}$.
In words, this is a confidence term, up to constant factors, for $n$ independent random samples.

The simplest case is very small gap, which trivially yields an upper bound on regret.

\begin{claim}[Negligible gap]
If
%  $\Delta \leq 3\sqrt{\frac{2\log(T)}{T_2}}$,
    $\Delta \leq 3\sqrt{2}\cdot  \conf{T_2}$
then
  $\reg(T)\leq O(T^{4/7} \log^{6/7}(T))$.
\end{claim}

Another simple case is when $\Delta$ is sufficiently large, so that
the data collected in any first-level group suffices to determine the best arm. The proof follows from \event{1} and \event{2}.

\begin{lemma}[Large gap]\label{3levelbigcase}
If
% $\Delta \geq 2\left(\sqrt{\frac{4\log(T)}{\fdpN[1]T_1}} +
%  \sqrt{\frac{4\log(T)}{\fdpN[2]T_1}}\right)$,
  $ \Delta \geq 4 \sum_{a\in\A}\; \conf{\fdpN[a]\cdot T_1}$
then all agents in the second and the third levels pull arm 1.
\end{lemma}

In the \emph{medium gap} case, the data collected in a given first-level group is no longer guaranteed to determine the best arm. However, agents in the third level see the history of not only one but all first-level groups and the data collected by all first-level groups enables agents in the third level to correctly identify the best arm.

%(even though one of the arms may not be pulled at all in the second level).

\begin{lemma}[Medium gap]\label{3levelmedium}
  All agents pull arm 1 in the third level, when $\Delta$ satisfies
%  \[
%  \Delta\in \left[
%   2\left(\sqrt{\frac{4\log(T)}{S\fdpN[1]T_1}} +
%    \sqrt{\frac{4\log(T)}{S\fdpN[2]T_1}}\right),
% 2\left(\sqrt{\frac{4\log(T)}{\fdpN[1]T_1}} +
%        \sqrt{\frac{4\log(T)}{\fdpN[2]T_1}}\right),
%    \right)
%  \]
\[ \textstyle \Delta\in \left[
    4\,\sum_{a\in\A}\; \conf{\NG\cdot \fdpN[a]\cdot T_1},\quad
    4\,\sum_{a\in\A}\; \conf{\fdpN[a]\cdot T_1}
 \right] \]
\end{lemma}


Finally, the \emph{small gap} case, when  $\Delta$ is between
$\tilde\Omega(\sqrt{1/T_2})$ and $\tilde O(\sqrt{1/(\NG\, T_1)})$
is more challenging since even aggregating the data from all $\NG$
first-level groups is not sufficient for identifying the best arm.
We need to ensure that both arms continue to be explored in the second level.
To achieve this, we leverage \event{3}, which implies
that each arm $a$ has a first-level group $s_a$ where it gets ``lucky", in the sense that its empirical mean reward is slightly higher than $\mu_a$, while the empirical mean reward of the other arm is slightly lower than its true mean. Since the
deviations are in the order of $\Omega(\sqrt{1/T_1})$, and Assumption~\ref{ass:embehave} guarantees the agents' reward estimates are also within $\Omega(\sqrt{1/T_1})$ of the empirical means, the sub-history
from this group $s_a$ ensures that all agents in the respective second-level group prefer arm $a$. Therefore, both arms are pulled at least $T_2$ times in the second level, which in turn gives the following guarantee:


\begin{lemma}[Small gap]\label{3levelsmallcase}
  All agents pull arm 1 in the third level, when $\Delta$ satisfies
%  \[
%    \Delta\in \left( 3\sqrt{\frac{2\log(T)}{T_2}},
%      2\left(\sqrt{\frac{4\log(T)}{S \fdpN[1]T_1}} +
%        \sqrt{\frac{4\log(T)}{S \fdpN[2]T_1}}\right) \right)
%  \]
\[ \textstyle
    \Delta\in \left( 3\sqrt{2}\cdot\conf{T_2},\quad
      4\,\sum_{a\in\A}\; \conf{\NG\cdot \fdpN[a]\cdot T_1}
\right)
\]
\end{lemma}


\paragraph{Wrapping up: proof of \Cref{thm:3level}. } In negligible
gap case, the stated regret bound holds regardless of what the algorithm does. In the large gap case, the regret only comes from the
first level, so it is upper-bounded by the total number of agents in this level, which is
    $\NG\cdot \GdT \cdot T_1 = O(T^{4/7} \log T)$.
In both intermediate cases, it suffices to bound the regret from the
first and second levels, so
\[ \textstyle
\reg(T) \leq (\NG\,T_1\cdot \GdT + \NG\, T_2)
\cdot 4\,\sum_{a\in\A}\; \conf{\fdpN[a]\cdot T_1}
= O(T^{4/7} \log^{6/7}(T)).
\]
Therefore, we obtain the stated regret bound in all cases.




\OMIT{
\begin{proof}
Wlog we assume $\mu_1 \geq \mu_2$ as the recommendation policy is symmetric to both arms. We do a case analysis based on $\mu_1-\mu_2$.

Before we start with the case analysis, we first define several clean events and show that the intersection of them happens with high probability.

\begin{itemize}

\item \textbf{Concentration of the number of arm $a$ pulls in the first level:}
\OMIT{By Lemma \ref{lem:greedy}, we know $\GdP \leq \fdpN  \leq \GdT$. For the $s$-th first-level group, define $W_1^{a,s}$ to be the event that the number of arm $a$ pulls in the $s$-th first-level group is between $\fdpN  T_1- \GdT \sqrt{T_1\log(T)}$ and $\fdpN  T_1 + \GdT \sqrt{T_1\log(T)}$. By Chernoff bound,}
For the $s$-th first-level group, define $W_1^{a,s}$ to be the event
that the number of arm $a$ pulls in the $s$-th first-level group is
between $\fdpN  T_1- \GdT \sqrt{T_1\log(T)}$ and
$\fdpN  T_1 + \GdT \sqrt{T_1\log(T)}$. By Lemma~\ref{lem:t1runs}
\[
\Pr[W_1^{a,s}] \geq 1-2\exp(-2\log(T)) \geq 1-2/T^2.
\]
Let $W_1$  be the intersection of all these events (i.e.
$W_1 = \bigcap_{a,s}W_1^{a,s}$). By union bound, we have
\[
\Pr[W_1] \geq 1- \frac{4S}{T^2}.
\]}

\OMIT{\item \textbf{Concentration of the empirical mean of arm $a$ pulls in the first level:}
For each first-level group and arm $a$, imagine there is a tape of enough arm $a$ pulls sampled before the recommendation policy starts and these samples are revealed one by one whenever agents in this group pull arm $a$. For the $s$-th first-level group and arm $a$, define $W_2^{s,a,t_1,t_2}$ to be the event that the mean of $t_1$-th to $t_2$-th pulls in the tape is at most $\sqrt{\frac{2\log(T)}{t_2-t_1+1}}$ away from $\mu_a$. By Chernoff bound,\swcomment{a bit confused about what $t_1$ and $t_2$ mean?}
\[
\Pr[W_2^{s,a,t_1,t_2}] \geq 1 - 2\exp(-4\log(T)) \geq 1- 2/T^4.
\]

Define $W_2$ to be the intersection of all these events (i.e. $W_2 = \bigcap_{a,s,t_1,t_2} W_2^{s,a,t_1,t_2}$). By union bound, we have
\[
\Pr[W_2] \geq 1- \frac{4S}{T^2}.
\]
}
\OMIT{\item \textbf{Concentration of the empirical mean of arm $a$ pulls in the first two levels:}

For all the groups in the first two levels and arm $a$, imagine there is a tape of enough arm $a$ pulls sampled before the recommendation policy starts and these samples are revealed one by one whenever agents in the first two levels pull arm $a$. Define $W_3^{a,t}$ to be the event that the mean of the first $t$ pulls in the tape is at most $\sqrt{\frac{2\log(T)}{t}}$ away from $\mu_a$. By Chernoff bound,
\[
\Pr[W_3^{a,t}] \geq 1 - 2\exp(-4\log(T)) \geq 1- 2/T^4.
\]
Define $W_3$ to be the intersection of all these events (i.e. $W_3 = \bigcap_{a,t} W_3^{a,t}$). By union bound, we have
\[
\Pr[W_3] \geq 1- \frac{4}{T^3}.
\]
}
\OMIT{\item \textbf{Anti-concentration of the empirical mean of arm $a$ pulls in the first level:}

Consider the tapes defined in the second bullet again. For the $s$-th first-level group and arm $a$, define $W_4^{s,a,high}$  to be the event that first $\fdpN  T_1$ pulls of arm $a$ in the corresponding tape has empirical mean at least $\mu_a + 1/\sqrt{\fdpN  T_1}$ and define  $W_4^{s,a,low}$  to be the event that first $\fdpN  T_1$ pulls of arm $a$ in the corresponding tape has empirical mean at most $\mu_a - 1/\sqrt{\fdpN  T_1}$. By Berry-Essen Theorem and $\mu_a \in [1/3,2/3]$, we have
\[
\Pr[W_4^{s,a,high}] \geq (1-\Phi(1/2)) - \frac{5}{\sqrt{\fdpN T_1}} > 1/4.
\]
The last inequality follows when $T$ is larger than some constant.
Similarly we also have
\[
\Pr[W_4^{s,a,low}] > 1/4.
\]
Since $W_4^{s,a,high}$ is independent with $W_4^{s,3-a,low}$, we have
\[
\Pr[W_4^{s,a,high} \cap W_4^{s,3-a,low}] =\Pr[W_4^{s,a,high}] \cdot  \Pr[W_4^{s,3-a,low}]>(1/4)^2 = 1/16.
\]
Now define $W_4$ as $\bigcap_a \bigcup_s (W_4^{s,a,high} \cap W_4^{s,3-a,low})$. Notice that $(W_4^{s,a,high} \cap W_4^{s,3-a,low})$ are independent across different $s$'s. By union bound, we have
\[
\Pr[W_4] \geq 1- 2(1-1/16)^S \geq 1 -2 /T.
\]
\end{itemize}

By union bound, the intersection of these clean events (i.e. $\bigcap_{i=1}^4 W_i$) happens with probability $1-O(1/T)$. When this intersection does not happen, since the probability is $O(1/T)$, it contributes $O(1/T) \cdot T = O(1)$ to the expected regret.

Now we assume the intersection of clean events happens and we summarize what these clean events imply.

\begin{itemize}
\item For the $s$-th first-level group and arm $a$, define $\bar{\mu}_a^{1,s}$ to be the empirical mean of arm $a$ pulls in this group. $W_1^{a,s}$, $W_2^{a,s,1,t}$ for $ = \fdpN  T_1- \GdT \sqrt{T_1\log(T)},...,\fdpN  T_1- \GdT \sqrt{T_1\log(T)}$ together imply that
\[
|\bar{\mu}_a^{1,s} - \mu_a| \leq \sqrt{\frac{2\log(T)}{\fdpN  T_1- \GdT \sqrt{T_1\log(T)}}} \leq \sqrt{\frac{4\log(T)}{\fdpN  T_1}}.
\]
The last inequality holds when $T$ is larger than some constant.
\item For each arm $a$, define $\bar{\mu}_a$ to be the empirical mean of arm $a$ pulls in the first two levels. $W_1^{a,s}$ for $s=1,...,S$ and $W_3^{a,t}$ for $t \geq  (\fdpN  T_1- \GdT \sqrt{T_1\log(T)})S$ together imply that
\[
|\bar{\mu}_a - \mu_a| \leq \sqrt{\frac{2\log(T)}{S\left(\fdpN  T_1- \GdT \sqrt{T_1\log(T)}\right)}} \leq \sqrt{\frac{4\log(T)}{S \fdpN  T_1}} .
\]
The last inequality holds when $T$ is larger than some constant.

If there are at least $T_2$ pulls of arm $a$ in the first two levels,
\[
|\bar{\mu}_a-\mu_a| \leq \sqrt{\frac{2\log(T)}{T_2}}.
\]

\item For each $a \in \{1,2\}$, $W_4$ implies that there exists $s_a$ such that $W_4^{s_a,a,high}$ and $W_4^{s_a,3-a,low}$ happen. $W_4^{s_a,a,high}$,  $W_1^{s_a,a}$, $W_2^{s_a,a,t, \fdpN T_1}$ for $t = \fdpN  T_1- \GdT \sqrt{T_1\log(T)}+1, ...,\fdpN T_1-1$ and $W_2^{s_a,a,\fdpN T_1,t}$ for $t= \fdpN T_1,...,\fdpN  T_1+ \GdT \sqrt{T_1\log(T)}$ together imply that
\begin{align*}
\bar{\mu}_a ^{1,s_a} &\geq \mu_a + \left(\fdpN T_1 \cdot \frac{1}{\sqrt{\fdpN T_1}} - \GdT \sqrt{T_1\log(T)} \cdot \sqrt{\frac{2\log(T)}{ \GdT \sqrt{T_1\log(T)}}} \right) \cdot \frac{1}{\fdpN  T_1+ \GdT \sqrt{T_1\log(T)}} \\
&> \mu_a + \frac{1}{4\sqrt{\fdpN T_1}}.
\end{align*}
The second last inequality holds when $T$ is larger than some constant.
Similarly, we also have
\[
\bar{\mu}_{3-a} ^{1,s_a} < \mu_{3-a}   - \frac{1}{4\sqrt{q_{3-a} T_1}}.
\]
\end{itemize}

Finally we proceed to the case analysis. We give upper bounds on the expected regret conditioned on the intersection of clean events.

\begin{itemize}

\item $\mu_1 - \mu_2 \geq 2\left(\sqrt{\frac{4\log(T)}{\fdpN[1]T_1}}
+ \sqrt{\frac{4\log(T)}{\fdpN[2]T_1}}\right)$. In this case, we want to show that agents in the second and the third levels all pull arm 1.

First consider the $s$-th second-level group. We know that
\[
\bar{\mu}_1^{1,s} - \bar{\mu}_2^{1,s} \geq \mu_1 -\mu_2 - \sqrt{\frac{4\log(T)}{\fdpN[1]T_1}} - \sqrt{\frac{4\log(T)}{\fdpN[2]T_1}} \geq  \sqrt{\frac{4\log(T)}{\fdpN[1]T_1}} + \sqrt{\frac{4\log(T)}{\fdpN[2]T_1}}.
\]
For any agent $t$ in the $s$-th second-level group, by Assumption \ref{ass:embehave}, we have
\begin{align*}
\hat{\mu}_1^t - \hat{\mu}_2^t &>\bar{\mu}_1^{1,s} - \bar{\mu}_2^{1,s} - \frac{c_m}{\sqrt{\fdpN[1]T_1/2}} - \frac{c_m}{\sqrt{\fdpN[2]T_1/2}}\\
&\geq  \sqrt{\frac{4\log(T)}{\fdpN[1]T_1}} + \sqrt{\frac{4\log(T)}{\fdpN[2]T_1}}- \frac{c_m}{\sqrt{\fdpN[1]T_1/2}} - \frac{c_m}{\sqrt{\fdpN[2]T_1/2}}\\
 &> 0.
\end{align*}
Therefore, we know agents in the $s$-th second-level group will all pull arm 1.

Now consider the agents in the third level group. Recall $\bar{\mu}_a$ is the empirical mean of arm $a$ in the history they see. We have
\[
\bar{\mu}_1 - \bar{\mu}_2 \geq \mu_1 -\mu_2 - \sqrt{\frac{4\log(T)}{S\fdpN[1]T_1}} - \sqrt{\frac{4\log(T)}{S\fdpN[2]T_1}} \geq  \sqrt{\frac{4\log(T)}{\fdpN[1]T_1}}
+ \sqrt{\frac{4\log(T)}{\fdpN[2]T_1}}.
\]
Similarly as above, by Assumption \ref{ass:embehave}, we know $\hat{\mu}_1^t - \hat{\mu}_2^t > 0$ for any agent $t$ in the third level. So we know agents in the third-level group will all pull arm 1. Therefore the expected regret is at most $S T_G T_1 = O(T^{4/7} \log^{6/7}(T))$.


\item $2\left(\sqrt{\frac{4\log(T)}{S\fdpN[1]T_1}}
+ \sqrt{\frac{4\log(T)}{S\fdpN[2]T_1}}\right) \leq \mu_1-\mu_2 < 2\left(\sqrt{\frac{4\log(T)}{\fdpN[1]T_1}}
+ \sqrt{\frac{4\log(T)}{\fdpN[2]T_1}}\right)$. In this case, we want to show agents in the third level all pull arm 1. Recall $\bar{\mu}_a$ is the empirical mean of arm $a$ in the first two levels. We have
\[
\bar{\mu}_1 - \bar{\mu}_2 \geq \mu_1 -\mu_2 - \sqrt{\frac{4\log(T)}{S\fdpN[1]T_1}} - \sqrt{\frac{4\log(T)}{S\fdpN[2]T_1}} \geq  \sqrt{\frac{4\log(T)}{S\fdpN[1]T_1}}
+ \sqrt{\frac{4\log(T)}{S\fdpN[2]T_1}}.
\]
For any agent $t$ in the third level, by Assumption \ref{ass:embehave}, we have
\begin{align*}
\hat{\mu}_1^t - \hat{\mu}_2^t &>\bar{\mu}_1 - \bar{\mu}_2 - \frac{c_m}{\sqrt{S\fdpN[1]T_1/2}} - \frac{c_m}{\sqrt{S\fdpN[2]T_1/2}}\\
&\geq  \sqrt{\frac{4\log(T)}{S\fdpN[1]T_1}} + \sqrt{\frac{4\log(T)}{S\fdpN[2]T_1}}- \frac{c_m}{\sqrt{S\fdpN[1]T_1/2}} - \frac{c_m}{\sqrt{S\fdpN[2]T_1/2}}\\
 &> 0.
\end{align*}
So we know agents in the third-level group will all pull arm 1. Therefore the expected regret is at most
\[
(S T_G T_1 + S T_2) \cdot 2\left(\sqrt{\frac{4\log(T)}{\fdpN[1]T_1}}
+ \sqrt{\frac{4\log(T)}{\fdpN[2]T_1}}\right) = O(T^{4/7} \log^{6/7}(T))
\]

\item $ 3\sqrt{\frac{2\log(T)}{T_2}} < \mu_1-\mu_2 < 2\left(\sqrt{\frac{4\log(T)}{S\fdpN[1]T_1}}
+ \sqrt{\frac{4\log(T)}{S\fdpN[2]T_1}}\right)$. In this case, we just need to make sure that agents in the third level all pull arm 1. To do so, we need both arms to be pulled at least $T_2$ rounds in the second level.

Now consider the $s_a$-th second-level group. We have
\begin{align*}
\bar{\mu}_a^{1,s_a} - \bar{\mu}_{3-a}^{1,s_a} &> \mu_a + \frac{1}{4\sqrt{\fdpN T_1}} -\mu_{3-a} +\frac{1}{4\sqrt{q_{3-a}T_1}} \\
&> \frac{1}{4\sqrt{\fdpN[1]T_1}}+ \frac{1}{4\sqrt{\fdpN[2]T_1}} - 2\left(\sqrt{\frac{4\log(T)}{S\fdpN[1]T_1}}
+ \sqrt{\frac{4\log(T)}{S\fdpN[2]T_1}}\right) \\
&\geq \frac{1}{8\sqrt{\fdpN[1]T_1}}+ \frac{1}{8\sqrt{\fdpN[2]T_1}}.
\end{align*}
For any agent $t$ in the $s_a$-th second-level group, by Assumption \ref{ass:embehave}, we have
\begin{align*}
\hat{\mu}_a^t - \hat{\mu}_{3-a}^t &>\bar{\mu}_a^{1,s_a} - \bar{\mu}_{3-a}^{1,s_a} - \frac{c_m}{\sqrt{\fdpN[1]T_1/2}} - \frac{c_m}{\sqrt{\fdpN[2]T_1/2}}\\
&\geq   \frac{1}{8\sqrt{\fdpN[1]T_1}}+ \frac{1}{8\sqrt{\fdpN[2]T_1}}- \frac{c_m}{\sqrt{\fdpN[1]T_1/2}} - \frac{c_m}{\sqrt{\fdpN[2]T_1/2}}\\
 &> 0.
\end{align*}
So we know agents in the $s_a$-th second-level group will all pull arm $a$. Therefore in the first two levels, both arms are pulled at least $T_2$ times. Now consider the third-level. We have
\[
\bar{\mu}_1 - \bar{\mu}_2  \geq \mu_1 -\mu_2 - 2\sqrt{\frac{2\log(T)}{T_2}} \geq \sqrt{\frac{2\log(T)}{T_2}}.
\]
Similarly as above, by Assumption \ref{ass:embehave}, we know $\hat{\mu}_1^t - \hat{\mu}_2^t > 0$ for any agent $t$ in the third level. So we know agents in the third-level group will all pull arm 1.

Therefore the expected regret is at most
\[
(S T_G T_1 + S T_2) \cdot 2\left(\sqrt{\frac{4\log(T)}{S\fdpN[1]T_1}}
+ \sqrt{\frac{4\log(T)}{S\fdpN[2]T_1}}\right) \leq O(T^{4/7} \log^{6/7}(T))
\]


\item $\mu_1 - \mu_2 \leq 3\sqrt{\frac{2\log(T)}{T_2}}$. This is the easy case. Even always pulling the sub-optimal arm (i.e. arm 2) gives regret at most $T \cdot (\mu_1-\mu_2) = O(T^{4/7} \log^{6/7}(T))$.
\end{itemize}
\end{proof}}


%%% Local Variables:
%%% mode: latex
%%% TeX-master: "main"
%%% End:


%l-level in the 10page
%!TEX root = main.tex

\section{$L$-level Recommendation Policy}
\label{sec:llevel}
In this section, we give an overview of how we extend our 3-level policy to an $L$-level policy for $L > 3$ which achieves better regret. Detailed proofs and discussions can be found in Section \ref{sec:llevel-details}.

\jmcomment{Will change these statements a bit.}
\begin{theorem}
\label{thm:llevel}
The $L$-level recommendation policy gets expected regret $O\left(T^{2^{L-1}/(2^L-1)} \log^2(T) \right)$ for $L \leq \log(\ln(T)/\log(S^4))$. In particular, if we pick $L = \log(\ln(T)/\log(S^4))$, the expected regret is $O(T^{1/2} polylog(T))$. 
\end{theorem}

\begin{corollary}
\label{cor:llevel}
With the proper setting of $L$ and $T_1,...,T_L$ described above, the $L$-level recommendation policy gets expected regret $O(\min(1/\Delta, T^{1/2})polylog(T))$. Here $\Delta$ is the difference between the largest mean and second largest mean of arms and the $L$-level recommendation policy does not need to know $\Delta$. Moreover, agent $t$ observes a subhistory of size at least $\Omega( \lfloor t/polylog(T)\rfloor)$. 
\end{corollary}

Now we give some overview of the additional techniques we use for our $L$-level policy. 

\xhdr{New connecting structures between levels.}

\xhdr{Additional groups for boundary cases.}

%\bibliographystyle{plain}
%

%\acks{We would like to thank Robert Kleinberg for discussions in the early stage of this project.}
\bibliography{bib-abbrv,bib-AGT,bib-bandits,bib-ML,bib-random,bib-slivkins}
\appendix
%!TEX root = main.tex
\section{Tools from Probability: concentration and anti-concentration}
\label{sec:prelim}

We use standard tools for concentration and anti-concentration, stated below.

\begin{theorem}[Chernoff Bounds]\label{thm:prelims-chernoff}
Let $X_1,...,X_n$  be independent random variables such that $X_i \in [0,1]$ for all $i$. Let $\bar{X} = \frac{1}{n}\sum_{i=1}^n X_i$ denote their empirical mean. Then
\[
\Pr[ |\bar{X} - \E[\bar{X}]| > \varepsilon] \leq 2\exp(-2n\varepsilon^2).
\]
\end{theorem}

\begin{theorem}[Berry-Esseen Theorem]\label{thm:prelims-Berry}
Let $X_1,...,X_n$ be i.i.d. variables with $\E[ (X_1 - \E[X_1])^2] = \sigma^2 >0$ and $\E[ |X_1 - \E[X_1]|^3] =\rho <\infty$. Let $\bar{X} = \frac{1}{n} \sum_{i=1}^n X_i$. Let $F_n$ be the cumulative distribution function of $\frac{(\bar{X} - \E[\bar{X}]) \sqrt{n}}{\sigma}$ and $\Phi$ be the cumulative distribution function of the standard normal distribution. For all $x$ and $n$,
\[
|F_n(x) - \Phi(x) | \leq \frac{\rho}{2\sigma^3\sqrt{n}}.
\]
\end{theorem}


\jmcomment{
In proofs, we use reward tapes to argue about concentration and anti-concentration of empirical means of some arm $a$ in certain rounds. 
\begin{definition}[Reward Tapes]
A reward tape of arm $a$ and length $L$ is a tape of $L$ cells, with each cell independently sampled from $\cD_a$. 
\end{definition}

}

%!TEX root = main.tex
\section{Proofs from Section~\ref{sec:warmup}}
\label{app:warmup}

\begin{proof}[Proof of Lemma~\ref{lem:greedy}]
  Fix any arm $a$. Let $\GdT = (K-1) \cdot N_0 + 1$ and
  $\GdP = (1/3)^{\GdT}$. \swedit{We will condition on the event that
    all the realized rewards in $\GdT$ rounds are 0, which occurs with
    probability at least $\GdP$ under Assumption~\ref{ass:embehave}.}
  In this case, we want to show that arm $a$ is pulled at least
  once. We prove this by contradiction. Suppose arm $a$ is not pulled. By
  the pigeonhole principle, we know that there is some other arm $a'$
  that is pulled at least $N_0 + 1$ rounds. Let $t$ be the round in
  which arm $a'$ is pulled exactly $N_0 + 1$ times. By Assumption
  \ref{ass:embehave}, we know
  \[
    \hat{\mu}_{a'}^t \leq 0 + C_0 / \sqrt{N_0} < 1/3.
  \]
  \nicomment{Wait, where did we set $N_0$?} On the other hand, we have
  $\hat{\mu}_a^t \geq 1/3 > \hat{\mu}_{a'}^t$. This contradicts with
  the fact that in round $t$, arm $a'$ is pulled, instead of arm $a$.
  \swdelete{ In addition, we know that this case happens with
    probability at least $(1/3)^{\GdT} = \GdP$ as each arm's mean is
    at most $2/3$. To sum up, we know that with probability at least
    $\GdP$, \ALGG of $\GdT$ rounds pulls arm $a$ at least once.}
   \jmcomment{Agree with these changes.}
\end{proof}


\begin{proof}[Proof of Theorem~\ref{thm:2level}]
  We will set $T_1$ later in the proof, depending on whether the gap
  parameter $\Delta$ is known. For now, we just need to know we will
  make $T_1 \geq \frac{4(\GdT)^2}{(\GdP)^2}\log(T)$. Since this policy is
  agnostic to the indices of the arms, we assume w.l.o.g. that arm 1
  has the highest mean.

  The first $T_1 \cdot \GdT$ rounds will get total regret at most
  $T_1 \cdot \GdT$.  We focus on bounding the regret from the second
  level of $T - T_1 \cdot \GdT$ rounds. We consider the following two
   events. We will first bound the probability that both of them
  happen and then we will show that they together imply upper bounds
  on $|\hat{\mu}^t_a - \mu_a|$'s for any agent $t$ in the second
  level. Recall $\hat{\mu}^t_a$ is the estimated mean of arm $a$ by
  agent $t$ and agent $t$ picks the arm with the highest
  $\hat{\mu}^t_a$.

% \begin{itemize}
  \OMIT{\paragraph{Concentration of the number of arm $a$ pulls in the first
    level.}
By Lemma \ref{lem:greedy}, we know $\GdP \leq \fdpN \leq \GdT$.}
  Define $W_1^a$ to be the event that the number of arm $a$ pulls in
  the first level is at least $\fdpN T_1- \GdT \sqrt{T_1\log(T)}$.
  \swedit{As long as we set $T_1 \geq \frac{4(\GdT)^2}{(\GdP)^2}\log(T)$,
    this implies that the number of arm $a$ pulls is then at least
    $\fdpN T_1/2$.}
\OMIT{  By Chernoff bound,
  \[
    \Pr[W_1^a] \geq 1-\exp(-2\log(T)) \geq 1-1/T^2.
  \]
}
Define $W_1$ to be the intersection of all these events (i.e. $W_1 = \bigcap_{a}W_1^a$). By Lemma~\ref{lem:t1runs}, we have
\[
\Pr[W_1] \geq 1- \frac{K}{T^2} \geq 1 - \frac{1}{T}.
\]
\OMIT{\paragraph{Concentration of the empirical mean of arm $a$ pulls
    in the first level.}}  Next, we show that the empirical mean of
each arm $a$ is close to the true mean. To facilitate our reasoning,
let us imagine there is a tape of length $T$ for each arm $a$, with
each cell containing an independent draw of the realized reward from
the distribution $\cD_a$. Then for each arm $a$ and any $N\in [T]$, we
can think of the sequence of the first $N$ realized rewards of $a$
coming from the prefix of $N$ cells in its reward tape. Define
$W^{a,t}_2$ to be the event that the empirical mean of the first $t$
\swedit{realized rewards in the tape} of arm $a$ is at most
$\sqrt{\frac{2\log(T)}{t}}$ away from $\mu_a$. Define $W_2$ to be the
intersection of these events (i.e.  $\bigcap_{a,t} W^{a,t}_2$).  By
Chernoff bound,\swcomment{$t$ may be confusing here}
\[
\Pr[W^{a,t}_2] \geq 1 - 2\exp(-4\log(T)) \geq 1-2/T^4.
\]
By union bound,
\[
\Pr[W_2] \geq 1 - KT \cdot \frac{2}{T^4} \geq 1 - \frac{2}{T}.
\]



By union bound, we know $\Pr[W_1 \cap W_2] \geq 1 - 3/T$. For the
remainder of the analysis, we will condition on the event
$W_1 \cap W_2$.

For any arm $a$ and agent $t$ in the second level, by $W_1$ and $W_2$, we have
\[
|\bar{\mu}^t_a - \mu_a| \leq \sqrt{\frac{2\log(T)}{\fdpN T_1 /2}}.
\]
By $W_1$ and Assumption \ref{ass:embehave}, we have
\[
|\bar{\mu}^t_a - \hat{\mu}^t_a| \leq \frac{C_0}{\sqrt{\fdpN T_1/2}}.
\]
Therefore,
\[
|\hat{\mu}^t_a - \mu_a|\leq \sqrt{\frac{2\log(T)}{\fdpN T_1 /2}}+\frac{C_0}{\sqrt{\fdpN T_1/2}} \leq 3 \sqrt{\frac{\log(T)}{\GdP T_1 }}.
\]
So the second-level agents will pick an arm $a$ which has $\mu_a$ at most $6 \sqrt{\frac{\log(T)}{\GdP T_1 }}$ away from $\mu_1$. To sum up, the total regret is at most
\[
T_1 \cdot \GdT + T \cdot (1-\Pr[W_1 \cap W_2]) + T \cdot  6 \sqrt{\frac{\log(T)}{\GdP T_1 }}.
\]
By setting $T_1 = T^{2/3}\log(T)^{1/3}$, we get regret $O(T^{2/3}\log(T)^{1/3})$.
\OMIT{Notice that if we know the gap parameter is known to be larger than
$\Delta$, we can set
\[
T_1 = \max\left( \frac{4\GdT^2}{\GdP^2}\log(T), \frac{36 }{\Delta^2 \cdot p_G} \log(T) \right).
\]
In this case, since $\Delta \geq 6 \sqrt{\frac{\log(T)}{\GdP T_1 }}$, we know agents in the second level will all pull arm 1. Therefore, the total regret is at most
\[
T_1 \cdot \GdT + T \cdot (1- \Pr[W_1 \cap W_2]) = O(\Delta^{-2} \log(T)).
\]
This completes the proof.}
\end{proof}

\section{Missing proofs from Section~\ref{sec:3level}}

\subsection{Clean Events}


\begin{proof}[Proof of Lemma~\ref{3levelw1}]
  For the $s$-th first-level group, define $W_1^{a,s}$ to be the event
  that the number of arm $a$ pulls in the $s$-th first-level group is
  between $q_a T_1- \GdT \sqrt{T_1\log(T)}$ and
  $q_a T_1 + \GdT \sqrt{T_1\log(T)}$. By Lemma~\ref{lem:t1runs}
\[
\Pr[W_1^{a,s}] \geq 1-2\exp(-2\log(T)) \geq 1-2/T^2.
\]
Let $W_1$  be the intersection of all these events (i.e.
$W_1 = \bigcap_{a,s}W_1^{a,s}$). By union bound, we have
\[
\Pr[W_1] \geq 1- \frac{4S}{T^2}.
\]
\end{proof}



\begin{proof}[Proof of Lemma~\ref{3levelw2}]
  By Chernoff bound,
\[
\Pr[W_2^{s,a,t_1,t_2}] \geq 1 - 2\exp(-4\log(T)) \geq 1- 2/T^4.
\]
By union bound, we have
\[
\Pr[W_2] \geq 1- \frac{4S}{T^2}.
\]
\end{proof}


\begin{proof}[Proof of Lemma~\ref{3levelw4}]
By Berry-Esseen Theorem and
  $\mu_a \in [1/3,2/3]$, we have for any $a$,
\[
\Pr[W_4^{s,a,high}] \geq (1-\Phi(1/2)) - \frac{5}{\sqrt{q_aT_1}} > 1/4.
\]
The last inequality follows when $T$ is larger than some constant.
Similarly we also have 
\[
\Pr[W_4^{s,a,low}] > 1/4.
\]
Since $W_4^{s,a,high}$ is independent with $W_4^{s,3-a,low}$, we have
\[
\Pr[W_4^{s,a,high} \cap W_4^{s,3-a,low}] =\Pr[W_4^{s,a,high}] \cdot  \Pr[W_4^{s,3-a,low}]>(1/4)^2 = 1/16.
\]
Notice that $(W_4^{s,a,high} \cap W_4^{s,3-a,low})$ are independent
across different $s$'s. By union bound, we have
\[
\Pr[W_4] \geq 1- 2(1-1/16)^S \geq 1 -2 /T.
\]
\end{proof}



\begin{proof}[Proof of Lemma~\ref{3levelw3}]
  For any arm $a$, let's imagine a hypothetical tape of length $T$,
  with each cell independently sampled from $\cD_a$. The tape encodes
  rewards of the first two levels as follows: the $j$-th time arm $a$
  is chosen in the first two levels, its reward is taken from the
  $j$-th cell in the tape. Define $W_3^{a,t}$ to be the event that the
  mean of the first $t$ pulls in the tape is at most
  $\sqrt{\frac{2\log(T)}{t}}$ away from $\mu_a$. By Chernoff bound,
\[
\Pr[W_3^{a,t}] \geq 1 - 2\exp(-4\log(T)) \geq 1- 2/T^4.
\]
By union bound, the intersection of all these events has probability
at least:
\[
\Pr[W_3] \geq 1- \frac{4}{T^3}.
\]
\end{proof}

\subsection{Case Analysis under Clean Events}
Now we assume the intersection of clean events $W$ happens. We will
first provide some helper lemmas for our case analysis.

\begin{lemma}
  For the $s$-th first-level group and arm $a$, define
  $\bar{\mu}_a^{1,s}$ to be the empirical mean of arm $a$ pulls in
  this group. If $W$ holds, then
  \[
    |\bar{\mu}_a^{1,s} - \mu_a| \leq \sqrt{\frac{4\log(T)}{q_a T_1}}.
  \]
\end{lemma}

\begin{proof}
  The events $W_1$ and $W_2^{a,s,1,t}$ for
  $t = q_a T_1- \GdT \sqrt{T_1\log(T)},...,q_a T_1 + \GdT
  \sqrt{T_1\log(T)}$ together imply that
\[
|\bar{\mu}_a^{1,s} - \mu_a| \leq \sqrt{\frac{2\log(T)}{q_a T_1- \GdT \sqrt{T_1\log(T)}}} \leq \sqrt{\frac{4\log(T)}{q_a T_1}}.
\]
The last inequality holds when $T$ is larger than some constant.
\end{proof}


\begin{lemma}
  For each arm $a$, define $\bar{\mu}_a$ to be the empirical mean of
  arm $a$ pulls in the first two levels. If $W$ holds, then
  \[
    |\bar{\mu}_a - \mu_a| \leq \sqrt{\frac{4\log(T)}{S q_a T_1}} .
  \]
Furthermore, if there are at least $T_2$ pulls of arm $a$ in the first
two levels,
\[
|\bar{\mu}_a-\mu_a| \leq \sqrt{\frac{2\log(T)}{T_2}}. 
\]
\end{lemma}

\begin{proof}
The events $W_1$ and $W_3^{a,t}$ for $t \geq  (q_a T_1- \GdT \sqrt{T_1\log(T)})S$ together imply that
  \[
    |\bar{\mu}_a - \mu_a| \leq \sqrt{\frac{2\log(T)}{S\left(q_a T_1- \GdT \sqrt{T_1\log(T)}\right)}} \leq \sqrt{\frac{4\log(T)}{S q_a T_1}} .
\]
The last inequality holds when $T$ is larger than some constant.
\end{proof}


\begin{lemma}\label{lem:luck}
  For the $s$-th first-level group and arm $a$, define
  $\bar{\mu}_a^{1,s}$ to be the empirical mean of arm $a$ pulls in
  this group. For each $a \in \{1,2\}$, there exists a group $s_a$
  such that
\[
\bar{\mu}_a ^{1,s_a} > \mu_a + \frac{1}{4\sqrt{q_aT_1}} \quad \mbox{and, } \quad
\bar{\mu}_{3-a} ^{1,s_a} < \mu_{3-a}   - \frac{1}{4\sqrt{q_{3-a} T_1}}.
\]
\end{lemma}




\begin{proof}% [Proof of Lemma~\ref{lem:luck}]
  For each $a \in \{1,2\}$, $W_4$ implies that there exists $s_a$ such
  that both $W_4^{s_a,a,high}$ and $W_4^{s_a,3-a,low}$ happen.  The
  events $W_4^{s_a,a,high}$, $W_1$, $W_2^{s_a,a,t, q_aT_1}$
  for $t = q_a T_1- \GdT \sqrt{T_1\log(T)}+1, ...,q_aT_1-1$ and
  $W_2^{s_a,a,q_aT_1,t}$ for
  $t= q_aT_1,...,q_a T_1+ \GdT \sqrt{T_1\log(T)}$ together imply that
\begin{align*}
\bar{\mu}_a ^{1,s_a} &\geq \mu_a + \left(q_aT_1 \cdot \frac{1}{\sqrt{q_aT_1}} - \GdT \sqrt{T_1\log(T)} \cdot \sqrt{\frac{2\log(T)}{ \GdT \sqrt{T_1\log(T)}}} \right) \cdot \frac{1}{q_a T_1+ \GdT \sqrt{T_1\log(T)}} \\
&> \mu_a + \frac{1}{4\sqrt{q_aT_1}}.
\end{align*}
The second to the last inequality holds when $T$ is larger than some constant.
Similarly, we also have
\[
\bar{\mu}_{3-a} ^{1,s_a} < \mu_{3-a}   - \frac{1}{4\sqrt{q_{3-a} T_1}}.
\]
This completes the proof.
\end{proof}


Now we proceed to the case analysis.


\begin{proof}[Proof of Lemma~\ref{3levelbigcase} (Large gap case)]
  Observe that for any group $s$ in the first level, the empirical
  means satisfy
\[
\bar{\mu}_1^{1,s} - \bar{\mu}_2^{1,s} \geq \mu_1 -\mu_2 - \sqrt{\frac{4\log(T)}{q_1T_1}} - \sqrt{\frac{4\log(T)}{q_2T_1}} \geq  \sqrt{\frac{4\log(T)}{q_1T_1}} + \sqrt{\frac{4\log(T)}{q_2T_1}}.
\]


For any agent $t$ in the $s$-th second-level group, by Assumption \ref{ass:embehave}, we have
\begin{align*}
\hat{\mu}_1^t - \hat{\mu}_2^t &>\bar{\mu}_1^{1,s} - \bar{\mu}_2^{1,s} - \frac{c_m}{\sqrt{q_1T_1/2}} - \frac{c_m}{\sqrt{q_2T_1/2}}\\
&\geq  \sqrt{\frac{4\log(T)}{q_1T_1}} + \sqrt{\frac{4\log(T)}{q_2T_1}}- \frac{c_m}{\sqrt{q_1T_1/2}} - \frac{c_m}{\sqrt{q_2T_1/2}} > 0
\end{align*}
Therefore, we know agents in the $s$-th second-level group will all pull arm 1.

Now consider the agents in the third level group. Recall $\bar{\mu}_a$
is the empirical mean of arm $a$ in the history they see. We have
\[
\bar{\mu}_1 - \bar{\mu}_2 \geq \mu_1 -\mu_2 - \sqrt{\frac{4\log(T)}{Sq_1T_1}} - \sqrt{\frac{4\log(T)}{Sq_2T_1}} \geq  \sqrt{\frac{4\log(T)}{q_1T_1}} 
+ \sqrt{\frac{4\log(T)}{q_2T_1}}.
\]
Similarly as above, by Assumption \ref{ass:embehave}, we know
$\hat{\mu}_1^t - \hat{\mu}_2^t > 0$ for any agent $t$ in the third
level. Therefore, the agents in the third-level group will all pull
arm 1.  \OMIT{Therefore the expected regret is at most
  $S T_G T_1 = O(T^{4/7} \log^{6/7}(T))$.}
\end{proof}


\begin{proof}[Proof of Lemma~\ref{3levelmedium} (Medium gap case)]
  % $2\left(\sqrt{\frac{4\log(T)}{Sq_1T_1}} +
  %   \sqrt{\frac{4\log(T)}{Sq_2T_1}}\right) \leq \mu_1-\mu_2 <
  % 2\left(\sqrt{\frac{4\log(T)}{q_1T_1}} +
  %   \sqrt{\frac{4\log(T)}{q_2T_1}}\right)$. 
  Recall $\bar{\mu}_a$ is
  the empirical mean of arm $a$ in the first two levels. We have
\[
\bar{\mu}_1 - \bar{\mu}_2 \geq \mu_1 -\mu_2 - \sqrt{\frac{4\log(T)}{Sq_1T_1}} - \sqrt{\frac{4\log(T)}{Sq_2T_1}} \geq  \sqrt{\frac{4\log(T)}{Sq_1T_1}} 
+ \sqrt{\frac{4\log(T)}{Sq_2T_1}}.
\]
For any agent $t$ in the third level, by Assumption \ref{ass:embehave}, we have
\begin{align*}
\hat{\mu}_1^t - \hat{\mu}_2^t &>\bar{\mu}_1 - \bar{\mu}_2 - \frac{c_m}{\sqrt{Sq_1T_1/2}} - \frac{c_m}{\sqrt{Sq_2T_1/2}}\\
&\geq  \sqrt{\frac{4\log(T)}{Sq_1T_1}} + \sqrt{\frac{4\log(T)}{Sq_2T_1}}- \frac{c_m}{\sqrt{Sq_1T_1/2}} - \frac{c_m}{\sqrt{Sq_2T_1/2}}\\
 &> 0.
\end{align*}
So we know agents in the third-level group will all pull arm 1. \OMIT{Therefore the expected regret is at most 
\[
(S T_G T_1 + S T_2) \cdot 2\left(\sqrt{\frac{4\log(T)}{q_1T_1}} 
+ \sqrt{\frac{4\log(T)}{q_2T_1}}\right) = O(T^{4/7} \log^{6/7}(T))
\]
}
\end{proof}


\begin{proof}[Proof of Lemma~\ref{3levelsmallcase} (Small gap case)]
  % $ 3\sqrt{\frac{2\log(T)}{T_2}} < \mu_1-\mu_2 <
  % 2\left(\sqrt{\frac{4\log(T)}{Sq_1T_1}} +
  %   \sqrt{\frac{4\log(T)}{Sq_2T_1}}\right)$.  
  In this case, we need both arms to be pulled at least $T_2$ rounds
  in the second level. For every arm $a$, consider the $s_a$-th
  second-level group, with $s_a$ given by Lemma~\ref{lem:luck}. We
  have
\begin{align*}
\bar{\mu}_a^{1,s_a} - \bar{\mu}_{3-a}^{1,s_a} &> \mu_a + \frac{1}{4\sqrt{q_aT_1}} -\mu_{3-a} +\frac{1}{4\sqrt{q_{3-a}T_1}} \\
&> \frac{1}{4\sqrt{q_1T_1}}+ \frac{1}{4\sqrt{q_2T_1}} - 2\left(\sqrt{\frac{4\log(T)}{Sq_1T_1}} 
+ \sqrt{\frac{4\log(T)}{Sq_2T_1}}\right) \\
&\geq \frac{1}{8\sqrt{q_1T_1}}+ \frac{1}{8\sqrt{q_2T_1}}.
\end{align*}
For any agent $t$ in the $s_a$-th second-level group, by Assumption \ref{ass:embehave}, we have
\begin{align*}
\hat{\mu}_a^t - \hat{\mu}_{3-a}^t &>\bar{\mu}_a^{1,s_a} - \bar{\mu}_{3-a}^{1,s_a} - \frac{c_m}{\sqrt{q_1T_1/2}} - \frac{c_m}{\sqrt{q_2T_1/2}}\\
&\geq   \frac{1}{8\sqrt{q_1T_1}}+ \frac{1}{8\sqrt{q_2T_1}}- \frac{c_m}{\sqrt{q_1T_1/2}} - \frac{c_m}{\sqrt{q_2T_1/2}}\\
 &> 0.
\end{align*}
So we know agents in the $s_a$-th second-level group will all pull arm $a$. Therefore in the first two levels, both arms are pulled at least $T_2$ times. Now consider the third-level. We have
\[
\bar{\mu}_1 - \bar{\mu}_2  \geq \mu_1 -\mu_2 - 2\sqrt{\frac{2\log(T)}{T_2}} \geq \sqrt{\frac{2\log(T)}{T_2}}.
\]
Similarly as above, by Assumption \ref{ass:embehave}, we know $\hat{\mu}_1^t - \hat{\mu}_2^t > 0$ for any agent $t$ in the third level. So we know agents in the third-level group will all pull arm 1.\OMIT{
Therefore the expected regret is at most 
\[
(S T_G T_1 + S T_2) \cdot 2\left(\sqrt{\frac{4\log(T)}{Sq_1T_1}} 
+ \sqrt{\frac{4\log(T)}{Sq_2T_1}}\right) \leq O(T^{4/7} \log^{6/7}(T))
\]
}
\end{proof}
%!TEX root = main.tex

\section{Missing Proofs of Section \ref{sec:llevel}}
\label{sec:llevel-details}
In this section, we design an $L$-level recommendation policy for $L > 3$ and 2 arms. By having more than 3 levels, we get even smaller regret. 

Our recommendation policy has $L$ levels and two types of groups: $G$-groups and $\Gamma$-groups. Each level has $S^2$ $G$-groups for $S = 2^{10}\log(T)$. Label the $G$-groups in the $l$-th level as $G_{l,u,v}$ for $u,v \in [S]$. Level $2$ to level $L$ also have $S^2$ $\Gamma$-groups. Label the $\Gamma$-groups in the $l$-th level as $\Gamma_{l,u,v}$ for $u,v \in [S]$. Each first-level group ($G_{1,u,v}$ for $u,v\in [S]$) has $T_1$ \ALGG of $\GdT$ rounds in parallel. For $l \geq 2$, there are $T_l$ agents in group $G_{l,u,v}$ and there are $T_l (S-1)$ agents in group $\Gamma_{l,u,v}$. We will pick $T_1,...,T_L$ in the proof of Theorem \ref{thm:llevel}.

Finally we define the information flow. Agents in the first level only observe the history defined in the \ALGG run. For agents in group $G_{l,u,v}$ with $l\geq 2$, they observe all the history in the first $l-2$ levels (both $G$-groups and $\Gamma$-groups) and history in group $G_{l-1,v,w}$ for all $w \in [S]$. Agents in group $\Gamma_{l,u,v}$ observe the same history as agents in group $G_{l,u,v}$.
 
\begin{figure}[H]
\centering
\begin{tikzpicture}  
 \foreach \x in {0,3,6,9}
 {
 \filldraw[fill=orange!20!white]
 (\x+0,10)--(\x+0.8,10)--(\x+0.8,11)--(\x+0,11)--cycle;
 \filldraw[fill=orange!30!white]
 (\x+0.8,10)--(\x+2.8,10)--(\x+2.8,11)--(\x+0.8,11)--cycle;
 \filldraw[fill=purple!20!white]
 (\x+0,7.5)--(\x+0.8,7.5)--(\x+0.8,8.5)--(\x+0,8.5)--cycle;
 \filldraw[fill=purple!30!white]
 (\x+0.8,7.5)--(\x+2.8,7.5)--(\x+2.8,8.5)--(\x+0.8,8.5)--cycle;
 \filldraw[fill=green!20!white]
 (\x+0,5)--(\x+0.8,5)--(\x+0.8,6)--(\x+0,6)--cycle;
 \filldraw[fill=green!30!white]
 (\x+0.8,5)--(\x+2.8,5)--(\x+2.8,6)--(\x+0.8,6)--cycle;
 \filldraw[fill=blue!20!white]
 (\x+0,2.5)--(\x+0.8,2.5)--(\x+0.8,3.5)--(\x+0,3.5)--cycle;
 \filldraw[fill=blue!30!white]
 (\x+0.8,2.5)--(\x+2.8,2.5)--(\x+2.8,3.5)--(\x+0.8,3.5)--cycle;
 \filldraw[fill=red!20!white]
 (\x+0,0)--(\x+2.8,0)--(\x+2.8,1)--(\x+0,1)--cycle;
 \draw[dashed] (\x+0.4,0)--(\x+0.4,1); 
 \draw[dashed] (\x+0.8,0)--(\x+0.8,1); 
 \draw[dashed] (\x+1.2,0)--(\x+1.2,1); 
 \draw[dashed] (\x+1.6,0)--(\x+1.6,1);  
 \draw[dashed] (\x+2,0)--(\x+2,1);  
 \draw[dashed] (\x+2.4,0)--(\x+2.4,1);  
 \node at(\x+0.4,3){$T_{l-2}$};
 \node at(\x+1.8,3){$T_{l-2} (S-1)$};
 \node at(\x+0.4,5.5){$T_{l-1}$};
 \node at(\x+1.8,5.5){$T_{l-1} (S-1)$};
 \node at(\x+0.4,8){$T_l$};
 \node at(\x+1.8,8){$T_l (S-1)$};
 \node at(\x+0.4,10.5){$T_L$};
 \node at(\x+1.8,10.5){$T_L (S-1)$};
 \node at(\x+1.4,0.5){$T_1 \cdot \GdT$};
 }
\foreach \y in {0,2.5,5,7.5,10}
{
  \node at (12.5,\y+0.5){$\cdots$};
} 
\foreach \u in {1,2}
{
	\foreach \v in {1,2}
	{
	\pgfmathsetmacro{\x}{((\u-1)*2+(\v-1))*3};
	\pgfmathsetmacro{\xa}{((\u-1))*3};
	\pgfmathsetmacro{\xb}{(2+(\u-1))*3};
   \node at(\x+0.6,7){$G_{l,\u,\v}$};
   \node at(\x+0.6,4.5){$G_{l-1,\u,\v}$};
   \node at(\x+0.6,2){$G_{l-2,\u,\v}$};
   \node at(\x+2,7){$\Gamma_{l,\u,\v}$};
   \node at(\x+2,4.5){$\Gamma_{l-1,\u,\v}$};
   \node at(\x+2,2){$\Gamma_{l-2,\u,\v}$};
   %\draw[dashed] (\x+0.4,3.5)--(\xa+0.4,5);  
   %\draw[dashed] (\x+0.4,3.5)--(\xb+0.4,5);  
   %\draw[dashed] (\x+0.4,3.5)--(\xa+1.8,5);  
   %\draw[dashed] (\x+0.4,3.5)--(\xb+1.8,5);  
   
   
   %\draw[dashed] (\x+0.4,6)--(\xa+0.4,7.5);  
   %\draw[dashed] (\x+0.4,6)--(\xb+0.4,7.5);  
   %\draw[dashed] (\x+0.4,6)--(\xa+1.8,7.5);  
   %\draw[dashed] (\x+0.4,6)--(\xb+1.8,7.5);  
	}
}

   \draw[dashed] (0+1.8,3.5)--(6.4,7.5);  	
   \draw[dashed] (3+1.8,3.5)--(6.4,7.5);  	
   \draw[dashed] (6+1.8,3.5)--(6.4,7.5);  	
   \draw[dashed] (9+1.8,3.5)--(6.4,7.5);  	
   
   %\draw[dashed] (0+0.4,3.5)--(6.4,7.5);  	
   %\draw[dashed] (3+0.4,3.5)--(6.4,7.5);  	
   %\draw[dashed] (6+0.4,3.5)--(6.4,7.5);  	
   %\draw[dashed] (9+0.4,3.5)--(6.4,7.5);  	
   
   \draw[dashed] (0+0.4,6)--(6.4,7.5);  	
   \draw[dashed] (3+0.4,6)--(6.4,7.5);  	
  \node at (6,1.5)[rotate = 90]{$\cdots$};
  \node at (-1,1.5)[rotate = 90]{$\cdots$};
  \node at (6,9.5)[rotate = 90]{$\cdots$};
  \node at (-1,9.5)[rotate = 90]{$\cdots$};
  \node at(-1.2,0.5){\textbf{Level 1}};
  \node at(-1.2,3){\textbf{Level $l-2$}};
  \node at(-1.2,5.5){\textbf{Level $l-1$}};
  \node at(-1.2,8){\textbf{Level $l$}};
  \node at(-1.2,10.5){\textbf{Level $L$}};
  \draw[->] (13.3,0)--(13.3,12);
  \node at(13.7,6)[ rotate=90]{Time};
  \draw [rounded corners=5mm, line width=1mm, red](-0.2,1.75)--(12,1.75)--(12,9)--(-0.2,9)--cycle;\draw [rounded corners=2.5mm, line width=0.5mm, brown](5.9,6.5)--(8.9,6.5)--(8.9,8.8)--(5.9,8.8)--cycle;
\end{tikzpicture}
\caption{$l$-level Recommendation Policy.}
\label{fig:llevel}
\end{figure}

\begin{theorem}
\label{thm:llevel}
The $L$-level recommendation policy gets expected regret $O\left(T^{2^{L-1}/(2^L-1)} \log^2(T) \right)$ for $L \leq \log(\ln(T)/\log(S^4))$. In particular, if we pick $L = \log(\ln(T)/\log(S^4))$, the expected regret is $O(T^{1/2} polylog(T))$. 
\end{theorem}

\begin{proof}
Wlog we assume $\mu_1 \geq \mu_2$ as the recommendation policy is symmetric to both arms. We will set $T_l$'s later in the proof. Before that, we are only going to assume $T_l / T_{l-1} \geq S^4$ for $l=2,...,L-1$ and $T_1 \geq S^4$. 

Similarly as the proof of Theorem \ref{thm:3level}, we start with some clean events.

\begin{itemize}
\item  \textbf{Concentration of the number of arm $a$ pulls in the first level:} 

For $a \in \{1,2\}$, define $q_a$ to be the expected number of arm $a$ pulls in one run of \ALGG used in the first level. By Lemma \ref{lem:greedy}, we know $\GdP \leq q_a \leq \GdT$ For group $G_{1,u,v}$, define $W_1^{a,u,v}$ to be the event that the number of arm $a$ pulls in this group is between $q_a T_1- \GdT \sqrt{T_1\log(T)}$ and $q_a T_1 + \GdT \sqrt{T_1\log(T)}$. By Chernoff bound,
\[
\Pr[W_1^{a,u,v}] \geq 1-2\exp(-2\log(T)) \geq 1-2/T^2.
\]
Define $W_1$ to be the intersection of all these events (i.e. $W_1 = \bigcap_{a,u,v}W_1^{a,u,v}$). By union bound, we have
\[
\Pr[W_1] \geq 1- \frac{4S^2}{T^2}.
\]


\item \textbf{Concentration of the empirical mean of arm $a$ pulls in the history observed by agent $t$:}

 For each agent $t$ and arm $a$, imagine there is a tape of enough arm $a$ pulls sampled before the recommendation policy starts and these samples are revealed one by one whenever agents in agent $t$'s observed history pull arm $a$.  Define $W_2^{t,a,t_1,t_2}$ to be the event that the mean of $t_1$-th to $t_2$-th pulls in the tape is at most $\sqrt{\frac{3\log(T)}{t_2-t_1+1}}$ away from $\mu_a$. By Chernoff bound, 
\[
\Pr[W_2^{t,a,t_1,t_2}] \geq 1 - 2\exp(-6\log(T)) \geq 1- 2/T^6.
\]

Define $W_2$ to be the intersection of all these events (i.e. $W_2 = \bigcap_{t,a,t_1,t_2} W_2^{t,a,t_1,t_2}$). By union bound, we have
\[
\Pr[W_2] \geq 1- \frac{4}{T^3}.
\]


\item \textbf{Anti-concentration of the empirical mean of arm $a$ pulls in the $l$-th level  for $l \geq 2$:}

For $2\leq l \leq L-1$, $u\in [S]$ and each arm $a$, define $n^{l,u,a}$ to be the number of arm $a$ pulls in groups $G_{l,u,1},...,G_{l,u,S}$. Define $W_3^{l,u,a,high}$ as the event that $n^{l,u,a} \geq T_l$ implies the empirical mean of arm $a$ pulls in group $G_{l,u,1},...,G_{l,u,S}$ is at least $\mu_a + 1/\sqrt{n^{l,u,a}}$. Define $W_3^{l,u,a,low}$ as the event that $n^{l,u,a} \geq T_l$ implies the empirical mean of arm $a$ pulls in group $G_{l,u,1},...,G_{l,u,S}$ is at most $\mu_a - 1/\sqrt{n^{l,u,a}}$.

Define $H_l$ to be random variable the history of all agents in the first $l-1$ levels and which agents are chosen in the $l$-th level. Let $h_l$ be some realization of $H_l$. Notice that once we fix $H_l$, $n^{l,u,a}$ is also fixed. 

Now consider $h_l$ to be any possible realized value of $H_l$. If fixing $H_l= h_l$ makes $n^{l,u,a}<T_l$, then $\Pr[W_3^{l,u,a,high} |H_l = h_l]=1$  If fixing $H_l = h_l$ makes $n^{l,u,a} \geq T_l$, by Berry-Esseen Theorem and $\mu_a \in [1/3,2/3]$, we have
\[
\Pr[W_3^{l,u,a,high}|H_l = h_l] \geq (1-\Phi(1/2)) - \frac{5}{\sqrt{T_l}} > 1/4.
\]
Similarly we also have
\[
\Pr[W_3^{l,u,a,low}|H_l = h_l]  > 1/4
\]
Since $W_3^{l,u,a,high}$ is independent with $W_3^{l,u,3-a,low}$ when fixing $H_l$, we have
\[
\Pr[ W_3^{l,u,a,high} \cap W_3^{l,u,3-a,low}|H_l = h_l]  > (1/4)^2 = 1/16.
\]
Now define $W_3^{l,a} = \bigcup_u (W_3^{l,u,a,high} \cap W_3^{l,u,3-a,low})$. Since  $(W_3^{l,u,a,high} \cap W_3^{l,u,3-a,low})$ are independent across different $u$'s when fixing $H_l=h_l$, we have
\[
\Pr[W_3^{l,a}|H_l= h_l] \geq 1- (1-1/16)^S \geq 1 - 1/T^2.
\]
Since this holds for all $h_l$'s, we have $\Pr[W_3^{l,a}] \geq 1-1/T^2$. Finally define $W_3 = \bigcap_{l,a} W_3^{l,a}$. By union bound, we have
\[
W_3 \geq 1 - 2L/T^2.
\]

\item \textbf{Anti-concentration of the empirical mean of arm $a$ pulls in the first level:}

For first-level groups $G_{1,u,1},...,G_{1,u,S}$ and arm $a$, imagine there is a tape of enough arm $a$ pulls sampled before the recommendation policy starts and these samples are revealed one by one whenever agents in these groups pull arm $a$. Define $W_4^{u,a,high}$  to be the event that first $q_a T_1 S$ pulls of arm $a$ in the tape has empirical mean at least $\mu_a + 1/\sqrt{q_a T_1 S}$ and define $W_4^{u,a,low}$  to be the event that first $q_a T_1S$ pulls of arm $a$ in the tape has empirical mean at most $\mu_a - 1/\sqrt{q_a T_1S }$. By Berry-Esseen Theorem and $\mu_a \in [1/3,2/3]$, we have
\[
\Pr[W_4^{u,a,high}] \geq (1-\Phi(1/2)) - \frac{5}{\sqrt{q_aT_1S}} > 1/4.
\]
The last inequality follows when $T$ is larger than some constant.
Similarly we also have 
\[
\Pr[W_4^{u,a,low}] > 1/4.
\]
Since $W_4^{u,a,high}$ is independent with $W_4^{u,3-a,low}$, we have
\[
\Pr[W_4^{u,a,high} \cap W_4^{u,3-a,low}] =\Pr[W_4^{u,a,high}] \cdot  \Pr[W_4^{u,3-a,low}]>(1/4)^2 = 1/16.
\]
Now define $W^{a}_4$ as $\bigcup_u (W_4^{u,a,high} \cap W_4^{u,3-a,low})$. Notice that $(W_4^{u,a,high} \cap W_4^{u,3-a,low})$ are independent across different $u$'s. So we have
\[
\Pr[W^{a}_4] \geq 1- (1-1/16)^S \geq 1 -1/T^2.
\]
Finally we define $W_4$ as $\bigcap_{a} W^{a}_4$. By union bound,
\[
\Pr[W_4] \geq 1- 2/T^2.
\]
\end{itemize}

By union bound, the intersection of these clean events (i.e. $\bigcap_{i=1}^4 W_i$) happens with probability $1-O(1/T)$. When this intersection does not happen, since the probability is $O(1/T)$, it cost expected regret $O(1/T) \cdot T = O(1)$. 

Now we assume the intersection of clean events happens and prove upper bound on the expected regret.

By event $W_1$, we know that in each first-level group, there are at least $q_a T_1- \GdT \sqrt{T_1\log(T)}$ pulls of arm $a$. We prove in the next claim that there are enough pulls of both arms in higher levels if $\mu_1-\mu_2$ is small enough. For notation convenience, we set $\varepsilon_0 = 1$, $\varepsilon_1 = \frac{1}{4\sqrt{q_aT_1S}} + \frac{1}{4\sqrt{q_{3-a} T_1S}}$ and $\varepsilon_l = 1/(4\sqrt{T_lS})$ for $l \geq 2$. 

\begin{claim}
\label{clm:l2_explore}
For any arm $a$ and $2\leq l \leq L$, if $\mu_1 - \mu_2 \leq \varepsilon_{l-1}$, then for any $u \in [S]$, there are at least $T_l$ pulls of arm $a$ in groups $G_{l,u,1},G_{l,u,2}, ... ,G_{l,u,S}$ and there are at least $T_lS(S-1)$ pulls of arm $a$ in the $l$-th level $\Gamma$-groups.
\end{claim}

\begin{proof}
We are going to show that for each $l$ and arm $a$ there exists $u_a$ such that agents in groups $G_{l,1,u_a},...,G_{l,S,u_a}$ and $\Gamma_{l,1,u_a},...,\Gamma_{l,S,u_a}$ all pull arm $a$. This suffices to prove the claim.

We prove the above via induction on $l$. %For notation convenience, define $\bar{\mu}^{l,u}_a$  to be the empirical mean of arm $a$ pulls in the history observed by agents in groups $G_{l,1,u},...,G_{l,S,u}$ and $\Gamma_{l,1,u},...,\Gamma_{l,S,u}$ (they observe the same history).
We start by the base case when $l=2$. For each arm $a$, $W_4$ implies there exists $u_a$ such that $W^{u_a,a,high}_4$ and $W^{u_a,3-a,low}_4$ happen. For an agent $t$  in groups $G_{2,1,u_a},...,G_{2,S,u_a}$ and $\Gamma_{2,1,u_a},...,\Gamma_{2,S,u_a}$.
$W_4^{u_a,a,high}$,  $W_1^{a,u_a,v}$ and $W_2$ together imply that 
\begin{align*}
\bar{\mu}_a ^t &\geq \mu_a + \left(q_aT_1S \cdot \frac{1}{\sqrt{q_aT_1S}} - \GdT \sqrt{T_1\log(T)} S\cdot \sqrt{\frac{3\log(T)}{ \GdT \sqrt{T_1\log(T)}S}} \right) \cdot \frac{1}{(q_a T_1+ \GdT \sqrt{T_1\log(T)})S} \\
&> \mu_a + \frac{1}{4\sqrt{q_aT_1S}}.
\end{align*}
The second last inequality holds when $T$ is larger than some constant.
Similarly, we also have
\[
\bar{\mu}_{3-a}^t< \mu_{3-a}   - \frac{1}{4\sqrt{q_{3-a} T_1S}}.
\]
Then we have
\begin{align*}
\bar{\mu}^t_a - \bar{\mu}^t_{3-a} &> \mu_a - \mu_{3-a} + \frac{1}{4\sqrt{q_aT_1S}} + \frac{1}{4\sqrt{q_{3-a} T_1S}}\\
&\geq -\varepsilon_1+ \frac{1}{4\sqrt{q_aT_1S}} + \frac{1}{4\sqrt{q_{3-a} T_1S}}\\
&\geq \frac{1}{8\sqrt{q_aT_1S}} + \frac{1}{8\sqrt{q_{3-a} T_1S}}.
\end{align*}
By Assumption \ref{ass:embehave}, we have
\begin{align*}
\hat{\mu}_a^t - \hat{\mu}_{3-a}^t &> \bar{\mu}^t_a - \bar{\mu}^t_{3-a} -  \frac{c_m}{\sqrt{q_aT_1S/2}} - \frac{c_m}{\sqrt{q_{3-a} T_1S/2}}\\
&> \frac{1}{8\sqrt{q_aT_1S}} + \frac{1}{8\sqrt{q_{3-a} T_1S}} -   \frac{c_m}{\sqrt{q_aT_1S/2}} - \frac{c_m}{\sqrt{q_{3-a} T_1S/2}}\\
&>0.
\end{align*}
The last inequality holds since $c_m$ is a small enough constant defined in Assumption \ref{ass:embehave}. Therefore we know agents in groups $G_{2,1,u_a},...,G_{2,S,u_a}$ and $\Gamma_{2,1,u_a},...,\Gamma_{2,S,u_a}$ all pull arm $a$.

Now we consider the case when $l > 2$ and assume the claim is true for smaller $l$'s. For each arm $a$, $W_3$ implies that there exists $u_a$ such that $W^{l-1,u_a,a,high}_3$ and $W^{l-1,u_a,3-a,low}_3$ happen. Recall $n^{l-1,u_a,a}$ is the number of arm $a$ pulls in groups $G_{l-1,u_a,1},...,G_{l-1,u_a,S}$. The induction hypothesis implies that $n^{l-1,u_a,a} \geq T_{l-1}$. $W^{l-1,u_a,a,high}_3$ together with $n^{l-1,u_a,a} \geq T_{l-1}$ implies that the empirical mean of arm $a$ pulls in group $G_{l-1,u_a,1},...,G_{l-1,u_a,S}$ is at least $\mu_a + 1/\sqrt{n^{l-1,u_a,a}}$. For any agent $t$ in groups $G_{l,1,u_a},...,G_{l,S,u_a}$ and $\Gamma_{l,1,u_a},...,\Gamma_{l,S,u_a}$, it observes history of groups $G_{l-1,u_a,1},...,G_{l-1,u_a,S}$ and all groups in levels below level $l-1$. Notice that the groups in the first $l-2$ levels have at most $(T_1 \GdT + T_2 + \cdots +T_{l-2})S^3 \leq T_{l-1}/(12\log(T)) \leq n^{l-1,u_a,a}/(12\log(T))$ agents. By $W_2$, we have
\begin{align*}
\bar{\mu}_a ^t &\geq \mu_a + \left(n^{l-1,u_a,a}  \cdot \frac{1}{\sqrt{n^{l-1,u_a,a} }}- (T_1 \GdT + T_2 + \cdots +T_{l-2})S^3\cdot \sqrt{\frac{3\log(T)}{ (T_1 \GdT + T_2 + \cdots +T_{l-2})S^3}} \right) \\
&~~~\cdot \frac{1}{n^{l-1,u_a,a}+ (T_1 \GdT + T_2 + \cdots +T_{l-2})S^3} \\
&> \mu_a + \frac{1}{4\sqrt{n^{l-1,u_a,a}  }}.
\end{align*}
The third last inequality holds when $T$ larger than some constant.
Similarly, we also have
\[
\bar{\mu}_{3-a}^t < \mu_{3-a}   -\frac{1}{4\sqrt{n^{l-1,u_a,3-a}  }}.
\]
Then we have
\begin{align*}
\bar{\mu}^t_a - \bar{\mu}^t_{3-a} &> \mu_a - \mu_{3-a}+ \frac{1}{4\sqrt{n^{l-1,u_a,a}  }} +\frac{1}{4\sqrt{n^{l-1,u_a,3-a}  }}\\
&\geq -\varepsilon_{l-1}+ \frac{1}{4\sqrt{n^{l-1,u_a,a}  }} +\frac{1}{4\sqrt{n^{l-1,u_a,3-a}  }}\\
&\geq \frac{1}{8\sqrt{n^{l-1,u_a,a}  }} +\frac{1}{8\sqrt{n^{l-1,u_a,3-a}  }}.
\end{align*}
The last inequality holds because $n^{l-1,u_a,a}$ and $n^{l-1,u_a,3-a}$ are at most $T_{l-1} S$. By Assumption \ref{ass:embehave}, we have
\begin{align*}
\hat{\mu}_a^t - \hat{\mu}_{3-a}^t &> \bar{\mu}^t_a - \bar{\mu}^t_{3-a} -  \frac{c_m}{\sqrt{n^{l-1,u_a,a} }} - \frac{c_m}{\sqrt{n^{l-1,u_a,3-a}}}\\
&> \frac{1}{8\sqrt{n^{l-1,u_a,a}  }} +\frac{1}{8\sqrt{n^{l-1,u_a,3-a}  }} -  \frac{c_m}{\sqrt{n^{l-1,u_a,a} }} - \frac{c_m}{\sqrt{n^{l-1,u_a,3-a}}}\\
&>0.
\end{align*}
The last inequality holds since $c_m$ is a small enough constant defined in Assumption \ref{ass:embehave}.
Therefore agents in groups $G_{l,1,u_a},...,G_{l,S,u_a}$ and $\Gamma_{l,1,u_a},...,\Gamma_{l,S,u_a}$ all pull arm $a$.
\end{proof}

\begin{claim}
\label{clm:l2_exploit}
For any $2 \leq l \leq L$, if $\varepsilon_{l-1} S\leq \mu_1 - \mu_2 < \varepsilon_{l-2} S$, there are no pulls of arm 2 in groups with level $l,...,L$. 
\end{claim}

\begin{proof}
We argue in 2 cases $\varepsilon_{l-1} \sqrt{S} \leq \mu_1 - \mu_2 \leq \varepsilon_{l-2}$ for $l \geq 2$ and $\varepsilon_{l-2}  \leq \mu_1 - \mu_2 \leq \varepsilon_{l-2} \sqrt{S}$ for $l > 2$. Since our recommendation policy's first level is slightly different from other levels, we need to argue case $\varepsilon_{l-1} \sqrt{S} \leq \mu_1 - \mu_2 \leq \varepsilon_{l-2}$ for $l=2$ and case $\varepsilon_{l-2}  \leq \mu_1 - \mu_2 \leq \varepsilon_{l-2} \sqrt{S}$ for $l =3$ separately.

\begin{itemize}
\item $\varepsilon_{l-1} S \leq \mu_1 - \mu_2 \leq \varepsilon_{l-2}$ for $l = 2$(i.e. $\varepsilon_1S\leq \mu_1 - \mu_2 \leq \varepsilon_0$): We know agents in level at least 2 will observe at least $q_aT_1/2$ pulls of arm $a$ for $a \in \{1,2\}$. By $W_2$, for any agent in level at least 2, we have
\[
|\bar{\mu}_a^t - \mu_a| \leq \sqrt{\frac{3\log(T)}{Sq_aT_1/2}}.
\]
By Assumption \ref{ass:embehave}, we have
\begin{align*}
\hat{\mu}_1^t - \hat{\mu}_2^t &\geq \bar{\mu}_1^t - \bar{\mu}_2^t - \frac{c_m}{\sqrt{Sq_1T_1/2}} -  \frac{c_m}{\sqrt{Sq_2T_1/2}}\\
&\geq \mu_1 -\mu_2-\sqrt{\frac{3\log(T)}{Sq_1T_1/2}} - \sqrt{\frac{3\log(T)}{Sq_2T_1/2}}- \frac{c_m}{\sqrt{Sq_1T_1/2}} -  \frac{c_m}{\sqrt{Sq_2T_1/2}}\\
&\geq\frac{\sqrt{S}}{4\sqrt{q_1T_1}} +  \frac{\sqrt{S}}{4\sqrt{q_2T_1}}-\sqrt{\frac{3\log(T)}{Sq_1T_1/2}} - \sqrt{\frac{3\log(T)}{Sq_2T_1/2}}- \frac{c_m}{\sqrt{Sq_1T_1/2}} -  \frac{c_m}{\sqrt{Sq_2T_1/2}}\\
&>0.
\end{align*}
Therefore agents in level at least 2 will all pull arm 1. 

\item $\varepsilon_{l-1} S \leq \mu_1 - \mu_2 \leq \varepsilon_{l-2}$ for $l > 2$: By claim \ref{clm:l2_explore}, for any agent $t$ in level at least $l$, that agent will observe at least $T_{l-1}$ arm $a$ pulls. By $W_2$, we have
\[
|\bar{\mu}_a^t - \mu_a| \leq \sqrt{\frac{3\log(T)}{T_{l-1}}}.
\]
By Assumption \ref{ass:embehave}, we have
\begin{align*}
\hat{\mu}_1^t - \hat{\mu}_2^t &\geq \bar{\mu}_1^t - \bar{\mu}_2^t - \frac{2c_m}{\sqrt{T_{l-1}}} \\
&\geq \mu_1 -\mu_2 - 2 \sqrt{\frac{3\log(T)}{T_{l-1}}}- \frac{2c_m}{\sqrt{T_{l-1}}} \\
&\geq\sqrt{\frac{S}{16T_{l-1}}} -  2 \sqrt{\frac{3\log(T)}{T_{l-1}}}- \frac{2c_m}{\sqrt{T_{l-1}}} \\
&>0.
\end{align*}
Therefore agents in level at least $l$ will all pull arm 1. 

\item $\varepsilon_{l-2} < \mu_1 - \mu_2 < \varepsilon_{l-2}S$ for $l =3$ (i.e. $\varepsilon_1 < \mu_1 - \mu_2 < \varepsilon_1S$): By Claim \ref{clm:l2_explore}, for any agent $t$ in level at least $3$, that agent will observe at least $T_1q_aS^2/2$ arm $a$ pulls (just from the first level). By $W_2$, we have
\[
|\bar{\mu}_a^t - \mu_a| \leq \sqrt{\frac{3\log(T)}{S^2q_aT_1/2}}.
\]
By Assumption \ref{ass:embehave}, we have
\begin{align*}
\hat{\mu}_1^t - \hat{\mu}_2^t &\geq \bar{\mu}_1^t - \bar{\mu}_2^t - \frac{c_m}{\sqrt{S^2q_1T_1/2}}-\frac{c_m}{\sqrt{S^2q_2T_1/2}}  \\
&\geq \mu_1 -\mu_2 -  \sqrt{\frac{3\log(T)}{S^2q_1T_1/2}}- \sqrt{\frac{3\log(T)}{S^2q_2T_1/2}}- \frac{c_m}{\sqrt{S^2q_1T_1/2}}-\frac{c_m}{\sqrt{S^2q_2T_1/2}}  \\
&\geq\frac{1}{4\sqrt{Sq_1T_1}} + \frac{1}{4\sqrt{Sq_2T_1}}  -  \sqrt{\frac{3\log(T)}{S^2q_1T_1/2}}- \sqrt{\frac{3\log(T)}{S^2q_2T_1/2}}- \frac{c_m}{\sqrt{S^2q_1T_1/2}}-\frac{c_m}{\sqrt{S^2q_2T_1/2}}  \\
&>0.
\end{align*}
Therefore agents in level at least 3 will all pull arm 1. 

\item $\varepsilon_{l-2} < \mu_1 - \mu_2 < \varepsilon_{l-2}S$ for $l >3$: Since $\mu_1-\mu_2 < \varepsilon_{l-2}S < \varepsilon_{l-3}$, by Claim \ref{clm:l2_explore}, for any agent $t$ in level at least $l$, that agent will observe at least $T_{l-2}S^2$ arm $a$ pulls (just from level $l-2$). By $W_2$, we have
\[
|\bar{\mu}_a^t - \mu_a| \leq \sqrt{\frac{3\log(T)}{S^2T_{l-2}}}.
\]
By Assumption \ref{ass:embehave}, we have
\begin{align*}
\hat{\mu}_1^t - \hat{\mu}_2^t &\geq \bar{\mu}_1^t - \bar{\mu}_2^t - \frac{2c_m}{\sqrt{S^2T_{l-2}}} \\
&\geq \mu_1 -\mu_2 - 2 \sqrt{\frac{3\log(T)}{S^2T_{l-2}}}- \frac{2c_m}{\sqrt{S^2T_{l-2}}} \\
&\geq\frac{1}{4\sqrt{ST_{l-2}}} -  2 \sqrt{\frac{3\log(T)}{T_{l-1}}}- \frac{2c_m}{\sqrt{T_{l-1}}} \\
&>0.
\end{align*}
Therefore agents in level at least $l$ will all pull arm 1. 
\end{itemize}
\end{proof}

Now we set the group sizes $T_l$'s as following. For $l < L$,
\[
T_l = T^{\frac{2^{L-1} + 2^{L-2} + \cdots + 2^{L-l}}{2^{L-1}+ 2^{L-2} + \cdots + 1}}/S^3.
\]
and 
\[
T_L = (T - T_1 \cdot \GdT\cdot S^2 - (T_2 + \cdots + T_{l-1}) S^3) /S^3
\]
We restrict $L$ to be at most $\log(\ln(T)/\log(S^4))$ so that $T_l / T_{l-1} \geq T^{1/2^L} \geq S^4$ for $l = 2,...,L-1$. $T_L$ is a little bit different because we want total number of agents to be $T$. 

By Claim \ref{clm:l2_exploit}, we know that the expected regret conditioned the intersection of clean events is at most 
\begin{align*}
&\max\left( T_1T_GS^2 , \max_{l \geq 2} \varepsilon_{l-1}S(T_1T_GS^2 + T_2 S^3 + \cdots + T_lS^3)\right) \\
\leq & \max\left( T_1T_GS^2 , \max_{l \geq 2} 2 \varepsilon_{l-1} T_l S^4 \right)\\
= &O\left(T^{2^{L-1}/(2^L-1)} \log^2(T) \right).
\end{align*}
\end{proof}

Now we are going to change the parameters of the $L$-level recommendation policy a little bit and prove the below corollary. We will keep $S$ the same (i.e. $S = 2^{10}\log(T)$). We are going to change $L$ and $T_1,...,T_L$. We set $L = \log(T)/\log(S^4)$, $T_l = (S^4)^l$ for $l=1,...,L-1$ and $T_L = (T - T_1 \GdT S^2 - S^3\sum_{l=2}^{L-1} T_l)/S^3$.  

\begin{corollary}
\label{cor:llevel}
With the proper setting of $L$ and $T_1,...,T_L$ described above, the $L$-level recommendation policy gets expected regret $O(\min(1/\Delta, T^{1/2})polylog(T))$. Here $\Delta = |\mu_1 -\mu_2|$ and the $L$-level recommendation policy does not need to know $\Delta$. Moreover, agent $t$ observes a subhistory of size at least $\Omega( \lfloor t/polylog(T)\rfloor)$. 
\end{corollary}

\begin{proof}
Notice that in the proof of Theorem \ref{thm:llevel}, by the end of Claim \ref{clm:l2_exploit}, the only constraint we need about $T_l$'s is that $T_l / T_{l-1} \geq S^4$ for $l=2,...,L-1$ and $T_1 \geq S^4$. And our new settings of $T_l$'s still satisfy this constraint. So we can reuse the proof of Theorem \ref{thm:llevel} till the end of Claim \ref{clm:l2_exploit}.

Recall in the proof of Theorem \ref{thm:llevel}, $\varepsilon_l =\Theta(1/\sqrt{T_l S})$ for $l \in [L-1]$ and $\varepsilon_0 = 1$. Consider two cases:
\begin{itemize}
\item $\Delta < \varepsilon_{L-1} S$. In this case, notice that even always picking the sub-optimal arm gives expected regret at most $T(\mu_1-\mu_2) = T\Delta = O(T^{1/2} polylog(T))$. On the other hand, $T^{1/2} = O(polylog(T)/\Delta)$. Therefore, the expected regret is $O(\min(1/\Delta, T^{1/2})polylog(T))$.
\item $\Delta \geq \varepsilon_{L-1} S$. In this case, we can find $l \in \{2,...,L\}$ such that $\varepsilon_{l-1} S\leq \Delta < \varepsilon_{l-2} S$. By Claim \ref{clm:l2_exploit}, we can upper bound the expected regret by
\begin{align*}
&\Delta \cdot (T_1 \GdT S^2  +T_2 S^3+ \cdots T_{l-1} S^3)\\
=& O(\Delta T_{l-1}S^3) \\
=&O(\Delta T_{l-2}S^7) \\
=&O(\Delta\cdot  \frac{1}{\varepsilon^2_{l-2}}\cdot S^6)\\
=&O(\Delta \cdot \frac{1}{\Delta^2}\cdot S^8)\\
=&O(polylog(T)/\Delta).
\end{align*}
We also have $1/\Delta \leq 1/(\varepsilon_{L-1}S)= O(T^{1/2})$. Therefore, the expected regret is \\$O(\min(1/\Delta, T^{1/2})polylog(T))$.
\end{itemize}

Finally we discuss about the subhistory sizes. We know that agents in level $l$ observes the history of all agents below level $l-2$ (including level $l-2$). It is easy to check that the ratio between the number of agents below level $l$ and the number of agents below level $l-2$ is bounded by $O(polylog(T))$. Therefore our statement about the subhistory sizes holds.
\end{proof}

Here we discuss about how to extend Theorem \ref{thm:llevel} and Corollary \ref{cor:llevel} to the case when $K$ is a constant larger than 2. As the proof is very similar to the proofs of Theorem \ref{thm:llevel} and Corollary \ref{cor:llevel}, we only provide a proof sketch of what changes to make.

\begin{theorem}
\label{thm:constarm}
Theorem \ref{thm:llevel} and Corollary \ref{cor:llevel} can be extended to the case when $K$ is constant larger than 2. In the extension of Corollary \ref{cor:llevel}, $\Delta$ is defined as the difference between means of the best and the second best arm.
\end{theorem}

\xhdr{Proof Sketch.}
We still wlog assume arm 1 has the highest mean (i.e. $\mu_1 \geq \mu_a, \forall a \in \A$. We first extend the clean events (i.e. $W_1,W_2,W_3,W_4$) in Theorem \ref{thm:llevel} to the case when $K$ is larger than 2. $W_1$ and $W_2$ extend naturally: we still set $W_1 = \bigcap_{a,s}W_1^{a,s}$ and $W_2 = \bigcap_{t,a,t_1,t_2} W_2^{t,a,t_1,t_2}$. The difference is that now $a$ is taken over $K$ arms instead of 2 arms. For $W_3$, we change the definition $W_3^{l,a} = \bigcup_u \left(W_3^{l,u,a,high}  \cap \left(\bigcap_{a' \neq a} W_3^{l,u,a',low}\right) \right)$ and $W_3 = \bigcap_{l,a} W_3^{l,a}$. We extend $W_4$ in a similar way: define $W^{a}_4$ as $\bigcup_u \left(W_4^{u,a,high} \cap \left(\bigcap_{a' \neq a} W_4^{u,a',low}\right) \right)$ and $W_4 = \bigcap_a W^a_4$. Since $K$ is a constant, it's easy to check that the same proof technique shows that the intersection of these clean events happen with probability  $1-O(1/T)$. So the case when some clean event does not happen contributes $O(1)$ to the expected regret. 

Now we proceed to extend Claim \ref{clm:l2_explore} and Claim \ref{clm:l2_exploit}. The statement of Claim \ref{clm:l2_explore} should be changed to ``For any arm $a$ and $2\leq l \leq L$, if $\mu_1 - \mu_a \leq \varepsilon_{l-1}$, then for any $u \in [S]$, there are at least $T_l$ pulls of arm $a$ in groups $G_{l,u,1},G_{l,u,2}, ... ,G_{l,u,S}$ and there are at least $T_lS(S-1)$ pulls of arm $a$ in the $l$-th level $\Gamma$-groups''. The statement of Claim \ref{clm:l2_exploit} should be changed to ``For any $2 \leq l \leq L$, if $\varepsilon_{l-1} S\leq \mu_1 - \mu_a < \varepsilon_{l-2} S$, there are no pulls of arm $a$ in groups with level $l,...,L$.'' 

The proof of Claim \ref{clm:l2_exploit} can be easily changed to prove the new version by changing ``arm 2'' to ``arm $a$''. The proof of Claim \ref{clm:l2_explore} needs some additional argument. In the proof of Claim \ref{clm:l2_explore}, we show that $\hat{\mu}_a^t - \hat{\mu}_{3-a} > 0 $ for agent $t$ in the chosen groups. When extending to more than 2 arms, we need to show $\hat{\mu}_a^t - \hat{\mu}_{a'}^t > 0$ for all arm $a' \neq a$. The proof of Claim \ref{clm:l2_explore} goes through if $\mu_1- \mu_{a'} \leq \varepsilon_{l-2}$ since then there will be enough arm $a'$ pulls in level $l-1$. We need some additional argument for the case when $\mu_1 - \mu_{a'} > \varepsilon_{l-2}$. Since $\mu_1- \mu_{a'} > \varepsilon_{l-2} > \varepsilon_{l-1}S$, we can use the same proof of Claim \ref{clm:l2_exploit} (which rely on Claim \ref{clm:l2_explore} but for smaller $l$'s) to show that there are no arm $a'$ pulls in level $l$ and therefore $\hat{\mu}_a^t - \hat{\mu}_{a'}^t > 0$. 

Finally we proceed to bound the expected regret conditioned on the intersection of clean events happens. The proofs of Theorem \ref{thm:llevel} and Corollary \ref{cor:llevel} bound it by consider the regret from pulling the suboptimal arm (i.e. arm 2). When extending to more than 2 arms, we can do the exactly same argument for all arms except arm 1. This will blow up the expected regret by a factor of $(K-1)$ which is a constant.
\qed

 



%%!TEX root = main.tex
\section{Posterior Mean}
\jmcomment{Do we want this section?}

\begin{example}
Let the prior distribution be the uniform distribution over Bernoulli distributions. Suppose there are $a$ 1's in $n$ samples. Then the posterior mean is $\frac{a+1}{n+2}$ and the empirical mean is $\frac{a}{n}$. Their difference is at most $\frac{1}{n+2}$. 
\end{example}

\jmcomment{This example is against us. Should delete later.}
\begin{example}
Let the prior distribution be the uniform distribution over Bernoulli distributions with mean between $1/3$ and $2/3$. Suppose we have $n$ samples. With probability at least $\Omega(1/\sqrt{n})$, the difference between empirical mean and posterior mean is larger than $1/\sqrt{n}$.
\end{example}

\begin{example}
Let the prior distribution be the sum of two independent 0-mean Gaussians. The first Gaussian is sampled once and has variance $\alpha^2$. The second Gaussian is sampled for each sample and has variance $\beta^2$. Suppose there are $n$ samples with empirical mean $m$. The posterior mean is $\frac{n \cdot m \cdot \alpha^2}{n \cdot \alpha^2 + \beta^2}$. The difference between the posterior mean and the empirical mean is $\frac{m\beta^2}{n\cdot \alpha^2 + \beta^2}$.
\end{example}

\jmcomment{Should modify this assumption a bit.}
\begin{assumption}
\label{ass:post}
For any arm $i$ and any $m >0$, let $H_m$ be the random variables of $m$ pulls of arm $i$. Let $\emn(H_m)$ be the empirical mean and $\pmn(H_m)$ be the posterior mean. With probability at least $1 - \exp(m)$, we have $|\emn(H_m) - \pmn(H_m)| \leq \frac{1}{m}$. 
\end{assumption}

\begin{lemma}
\label{lem:post}
For any arbitrary strategy of pulling arms for $T$ rounds, let its history to be $G_T$. We allow this strategy to be adaptive based on history. Fix an arm $i$. Let $m_i(G_T)$ be the number of pulls of arm $i$, $\pmn_i(G_T)$ be the posterior mean of arm $i$. Let $H_{m(G_T)}$ be the sub history of arm $i$ in $G_T$. Then $\pmn_i(G_T)$ is the same as the posterior mean of seeing $H_{m_i(G_T)}$ after $m_i(G_T)$ pulls of arm $i$.
\end{lemma}

\begin{proof}
Let's assume there are $m^0_i(G_T)$ 0's and $m^1_i(G_T)$ 1's among of $m_i(G_T)$ pulls of arm $i$ in $G_T$. By the definition of posterior mean, we have
\[
\pmn_i(G_T) = \E[\mu_i|G_T] = \frac{\sum_{x_i} x_i \cdot \Pr[\mu_i=x_i, G_T]}{\sum_{x_i} \Pr[\mu_i=x_i, G_T]}.
\]

We also know that 
\[
\frac{\Pr[\mu_i = x_i, G_T]}{\Pr[\mu_i = x_i', G_T]} = \frac{\Pr[\mu_i = x_i] \cdot  (1-x_i)^{m^0_i(G_T)}x_i^{m^1_i(G_T)}}{\Pr[\mu_i = x_i'] \cdot (1-x_i)^{m^0_i(G_T)}(x_i')^{m^1_i(G_T)} }.
\]

Therefore ,
\[
\pmn_i(G_T) = \frac{\sum_{x_i} x_i \cdot \Pr[\mu_i = x_i] \cdot  (1-x_i)^{m^0_i(G_T)}x_i^{m^1_i(G_T)}}{\sum_{x_i}\Pr[\mu_i = x_i] \cdot  (1-x_i)^{m^0_i(G_T)}x_i^{m^1_i(G_T)}}
\]

Finally, by the definition of posterior mean, the posterior mean of seeing $H_{m_i(G_T)}$ after $m_i(G_T)$ pulls of arm $i$ is
\[
\frac{\sum_{x_i} x_i \cdot \Pr[\mu_i = x_i] \cdot  (1-x_i)^{m^0_i(G_T)}x_i^{m^1_i(G_T)}}{\sum_{x_i}\Pr[\mu_i = x_i] \cdot  (1-x_i)^{m^0_i(G_T)}x_i^{m^1_i(G_T)}} = \pmn_i(G_T).
\]
\end{proof}

\begin{corollary}
\label{cor:post}
For any arbitrary strategy of pulling arms for $T$ rounds, let its history to be $G_T$. We allow this strategy to be adaptive based on history. Fix an arm $i$. Let $m(G_T)$ be the number of pulls of arm $i$, $\emn(G_T)$ be the empirical mean of arm $i$ and $\pmn(G_T)$ be the posterior mean of arm $i$. Let $p_0$ be the probability that $m(G_T) \geq m_0$. With probability $p_0 - 2\exp(m_0)$, $|\emn(G_T) - \pmn(G_T)| \leq \frac{1}{m(G_T)}$.
\end{corollary}

\begin{proof}
Simply by applying Assumption \ref{ass:post}, Lemma \ref{lem:post} and union bound, we can prove the corollary.
\end{proof}

\end{document}

%%% Local Variables:
%%% mode: latex
%%% TeX-master: t
%%% End:

\documentclass[anon,12pt]{colt2019}
%\usepackage{fullpage}
\usepackage{mathrsfs}
\usepackage{float}
\usepackage{times}
% ======    PACKAGES
\usepackage{slivkins-setup}
%,slivkins-theorems}
%\usepackage{amsmath, amsfonts, amssymb, amsthm, amsbsy, amscd, bm, bbm}
%\usepackage{amsbsy,amscd, bm, bbm}
\usepackage{kpfonts}
\usepackage{array}
%\usepackage{booktabs}
\usepackage{graphicx}
%\usepackage[small,bf]{caption}
%\setlength{\captionmargin}{30pt}
%\usepackage{subcaption}
%\captionsetup[sub]{margin=10pt,font=small}
\usepackage{color}
\usepackage{ifthen}
\usepackage{xspace}
\usepackage[noend]{algorithmic}
\usepackage{algorithm}
%\usepackage[colorlinks,citecolor={black},urlcolor={black},linkcolor={black}]{hyperref}
\usepackage{url}
%\usepackage{tocbibind}
\usepackage{enumerate}
\usepackage{mdframed}
\usepackage{comment}
\usepackage{tikz}
\usetikzlibrary{decorations.pathreplacing}
\usepackage{cleveref}

\newtheorem{assumption}[theorem]{Assumption}
\newtheorem{construction}[theorem]{Construction}
\newtheorem{claim}[theorem]{Claim}
\DeclareMathOperator*{\argmin}{argmin}

% a very useful package for edits and comments, from David Kempe (USC)
%\usepackage{color-edits}
%\usepackage[suppress]{color-edits}  % use this to suppress the package
%\addauthor{as}{red}      % as for Alex
%\addauthor{jm}{blue}     % jm for Jieming
%\addauthor{ni}{magenta}     % ni for Nicole
%\addauthor{sw}{brown}     % sw for Steven
% e.g. for Alex, provides \asedit{}, \ascomment{} and \asdelete{}.


%%%%%%%%%%%%%%%%%%%%%%%%%%%%%%%%%%
% \newtheorem{theorem}{Theorem}[section]
% \newtheorem{corollary}[theorem]{Corollary}
% \newtheorem{conjecture}[theorem]{Conjecture}
% \newtheorem{proposition}[theorem]{Proposition}
% \newtheorem{definition}[theorem]{Definition}
% \newtheorem{lemma}[theorem]{Lemma}
% \newtheorem{remark}{Remark}[section]
% \newtheorem{claim}{Claim}[section]
% \newtheorem{example}{Example}[section]


\newcommand{\term}[1]{\ensuremath{\mathtt{#1}}\xspace}
\newcommand{\thread}{\term{thread}}
\newcommand{\ALGG}{full-disclosure path }
\newcommand{\ALGPG}{\term{Parallel Greedy}}

% Alex's notation!
\newcommand{\SubH}[1]{\mathcal{H}_{#1}} % subhistory
\newcommand{\AnonSubH}[1]{H^{\texttt{anon}}_{#1}} % subhistory

% new notation for FDPs (full-disclosure paths)
\newcommand{\fdp}{\term{FDP}}
\newcommand{\fdpL}{L^\fdp_K} % sufficient length of greedy path
\newcommand{\fdpP}{p^\fdp_K} % success prob for greedy path
\newcommand{\fdpN}[1][a]{N^\fdp_{K,#1}}
    % Exp #pulls for arm $a$ in greedy path of length L_K.

% new notation for low-deviation assumption on agents' reward estimates
\newcommand{\estN}{N_{\term{est}}}
\newcommand{\estC}{C_{\term{est}}}

% notation for 3level section
\newcommand{\conf}[1]{\mathtt{conf}\left(#1\right)}
\newcommand{\event}[1]{\ensuremath{\mathtt{event}_{#1}}\xspace}
\newcommand{\cG}{\mathcal{G}}

% Greedy notation
\def\GdT{L^\fdp_K} % sufficient length of greedy path
\def\GdP{p^\fdp_K} % success prob for greedy path

\def\2LEVEL{two-level policy}

\newcommand{\DT}{\mD} % Alex' notation for distribution over types.


%number of groups
\def\NG{\sigma}

\def\DKL{\textbf{D}_{\term{KL}}}
\def\D{\mathbb{D}}
\def\E{\mathbb{E}}
\def\reg{\mathrm{Reg}}
\def\A{\mathcal{A}}
\def\M{\mathcal{M}}
\def\S{\mathcal{S}}
\def\X{\mathcal{X}}
\def\EX{\term{EX}}
\def\Ds{*}
\def\EE{\mathcal{E}}
\def\varTheta{\bold{\Theta}}
\def\varOmega{\bold{\Omega}}
\def\cD{\mathcal{D}}
\def\cT{\mathcal{T}}
\def\Ber{\emph{Ber} }

\def\pmn{\hat{\mu}}
\def\emn{\bar{\mu}}
%\def\q_a{\fdpN}
%\def\ell{l}

%index on the tape
\def\z{\tau}
\title{Incentivizing Exploration with Unbiased Histories}

% Authors with different addresses:
\coltauthor{
\Name{Nicole Immorlica} \Email{nicimm@microsoft.com}\\
\addr Microsoft Research, Cambridge, MA.
\and
\Name{Jieming Mao} \Email{maojm517@gmail.com}\\
\addr University of Pennsylvania, Philadelphia, PA.
\and
\Name{Aleksandrs Slivkins}  \Email{slivkins@microsoft.com}\\
\addr Microsoft Research, New York, NY.
\and
\Name{Zhiwei Steven Wu} \Email{zsw@umn.edu}\\
\addr University of Minnesota, Minneapolis, MN.
}
\begin{document}
\begin{titlepage}
\maketitle

\thispagestyle{empty}
\begin{abstract}
In a social learning setting, there is a set of actions, each of which has a payoff that depends on a hidden state of the world. A sequence of agents each chooses an action with the goal of maximizing payoff given estimates of the state of the world.  A disclosure policy tries to coordinate the choices of the agents by sending messages about the history of past actions.  The goal of the algorithm is to minimize the regret of the action sequence.

In this paper, we study a particular class of disclosure policies that use messages, called {\em unbiased subhistories}, consisting of the actions and rewards from by a subsequence of past agents, where the subsequence is chosen ahead of time. One trivial message of this form contains the full history; a disclosure policy that chooses to use such messages risks inducing herding behavior among the agents and thus has regret linear in the number of rounds.  Our main result is a disclosure policy using unbiased subhistories that obtains regret $\tilde{O}(\sqrt{T})$.  We also exhibit simpler policies with higher, but still sublinear, regret.  These policies can be interpreted as dividing a sublinear number of agents into constant-sized focus groups, whose histories are then fed to future agents.
\end{abstract}
\begin{keywords}%
  incentivizing exploration, multi-armed bandit, social learning
\end{keywords}
\end{titlepage}
\section{Introduction}
\label{sec:intro}




In the classic literature on multi-armed bandits, an agent repeatedly selects one of a set of actions, each of which has a payoff drawn from an unknown fixed distribution.  Over time, she can trade off {\em exploitation}, in which she picks an action to maximize her expected reward, with {\em exploration}, in which she takes potentially sub-optimal actions to learn more about their rewards.  By coordinating her actions across time, she can guarantee an average reward which converges to that of the optimal action in hindsight at a rate proportional to the inverse square-root of the time horizon.

In many decision problems of interest, the actions are not chosen by a single agent, as above, but rather a sequence of agents.  In such settings, each agent will choose an exploitive action as the benefits of explorative actions are only accrued by future agents.  For example, in online retail, products are purchased by a sequence of customers, each of which buys what she estimates to be the best available product.  This behavior can cause herding, in which all agents eventually take a sub-optimal action of maximum expected payoff given the available information.

This situation can be circumvented by a centralized algorithm that induces agents to take explorative actions, a line of work called {\em incentivizing exploration}.  One way to induce exploration is to introduce payments~\cite{Frazier-ec14,Kempe-colt18}. For example, an online retailer can give coupons to agents for trying certain products.  When payments are financially or technologically infeasible, another alternative is to rely on {\em information asymmetry}, \eg \cite{Kremer-JPE14,Che-13,ICexploration-ec15,Bimpikis-exploration-ms17}.
Here the idea is that the centralized algorithm can choose to selectively release information about the past actions and rewards to the agents in the form of a {\em message}.  Importantly, agents can not directly observe the past, but only learn about it through this message.  The agent then chooses an action, using the content of the message as input.  These systems are often referred to as {\em recommendation systems} as the content of the message can be interpreted as a recommendation for a particular action.

\ascomment{Nicole: It would be good to introduce "disclosure policy" (= online algorithm that chooses messages, in the context of incentivizing exploration). And also talk about recommendation system as smth that exists regardless of incentivizing exploration, and is very common/important out there.}

\OMIT{ %%% Alex's original text
Recommendation systems are ubiquitous, providing recommendations for movies (\eg  Netflix), products (\eg  Amazon), restaurants (\eg  Yelp), vacations (\eg Tripadvisor), etc.  A typical recommendation system elicits user feedback about their experiences, and aggregates this feedback in order to provide better recommendations in the future. Thus, each user she consumes information from the previous users (indirectly, \eg via recommendations), and produces new information (\eg  a review) that benefits future users. This dual role creates a three-way tension between exploration, exploitation, and users' incentives.
A social planner would balance ``exploration" of insufficiently known alternatives and ``exploitation" of the information acquired so far. Designing algorithms to trade off these two objectives is a well-researched subject in machine learning and operations research. However, a given user who decides to ``explore" typically suffers all the downside of this decision, whereas the upside (improved recommendations) is spread over many users in the future. Therefore, users' incentives are skewed in favor of exploitation. As a result, observations may be collected at a slower rate, and may suffer from selection bias (\eg  ratings of a particular movie may mostly come from people who like this type of movies). Moreover, in some natural but idealized models (\eg  \cite{Kremer-JPE14,ICexploration-ec15}), there are simple  examples when optimal recommendations are never found because the corresponding actions are never taken.

Thus, we have a problem of \emph{incentivizing exploration}. Providing monetary incentives can be financially or technologically unfeasible, and relying on voluntary exploration can lead to selection biases. A recent line of work, started by \cite{Kremer-JPE14}, relies on the inherent \emph{information asymmetry} between the recommendation system and a user. These papers posit a simple model, termed \emph{Bayesian Exploration} in \cite{ICexplorationGames-ec16}. The recommendation system is a ``principal" that interacts with self-interested ``agents" who arrive one by one. Each agent needs to make a decision: take an action from a given set of alternatives. The principal sends a \emph{message} to the agent according to some ``disclosure policy", \eg issues a recommendation. Then the agent chooses an action, and both the agent and the principal observe the outcome.  Crucially, the principal does not control the agent's decision. The problem is to design the disclosure policy for the principal that learns over time to issue messages so as to incentivize the agents to balance exploration and exploitation in a socially optimal way.
A single round of this model is a version of a well-known ``Bayesian Persuasion game" \cite{Kamenica-aer11}.
} %%%%%%%%

\xhdr{Our scope.} Prior work on incentivizing exploration, with or without monetary incentives, achieves much progress (more on this in ``related work"), but relies heavily on the standard assumptions of Bayesian rationality and the ``power to commit" (\ie users trust that the principal actually implements the policy that it claims to implement). However, these assumptions appear quite problematic in the context of recommendation systems. In particular, disclosure policies studied in prior work merely recommend an action to each agent, without any other supporting information, and moreover recommend exploratory actions to some randomly selected users. This works out extremely well in theory, but it is very unclear whether users would trust the principal to implement the stated policy and, even if they do, whether they would react to it rationally. Several issues are in play: to wit, whether the principal intentionally uses a different disclosure policy than the claimed one (\eg because its incentives are not quite aligned with the users'), whether the principal correctly implements the policy that it wants to implement, whether the users trust the principal to make correct inferences on their behalf, and whether they find it acceptable that they may be singled out for exploration. Furthermore, regardless of how the users react to such disclosure policies, they may prefer not to be subjected to them, and leave the system.

We strive to design disclosure policies which mitigate these issues and (still) incentivize a good balance between exploration and exploitation. While some assumptions on human behavior are unavoidable, we are looking for a class of disclosure policies for which we can make plausible behavioral assumptions. Then we arrive at a concrete mathematical problem: design policies from this class so as to optimize performance, \ie  the induced explore-exploit tradeoff. Our goal in terms of performance is as ambitious as the one in prior work on incentivizing exploration: essentially, to approach the performance of the social planner.

\ascomment{stable text up to here.}

\xhdr{Our contributions.}
We build one such conceptual framework for a particular type of disclosure policies. Consider the \emph{full-disclosure policy} that reveals the full history of observations from the previous users. We interpret it as the ``gold standard": we posit that users would trust such policy, even if they cannot verify it. Unfortunately, the full-disclosure policy is not good for our purposes, essentially because rational users would \emph{exploit} rather than \emph{explore}. However, for the sake of the argument, what if the policy reveals the history for every other agent, rather than the history for all agents? We posit that users would trust such policy, too. Given a large volume of data, we posit that users would not be too unhappy with having access to only a fraction this data. A crucial aspect of our intuition here is that the ``subhistory" revealed to a given user is chosen in advance, without looking at the observations. In particular, the disclosure policy cannot subsample the observations that make a particular action look good.






\xhdr{Related work.}


\cite{Perchet2015BatchedBP} gives regret guarantees for mult-armed bandit in $M$ batches. They have 2 upper bounds which are similar to our Theorem \ref{thm:llevel-1} and Theorem \ref{thm:llevel-2}.

%!TEX root = main.tex
\section{Model and Preliminaries}
\label{sec:model}

\newcommand{\SubH}[1]{H_{#1}} % subhistory
\newcommand{\AnonSubH}[1]{H^{\texttt{anon}}_{#1}} % subhistory


We study the multi-armed bandit problem in a social learning context, in which a principal faces a sequence of $T$ myopic agents. There is a set $\A$ of $K$ possible actions, a.k.a. \emph{arms}. At each round $t\in [T]$, a new agent $t$ arrives, receives a message $m_t$ from the principal, chooses an arm $a_t\in \A$, and collects a reward $r_t\in \{0,1\}$ that is immediately observed by the principal. The reward from pulling an arm $a\in \A$ is drawn independently from Bernoulli distribution $\cD_a$ with an unknown mean $\mu_a$. An agent does not observe anything from the previous rounds, other than the message $m_t$. The problem instance is defined by (known) parameters $K,T$ and the (unknown) tuple of mean rewards, $(\mu_a:\,a\in\A)$. We are interested in \emph{regret}, defined as
\[\textstyle 
  \reg(T)
  %= \reg(a_1, \ldots , a_T) 
  = T \max_{a\in \A} \mu_a -
  \sum_{t\in [T]} \E[\mu_{a_t}].
\]
The principal chooses messages $m_t$ according to an online algorithm called \emph{disclosure policy}, with a goal to minimize regret. We assume that mean rewards are bounded away from $0$ and $1$, to ensure sufficient entropy in rewards. For concreteness, we posit that
    $\mu_a\in [\tfrac13,\tfrac23]$.

\xhdr{Unbiased subhistories.}  
The \emph{subhistory} for a subset of rounds $S\subset [T]$, and the corresponding \emph{anonymized subhistory} are defined, respectively, as
\[ \SubH{S} = \{\; (s,a_s,r_s):\;s\in S \;\}
\quad\text{and}\quad
   \AnonSubH{S} = \{\; (a_s,r_s):\;s\in S \}.
\]
Accordingly, $\SubH{[t-1]}$ is called the \emph{full history} at time $t$.


We focus on disclosure policies of a particular form, where the message in each round $t$ is $m_t = \SubH{S_t}$ for some subset $S_t\subset [t-1]$. We assume that the subset $S_t$ is chosen ahead of time, before round $1$ (and therefore does not depend on the observations $\SubH{t-1}$). Such message is called \emph{unbiased subhistory}, and the resulting disclosure policy is called an \emph{unbiased-history policy}.

Further, we assume that disclosure policies are \emph{transitive}, in the following sense:
\[ t'\in S_t \Rightarrow S_{t'}\subset S_t
    \qquad \text{for all rounds $t,t'\in [T]$}. \]
In words, if agent observes the subhistory for some previous agent, then she observes the entire message for that agent. 

A transitive unbiased-history policy $\pi$ can be represented as an undirected graph $G_\pi$, where nodes correspond to rounds, and any two rounds $t'<t$ are connected if and only if $t'\in S_t$ and there is no intermediate round $t''$ with
    $t'\in S_{t''}$ and $t''\in S_t$.
This graph is henceforth called the \emph{information flow graph} of policy $\pi$, or \emph{info-graph} for short.

\xhdr{Agents' behavior.} Let us define agents' behavior in response to an unbiased-history policy. We posit that each agent $t$ uses its observed subhistory $m_t$ to form a reward estimate $\hat{\mu}_{t,a} \in [0,1]$ for each arm $a\in \A$, and chooses an arm with a maximal estimator. (Ties are broken according to an arbitrary rule that is the same for all agents.)
We make some mild assumptions:

%If an arm is not pulled in $H_t$, we will let $\bar{\mu}_a^t = 0$.

\begin{assumption}\label{ass:embehave}
Reward estimates satisfy the following assumptions:
\begin{itemize}
\item[(a)] Reward estimates are close to empirical averages. Let $N_{t,a}$ and $\bar{\mu}_{t,a}$ denote the number of pulls and the empirical mean reward of arm $a$ in subhistory $m_t$. Then for some absolute constant $N_0\in \N$ and $C_0=\tfrac{1}{16}$ it holds that
\[
\forall t\in[T],\, a\in\A \qquad
N_{t,a} \geq N_0 \quad\Rightarrow\quad
    \left|\hat{\mu}^t_a - \bar{\mu}^t_a \right| <
		\frac{C_0}{\sqrt{N_{t,a}}},
\]
\item[(b)] Reward estimates have a warm-start property:
\[
\forall t\in[T],\, a\in\A \qquad
N_{t,a} =0 \quad\Rightarrow\quad
    \hat{\mu}^t_a \geq \tfrac13.
\]

\item[(c)] Reward estimates $\hat{\mu}_a^t$ depend only on the anonymized subhistory $\AnonSubH{S_t}$. That is, reward estimates depend only on the observations, not on the rounds in which these observations are collected.

\item[(c)] Reward estimates are consistent over time, but allow some agent heterogeneity. Formally, each agent $t$ forms its estimates according to a \emph{reward estimate function}: a function $f_t$ from anonymous subhistories to $[0,1]^K$, so that the estimate vector
        $(\hat{\mu}_{t,a}:\, a\in\A)$
    equals $f_t(\AnonSubH{S_t})$. And this function is drawn independently from some fixed distribution over reward estimate functions.
\end{itemize}
\end{assumption}

\OMIT{ %%%%%%%%%%
The first assumption ensures that the reward distributions have sufficient entropy to induce natural exploration. We choose the bounded range $[1/3, 2/3]$ for the simplicity of our analysis, and it can further relaxed to $[1/C, (C-1)/C]$ for any constant $C$. The second assumption says that the estimates computed by the agents are well-behaved, and are close to the empirical estimates given by the sub-history, provided that the number of observations is sufficiently large.

We model agents with heterogeneity in their arm selections. In particular, there is an unknown distribution over the set of agent estimators satisfying Assumption~\ref{ass:embehave}.  Each agent $t$ indepedently draws an estimator from this distribution, uses it to calculate the mean reward estimates $\hat{\mu}_a^t$ for every arm $a$, and then chooses the arm $a_t$ with the highest estimate.
% \swcomment{why independently? can be be
%   adversarial?} \jmcomment{Actually current proofs don't work when it's adversarial. It's fairly annoying for the anti-concentration argument. In the proof, we need to use independent to show that the amount of agents pulling arm $a$ in the first level is concentrated to $T_1 q_a$.}

\begin{remark}
We emphasize that the agents we consider in this paper are {\em frequentists}.  Thus their estimators, which determine their behavior, take samples as inputs and not priors.  The estimators satisfying Assumption~\ref{ass:embehave} include that of the natural greedy frequentist, who always pulls the arm with the highest empirical mean.
\end{remark}
} %%%%%%%

\xhdr{Connection to multi-armed bandits.}
The special case when each message $m_t$ is an arm, and the $t$-th agent always chooses this arm, corresponds to a standard multi-armed bandit problem with IID rewards. Following the literature on bandits, we define the \emph{gap parameter} $\Delta$ as the difference between the largest and second largest mean rewards.%
\footnote{Formally, the second-largest mean reward is 
    $\max_{a\in\A:\mu(a)<\mu^*} \mu(a)$,
where $\mu^* = \max_{a\in\A} \mu(a)$.
}
The gap parameter is not known to the principal (in our problem), or to the algorithm (in the bandit problem). However, it is essential for regret bounds. 

\ascomment{We'll need to state/discuss optimal regret bounds somewhere, probably here.}



%%% Local Variables:
%%% mode: latex
%%% TeX-master: "main"
%%% End:


%%!TEX root = main.tex
\section{\ALGG: Revealing Full History}

\swedit{Before we describe our algorithms, we will first analyze a
  simple policy \ALGG \nicomment{as jieming and i discussed offline, this is a bad name for a policy; policies just determine the history and so aren't ``greedy'' in any sense i can see} that fully discloses the history over all
  previous rounds\nidelete{, and lets the agents make the ``greedy'' choices
  across all rounds}. Even though the algorithm does not guarantee low
  regret, it will be used as a subroutine in our main algorithms.  In
  particular, we show that because of the randomness in the rewards,
  with constant probability, \ALGG will induce the agents to
  play each arm at least once.} \nicomment{as we're talking about behavioral agents and not necessarily rational ones, i find it strange to talk about ``greedy'' agents -- which suggests to me that agents use an estimator equal to the empirical mean -- and to use the word ``incentivize'' (i changed it to ``induce'' in the previous paragraph).}

\begin{lemma}
\label{lem:greedy}
Under Assumption \ref{ass:embehave}, there exist two constants $\GdT = O_K(1)$
and $\GdP = \Omega_K(1)$ such that for any arm $a$, with probability
at least $\GdP$, \ALGG of $\GdT$ rounds pulls this arm at least
once.\footnote{The notations $O_k$ and $\Omega_K$ hide the dependence
  on the number of arms $K$.} \nicomment{If we change the name of the policy, we might also want to change the subscripts of these constants.}
\end{lemma}

\begin{proof}
  Fix any arm $a$. Let $\GdT = (K-1) \cdot c_T + 1$ and
  $\GdP = (1/3)^{\GdT}$. \swedit{We will condition on the event that
    all the realized rewards in $\GdT$ rounds are 0, which occurs with
    probability at least $\GdP$ under Assumption~\ref{ass:embehave}.}
  In this case, we want to show that arm $a$ is pulled at least
  once. We prove this by contradiction. Suppose arm $a$ is not pulled. By
  the pigeonhole principle, we know that there is some other arm $a'$
  that is pulled at least $c_T + 1$ rounds. Let $t$ be the round in
  which arm $a'$ is pulled exactly $c_T + 1$ times. By Assumption
  \ref{ass:embehave}, we know
  \[
    \hat{\mu}_{a'}^t \leq 0 + c_m / \sqrt{c_T} < 1/3. 
  \]
  \nicomment{Wait, where did we set $c_T$?} On the other hand, we have
  $\hat{\mu}_a^t \geq 1/3 > \hat{\mu}_{a'}^t$. This contradicts with
  the fact that in round $t$, arm $a'$ is pulled, instead of arm $a$.
  \swdelete{ In addition, we know that this case happens with
    probability at least $(1/3)^{\GdT} = \GdP$ as each arm's mean is
    at most $2/3$. To sum up, we know that with probability at least
    $\GdP$, \ALGG of $\GdT$ rounds pulls arm $a$ at least once.}
   \jmcomment{Agree with these changes.}
\end{proof}


\swedit{In our policies presented in the next sections, we will use
  \ALGG of $T_G$ rounds as a subroutine for initial exploration. For
  any $a$, let $q_a$ \nicomment{$q$ sounds like a fraction; maybe $n^G_a$ or $N^G_a$} be the expected number of arm $a$ pulls in one
  run of \ALGG of $T_G$ rounds.} \nicomment{\ALGG of $T_G$ rounds is a mouthful; what about using notation like \ALGG($T_G$)? or if we will always use $T_G$ rounds, then just define the policy to be that from now on?}


%%% Local Variables:
%%% mode: latex
%%% TeX-master: "main"
%%% End:


%!TEX root = main.tex
\section{Warm-up: full-disclosure paths}

We first consider a disclosure policy that reveals the full history in each round $t$, \ie $m_t=\SubH{t-1}$; we call it the \emph{full-disclosure policy}. The info-path for this policy is a simple path. We use this policy as a ``gadget" in our constructions. Hence, we formulate it slightly more generally:
  
\begin{definition}
A subset of rounds $S\subset [T]$ is called a \emph{full-disclosure path} in the  info-graph $G$ if the induced subgraph $G_S$ is a simple path, and it connects to the rest of the graph only through the terminal node $\max(S)$, if at all.
\end{definition}
  
We prove that for a constant number of arms, with constant probability, a full-disclosure path of constant length suffices to sample each arm at least once. We will build on this fact throughout.

\begin{lemma}\label{lem:greedy}
There exist numbers $\GdT>0$ and $\GdP>0$ that depend only on $K$, the number of arms, with the following property. Consider an arbitrary disclosure policy, and let $S\subset [T]$ be a full-disclosure path in its info-graph, of length $|S|\geq \GdT$. Under Assumption \ref{ass:embehave}, with probability at least $\GdP$, subhistory $\SubH{S}$ contains at least once sample of each arm $a$.
\end{lemma}

\begin{proof}[Proof Sketch]
\end{proof}

\ascomment{Where do we need to introduce the expected \#pulls in full-disclosure path?}

We provide a simple disclosure policy based on full-disclosure paths. The policy follows the ``explore-then-exploit'' paradigm. The ``exploration phase" comprises the first $N = T_1\cdot \GdT$ rounds, and consists of $T_1$ full-disclosure paths of length $\GdT$ each. Here $\GdT$ is the constant from Lemma~\ref{lem:greedy}, and $T_1$ is a parameter. In the ``exploitation phase", each agent $t>N$ receives the full subhistory from exploration, \ie $m_t = \SubH{[N]}$. The info-graph for this disclosure policy is shown in Figure~\ref{fig:2level}.

\begin{figure}[H]
\centering
\begin{tikzpicture}
 \filldraw[fill=blue!20!white]
 (0,2)--(10,2)--(10,3)--(0,3)--cycle;
  \filldraw[fill=red!20!white]
  (0,0)--(1,0)--(1,1)--(0,1)--cycle;
  \draw (0.5,1)--(5,2);
  \filldraw[fill=red!20!white]
  (1,0)--(2,0)--(2,1)--(1,1)--cycle;
  \draw (1.5,1)--(5,2);
  \filldraw[fill=red!20!white]
  (2,0)--(3,0)--(3,1)--(2,1)--cycle;
  \draw(2.5,1)--(5,2);
  \filldraw[fill=red!20!white]
  (3,0)--(4,0)--(4,1)--(3,1)--cycle;
  \draw(3.5,1)--(5,2);
  \filldraw[fill=red!20!white]
  (9,0)--(10,0)--(10,1)--(9,1)--cycle;
  \draw(9.5,1)--(5,2);
  \node at(5,0.5){$\cdots$};
  \node at(6,0.5){$\cdots$};
  \node at(7,0.5){$\cdots$};
  \node at(8,0.5){$\cdots$};
  \node at(5,2.5){$T -T_1 \cdot \GdT$ rounds};
  \node at(0.5,0.5){$\GdT$};
  \node at(1.5,0.5){$\GdT$};
  \node at(2.5,0.5){$\GdT$};
  \node at(3.5,0.5){$\GdT$};
  \node at(9.5,0.5){$\GdT$};
  \node at(-1,0.5){\textbf{Level 1}};
  \node at(-1,2.5){\textbf{Level 2}};
  \draw[->] (11,0)--(11,3);
  \node at(11.5,1.5)[ rotate=90]{Time};

  \draw [decorate,decoration={brace,amplitude=10pt},xshift=0pt,yshift=0pt] (10,-0.2) -- (0,-0.2) node [black,midway,yshift=-0.6cm] 
  {$T_1$ full-disclosure paths of length $\GdT$ each};
\end{tikzpicture}

\caption{Info-graph for the 2-level policy. }
%Each red box in level 1 corresponds to a path connecting a set of
%  $T_G$ agents. The entire history in level 1 is then aggregated and
%  shown to each agent in level 2.}
\label{fig:2level}
\end{figure}


It is useful to think of the info-graph as having two levels (corresponding to exploration and exploitation). Accordingly, we call this policy the \emph{2-level policy}. We show that it incentivizes the agents to perform non-adaptive exploration, and achieves a regret rate of  $\tilde O_K(T^{2/3})$. The key idea is that using many full-disclosure paths ``in parallel" ensures that sufficiently many samples of each arm are collected during exploration.

\begin{theorem}\label{thm:2level}
The 2-level policy with parameter $T_1 = T^{2/3}\,\log(T)^{1/3}$ achieves regret
\[ \reg(T) \leq O_K\left( T^{2/3}\, \log(T)^{1/3} \right).\]
\end{theorem}

\begin{remark}
For a constant $K$, the number of arms, we match the optimal regret rate for non-adaptive multi-armed bandit algorithms. If the gap parameter $\Delta$ is known to the principal, then (for an appropriate tuning of parameter $T_1$) we can achieve regret 
  $\reg(T) \leq O_K(\log(T) \cdot \Delta^{-2})$.
\end{remark}


\begin{lemma}[Concentration of arm pulls]\label{lem:t1runs}
  Consider $T_1$ indepdent runs of \ALGG of $T_G$ rounds. Let $N_a$ be
  the number of arm $a$ pulls among all runs. Then with probability at
  least $1-\delta$, for all $a\in \A$,
  \[
    \left| N_a - q_a T_1\right| \leq T_G \sqrt{T_1 \log(2K/\delta) / 2}
  \]
\end{lemma}






%%% Local Variables:
%%% mode: latex
%%% TeX-master: "main"
%%% End:


%!TEX root = main.tex

\section{3-level Adaptive Policy}
\label{sec:3level}
\swcomment{rewrote below}
Now we demonstrate how to design a 3-level policy to incentivize
adaptive exploration and improve the $O(T^{2/3})$ regret rate to
$\tilde O(T^{4/7})$. In \Cref{sec:llevel}, we will show how to extend
this idea to achieve a regret rate of $\tilde O(T^{1/2})$. For
simplicity, we will focus on the setting with 2 arms.


The policy consists of three phases, each of which corresponds to a
level (see Figure~\ref{fig:3level}).  In the first level, the policy
performs the first round of exploration over the first
$(S\cdot T_1\cdot T_G)$ agents, where $T_1 = T^{4/7}\log^{-1/7}(T)$
and $S = 2^{10}\log(T)$. In particular, the policy randomly partitions
the agents into $S$ groups such that each group consists of $T_1$ runs
of \ALGG with $T_G$ rounds. In the second level, the policy performs
further exploration on the next $(S\cdot T_2)$ agents by incorporating
data collected in the first level, where
$T_2 = T^{6/7}\log^{-5/7}(T)$. The agents are again partitioned into
$S$ groups such that the agents in the $s$-th group are shown the same
sub-history from the $s$-th group in the first level. In the third
level, the policy exploits by showing all remaining agents the entire
history from the first two levels.


They main idea for the 3-level policy is that it inserts the second
level as an additional ``checkpoint'' to induce more adaptivity in the
exploration. While the first level ensures that the agents explore
both arms, the second level will continue to explore both arms only if
the rewards of the two arms are still indistinguishable (that is, when
$|\mu_1 - \mu_2| < \tilde{O}(1/\sqrt{T_1})$). In the third level,
agents will pull the best arm unless $|\mu_1-\mu_2|$ is much smaller
than $\tilde{O}(1/\sqrt{T_2})$, in which case pulling either arm will
incur small regret. The formal guarantee of the policy is the
following.

\begin{theorem}
\label{thm:3level}
The 3-level recommendation policy gets expected regret $O(T^{4/7} \log^{6/7}(T))$. 
\end{theorem}



In the proof, we need not only the concentration argument as the
2-level recommendation policy, but also the anti-concentration
argument to show that agents explore more in the second round if
$\mu_1-\mu_2$ is not too big.


\iffalse
Set $T_1 = T^{4/7}\log^{-1/7}(T)$ and $T_2 =
T^{6/7}\log^{-5/7}(T)$. The first level has $S = 2^{10}\log(T)$ groups
of agents. Each group runs $T_1$ \ALGG of $T_G$ rounds in
parallel. The second level also has $S$ groups of agents. Each group
has $T_2$ agents and they observe the history of corresponding group
in the first level. All the rest agents are in the third level and
they observe the entire history of the first two levels.  See also
Figure \ref{fig:3level} for a graphical view of the information flow.



Here we give some intuition about how the 3-level recommendation
policy works. The main idea is to do the exploration more
adaptively. In the first level, agents explore both arms. In the
second level, agents will pull the best arm when $\mu_1 - \mu_2$ is
large ($\tilde{\Omega}(1/\sqrt{T_1})$). Otherwise agents will explore
both arms more. In the third level, agents will pull the best arm
unless $\mu_1-\mu_2$ is much smaller ($\tilde{O}(1/\sqrt{T_2})$). In
the proof, we need not only the concentration argument as the 2-level
recommendation policy, but also the anti-concentration argument to
show that agents explore more in the second round if $\mu_1-\mu_2$ is
not too big.
\fi


\begin{figure}[H]
\centering
\begin{tikzpicture}  
 \filldraw[fill=green!20!white]
 (0,4)--(10,4)--(10,5)--(0,5)--cycle;
 \foreach \x in {0,3,8}
 {
 \filldraw[fill=blue!20!white]
 (\x+0,2)--(\x+2,2)--(\x+2,3)--(\x+0,3)--cycle;
 \draw[dashed] (\x+1,3)--(5,4);
 \filldraw[fill=red!20!white]
 (\x+0,0)--(\x+2,0)--(\x+2,1)--(\x+0,1)--cycle;
 \draw[dashed] (\x+1,1)--(\x+1,2);
 \draw[dashed] (\x+0.2,1)--(\x+1,2);
 \draw[dashed] (\x+0.6,1)--(\x+1,2);
 \draw[dashed] (\x+1.4,1)--(\x+1,2);
 \draw[dashed] (\x+1.8,1)--(\x+1,2);
 \draw[dashed] (\x+0.4,0)--(\x+0.4,1); 
 \draw[dashed] (\x+0.8,0)--(\x+0.8,1); 
 \draw[dashed] (\x+1.2,0)--(\x+1.2,1); 
 \draw[dashed] (\x+1.6,0)--(\x+1.6,1);  
 \node at(\x+1,2.5){$T_2$};
 \node at(\x+0.2, 0.5){$\GdT$};
 \node at(\x+0.6, 0.5){$\cdot$};
 \node at(\x+1.0, 0.5){$\cdot$};
 \node at(\x+1.4, 0.5){$\cdot$};
 \node at(\x+1.8, 0.5){$\GdT$};
 %\node at(\x+1,0.5){$T_1 \cdot \GdT$}; 
 \draw [decorate,decoration={brace,amplitude=10pt},xshift=0pt,yshift=0pt] (\x+2,-0.2) -- (\x+0,-0.2) node [black,midway,yshift=-0.6cm] {$T_1$ runs};
 }
  \node at(5,4.5){$T -S(T_1 \cdot \GdT + T_2)$};
  \node at (6,0.5){$\cdots$};
  \node at (7,0.5){$\cdots$};
  \node at (6,2.5){$\cdots$};
  \node at (7,2.5){$\cdots$};
  \node at(-1,0.5){\textbf{Level 1}};
  \node at(-1,2.5){\textbf{Level 2}};
  \node at(-1,4.5){\textbf{Levle 3}};
  \draw[->] (11,0)--(11,5);
  \node at(11.5,2.5)[ rotate=90]{Time};
  
  \draw [decorate,decoration={brace,amplitude=10pt,aspect=0.33},xshift=0pt,yshift=0pt] (10,1.8) -- (0,1.8) node [black,pos=0.33,xshift = 0cm,yshift=-0.6cm] {$S$ groups};
  
\end{tikzpicture}
\caption{Structure of the information graph for the 3-level
  Policy. Each red box in level 1 corresponds to $T_1$ \ALGG paths of
  length $T_G$. The sub-history of each red box is then aggregated and
  sent to a corresponding purple box of agents in level 2. Then the
  entire history of levels 1 and 2 is aggregated and shown to all
  agents in level 3.}
\label{fig:3level}
\end{figure}
\swcomment{check the figure caption above. can we index the indicate
  there are $S$ groups in the figure?}


\begin{lemma}[``Clean'' events]

\end{lemma}





\begin{proof}
Wlog we assume $\mu_1 \geq \mu_2$ as the recommendation policy is symmetric to both arms. We do a case analysis based on $\mu_1-\mu_2$. 

Before we start with the case analysis, we first define several clean events and show that the intersection of them happens with high probability. 
\begin{itemize}
\item \textbf{Concentration of the number of arm $a$ pulls in the first level:} 
By Lemma \ref{lem:greedy}, we know $\GdP \leq q_a \leq \GdT$. For the $s$-th first-level group, define $W_1^{a,s}$ to be the event that the number of arm $a$ pulls in the $s$-th first-level group is between $q_a T_1- \GdT \sqrt{T_1\log(T)}$ and $q_a T_1 + \GdT \sqrt{T_1\log(T)}$. By Chernoff bound,
\[
\Pr[W_1^{a,s}] \geq 1-2\exp(-2\log(T)) \geq 1-2/T^2.
\]
Define $W_1$ to be the intersection of all these events (i.e. $W_1 = \bigcap_{a,s}W_1^{a,s}$). By union bound, we have
\[
\Pr[W_1] \geq 1- \frac{4S}{T^2}.
\]
\item \textbf{Concentration of the empirical mean of arm $a$ pulls in the first level:}
For each first-level group and arm $a$, imagine there is a tape of enough arm $a$ pulls sampled before the recommendation policy starts and these samples are revealed one by one whenever agents in this group pull arm $a$. For the $s$-th first-level group and arm $a$, define $W_2^{s,a,t_1,t_2}$ to be the event that the mean of $t_1$-th to $t_2$-th pulls in the tape is at most $\sqrt{\frac{2\log(T)}{t_2-t_1+1}}$ away from $\mu_a$. By Chernoff bound,\swcomment{a bit confused about what $t_1$ and $t_2$ mean?}
\[
\Pr[W_2^{s,a,t_1,t_2}] \geq 1 - 2\exp(-4\log(T)) \geq 1- 2/T^4.
\]

Define $W_2$ to be the intersection of all these events (i.e. $W_2 = \bigcap_{a,s,t_1,t_2} W_2^{s,a,t_1,t_2}$). By union bound, we have
\[
\Pr[W_2] \geq 1- \frac{4S}{T^2}.
\]

\item \textbf{Concentration of the empirical mean of arm $a$ pulls in the first two levels:}

For all the groups in the first two levels and arm $a$, imagine there is a tape of enough arm $a$ pulls sampled before the recommendation policy starts and these samples are revealed one by one whenever agents in the first two levels pull arm $a$. Define $W_3^{a,t}$ to be the event that the mean of the first $t$ pulls in the tape is at most $\sqrt{\frac{2\log(T)}{t}}$ away from $\mu_a$. By Chernoff bound, 
\[
\Pr[W_3^{a,t}] \geq 1 - 2\exp(-4\log(T)) \geq 1- 2/T^4.
\]
Define $W_3$ to be the intersection of all these events (i.e. $W_3 = \bigcap_{a,t} W_3^{a,t}$). By union bound, we have
\[
\Pr[W_3] \geq 1- \frac{4}{T^3}.
\]

\item \textbf{Anti-concentration of the empirical mean of arm $a$ pulls in the first level:}

Consider the tapes defined in the second bullet again. For the $s$-th first-level group and arm $a$, define $W_4^{s,a,high}$  to be the event that first $q_a T_1$ pulls of arm $a$ in the corresponding tape has empirical mean at least $\mu_a + 1/\sqrt{q_a T_1}$ and define  $W_4^{s,a,low}$  to be the event that first $q_a T_1$ pulls of arm $a$ in the corresponding tape has empirical mean at most $\mu_a - 1/\sqrt{q_a T_1}$. By Berry-Essen Theorem and $\mu_a \in [1/3,2/3]$, we have
\[
\Pr[W_4^{s,a,high}] \geq (1-\Phi(1/2)) - \frac{5}{\sqrt{q_aT_1}} > 1/4.
\]
The last inequality follows when $T$ is larger than some constant.
Similarly we also have 
\[
\Pr[W_4^{s,a,low}] > 1/4.
\]
Since $W_4^{s,a,high}$ is independent with $W_4^{s,3-a,low}$, we have
\[
\Pr[W_4^{s,a,high} \cap W_4^{s,3-a,low}] =\Pr[W_4^{s,a,high}] \cdot  \Pr[W_4^{s,3-a,low}]>(1/4)^2 = 1/16.
\]
Now define $W_4$ as $\bigcap_a \bigcup_s (W_4^{s,a,high} \cap W_4^{s,3-a,low})$. Notice that $(W_4^{s,a,high} \cap W_4^{s,3-a,low})$ are independent across different $s$'s. By union bound, we have
\[
\Pr[W_4] \geq 1- 2(1-1/16)^S \geq 1 -2 /T.
\]
\end{itemize}

By union bound, the intersection of these clean events (i.e. $\bigcap_{i=1}^4 W_i$) happens with probability $1-O(1/T)$. When this intersection does not happen, since the probability is $O(1/T)$, it contributes $O(1/T) \cdot T = O(1)$ to the expected regret. 

Now we assume the intersection of clean events happens and we summarize what these clean events imply.

\begin{itemize}
\item For the $s$-th first-level group and arm $a$, define $\bar{\mu}_a^{1,s}$ to be the empirical mean of arm $a$ pulls in this group. $W_1^{a,s}$, $W_2^{a,s,1,t}$ for $ = q_a T_1- \GdT \sqrt{T_1\log(T)},...,q_a T_1- \GdT \sqrt{T_1\log(T)}$ together imply that
\[
|\bar{\mu}_a^{1,s} - \mu_a| \leq \sqrt{\frac{2\log(T)}{q_a T_1- \GdT \sqrt{T_1\log(T)}}} \leq \sqrt{\frac{4\log(T)}{q_a T_1}}.
\]
The last inequality holds when $T$ is larger than some constant.
\item For each arm $a$, define $\bar{\mu}_a$ to be the empirical mean of arm $a$ pulls in the first two levels. $W_1^{a,s}$ for $s=1,...,S$ and $W_3^{a,t}$ for $t \geq  (q_a T_1- \GdT \sqrt{T_1\log(T)})S$ together imply that
\[
|\bar{\mu}_a - \mu_a| \leq \sqrt{\frac{2\log(T)}{S\left(q_a T_1- \GdT \sqrt{T_1\log(T)}\right)}} \leq \sqrt{\frac{4\log(T)}{S q_a T_1}} .
\]
The last inequality holds when $T$ is larger than some constant.

If there are at least $T_2$ pulls of arm $a$ in the first two levels, 
\[
|\bar{\mu}_a-\mu_a| \leq \sqrt{\frac{2\log(T)}{T_2}}. 
\]

\item For each $a \in \{1,2\}$, $W_4$ implies that there exists $s_a$ such that $W_4^{s_a,a,high}$ and $W_4^{s_a,3-a,low}$ happen. $W_4^{s_a,a,high}$,  $W_1^{s_a,a}$, $W_2^{s_a,a,t, q_aT_1}$ for $t = q_a T_1- \GdT \sqrt{T_1\log(T)}+1, ...,q_aT_1-1$ and $W_2^{s_a,a,q_aT_1,t}$ for $t= q_aT_1,...,q_a T_1+ \GdT \sqrt{T_1\log(T)}$ together imply that 
\begin{align*}
\bar{\mu}_a ^{1,s_a} &\geq \mu_a + \left(q_aT_1 \cdot \frac{1}{\sqrt{q_aT_1}} - \GdT \sqrt{T_1\log(T)} \cdot \sqrt{\frac{2\log(T)}{ \GdT \sqrt{T_1\log(T)}}} \right) \cdot \frac{1}{q_a T_1+ \GdT \sqrt{T_1\log(T)}} \\
&> \mu_a + \frac{1}{4\sqrt{q_aT_1}}.
\end{align*}
The second last inequality holds when $T$ is larger than some constant.
Similarly, we also have
\[
\bar{\mu}_{3-a} ^{1,s_a} < \mu_{3-a}   - \frac{1}{4\sqrt{q_{3-a} T_1}}.
\]
\end{itemize}

Finally we proceed to the case analysis. We give upper bounds on the expected regret conditioned on the intersection of clean events.

\begin{itemize}
\item $\mu_1 - \mu_2 \geq 2\left(\sqrt{\frac{4\log(T)}{q_1T_1}} 
+ \sqrt{\frac{4\log(T)}{q_2T_1}}\right)$. In this case, we want to show that agents in the second and the third levels all pull arm 1. 

First consider the $s$-th second-level group. We know that 
\[
\bar{\mu}_1^{1,s} - \bar{\mu}_2^{1,s} \geq \mu_1 -\mu_2 - \sqrt{\frac{4\log(T)}{q_1T_1}} - \sqrt{\frac{4\log(T)}{q_2T_1}} \geq  \sqrt{\frac{4\log(T)}{q_1T_1}} + \sqrt{\frac{4\log(T)}{q_2T_1}}.
\]
For any agent $t$ in the $s$-th second-level group, by Assumption \ref{ass:embehave}, we have
\begin{align*}
\hat{\mu}_1^t - \hat{\mu}_2^t &>\bar{\mu}_1^{1,s} - \bar{\mu}_2^{1,s} - \frac{c_m}{\sqrt{q_1T_1/2}} - \frac{c_m}{\sqrt{q_2T_1/2}}\\
&\geq  \sqrt{\frac{4\log(T)}{q_1T_1}} + \sqrt{\frac{4\log(T)}{q_2T_1}}- \frac{c_m}{\sqrt{q_1T_1/2}} - \frac{c_m}{\sqrt{q_2T_1/2}}\\
 &> 0.
\end{align*}
Therefore, we know agents in the $s$-th second-level group will all pull arm 1.

Now consider the agents in the third level group. Recall $\bar{\mu}_a$ is the empirical mean of arm $a$ in the history they see. We have
\[
\bar{\mu}_1 - \bar{\mu}_2 \geq \mu_1 -\mu_2 - \sqrt{\frac{4\log(T)}{Sq_1T_1}} - \sqrt{\frac{4\log(T)}{Sq_2T_1}} \geq  \sqrt{\frac{4\log(T)}{q_1T_1}} 
+ \sqrt{\frac{4\log(T)}{q_2T_1}}.
\]
Similarly as above, by Assumption \ref{ass:embehave}, we know $\hat{\mu}_1^t - \hat{\mu}_2^t > 0$ for any agent $t$ in the third level. So we know agents in the third-level group will all pull arm 1. Therefore the expected regret is at most $S T_G T_1 = O(T^{4/7} \log^{6/7}(T))$. 


\item $2\left(\sqrt{\frac{4\log(T)}{Sq_1T_1}} 
+ \sqrt{\frac{4\log(T)}{Sq_2T_1}}\right) \leq \mu_1-\mu_2 < 2\left(\sqrt{\frac{4\log(T)}{q_1T_1}} 
+ \sqrt{\frac{4\log(T)}{q_2T_1}}\right)$. In this case, we want to show agents in the third level all pull arm 1. Recall $\bar{\mu}_a$ is the empirical mean of arm $a$ in the first two levels. We have
\[
\bar{\mu}_1 - \bar{\mu}_2 \geq \mu_1 -\mu_2 - \sqrt{\frac{4\log(T)}{Sq_1T_1}} - \sqrt{\frac{4\log(T)}{Sq_2T_1}} \geq  \sqrt{\frac{4\log(T)}{Sq_1T_1}} 
+ \sqrt{\frac{4\log(T)}{Sq_2T_1}}.
\]
For any agent $t$ in the third level, by Assumption \ref{ass:embehave}, we have
\begin{align*}
\hat{\mu}_1^t - \hat{\mu}_2^t &>\bar{\mu}_1 - \bar{\mu}_2 - \frac{c_m}{\sqrt{Sq_1T_1/2}} - \frac{c_m}{\sqrt{Sq_2T_1/2}}\\
&\geq  \sqrt{\frac{4\log(T)}{Sq_1T_1}} + \sqrt{\frac{4\log(T)}{Sq_2T_1}}- \frac{c_m}{\sqrt{Sq_1T_1/2}} - \frac{c_m}{\sqrt{Sq_2T_1/2}}\\
 &> 0.
\end{align*}
So we know agents in the third-level group will all pull arm 1. Therefore the expected regret is at most 
\[
(S T_G T_1 + S T_2) \cdot 2\left(\sqrt{\frac{4\log(T)}{q_1T_1}} 
+ \sqrt{\frac{4\log(T)}{q_2T_1}}\right) = O(T^{4/7} \log^{6/7}(T))
\]

\item $ 3\sqrt{\frac{2\log(T)}{T_2}} < \mu_1-\mu_2 < 2\left(\sqrt{\frac{4\log(T)}{Sq_1T_1}} 
+ \sqrt{\frac{4\log(T)}{Sq_2T_1}}\right)$. In this case, we just need to make sure that agents in the third level all pull arm 1. To do so, we need both arms to be pulled at least $T_2$ rounds in the second level.  

Now consider the $s_a$-th second-level group. We have
\begin{align*}
\bar{\mu}_a^{1,s_a} - \bar{\mu}_{3-a}^{1,s_a} &> \mu_a + \frac{1}{4\sqrt{q_aT_1}} -\mu_{3-a} +\frac{1}{4\sqrt{q_{3-a}T_1}} \\
&> \frac{1}{4\sqrt{q_1T_1}}+ \frac{1}{4\sqrt{q_2T_1}} - 2\left(\sqrt{\frac{4\log(T)}{Sq_1T_1}} 
+ \sqrt{\frac{4\log(T)}{Sq_2T_1}}\right) \\
&\geq \frac{1}{8\sqrt{q_1T_1}}+ \frac{1}{8\sqrt{q_2T_1}}.
\end{align*}
For any agent $t$ in the $s_a$-th second-level group, by Assumption \ref{ass:embehave}, we have
\begin{align*}
\hat{\mu}_a^t - \hat{\mu}_{3-a}^t &>\bar{\mu}_a^{1,s_a} - \bar{\mu}_{3-a}^{1,s_a} - \frac{c_m}{\sqrt{q_1T_1/2}} - \frac{c_m}{\sqrt{q_2T_1/2}}\\
&\geq   \frac{1}{8\sqrt{q_1T_1}}+ \frac{1}{8\sqrt{q_2T_1}}- \frac{c_m}{\sqrt{q_1T_1/2}} - \frac{c_m}{\sqrt{q_2T_1/2}}\\
 &> 0.
\end{align*}
So we know agents in the $s_a$-th second-level group will all pull arm $a$. Therefore in the first two levels, both arms are pulled at least $T_2$ times. Now consider the third-level. We have
\[
\bar{\mu}_1 - \bar{\mu}_2  \geq \mu_1 -\mu_2 - 2\sqrt{\frac{2\log(T)}{T_2}} \geq \sqrt{\frac{2\log(T)}{T_2}}.
\]
Similarly as above, by Assumption \ref{ass:embehave}, we know $\hat{\mu}_1^t - \hat{\mu}_2^t > 0$ for any agent $t$ in the third level. So we know agents in the third-level group will all pull arm 1.

Therefore the expected regret is at most 
\[
(S T_G T_1 + S T_2) \cdot 2\left(\sqrt{\frac{4\log(T)}{Sq_1T_1}} 
+ \sqrt{\frac{4\log(T)}{Sq_2T_1}}\right) \leq O(T^{4/7} \log^{6/7}(T))
\]


\item $\mu_1 - \mu_2 \leq 3\sqrt{\frac{2\log(T)}{T_2}}$. This is the easy case. Even always pulling the sub-optimal arm (i.e. arm 2) gives regret at most $T \cdot (\mu_1-\mu_2) = O(T^{4/7} \log^{6/7}(T))$. 
\end{itemize}

\end{proof}

%%% Local Variables:
%%% mode: latex
%%% TeX-master: "main"
%%% End:


%l-level in the 10page
%!TEX root = main.tex

\section{$\tilde O(\sqrt{T})$ regret with $L$-level policy}
\label{sec:llevel}
In this section, we give an overview of how we extend our three-level
policy to a more adaptive $L$-level policy for $L > 3$ in order to
achieve a regret rate of $O_K(\sqrt{T} \polylog(T))$. We provide two
such policies. The first policy achieves the root-$T$ regret rate with
$O(\log \log T)$ levels.

% In Theorem \ref{thm:llevel-1}, we show our $L$-level policy. We can
% achieve nearly optimal regret $O_K(T^{1/2} \polylog(T))$ with
% $O(\log\log(T))$ levels.
\begin{theorem}
\label{thm:llevel-1}
For any $L > 3$, there exists an $L$-level disclosure policy with
regret $$O_K\left(T^{2^{L-1}/(2^L-1)} \cdot \polylog(T) \right).$$ In
particular, there exists a $O(\log\log(T))$-level recommendation
policy with regret $O_K(T^{1/2} \polylog(T))$.
\end{theorem}

% In Theorem \ref{thm:llevel-2}, we show a policy with
Our second policy achieves an instance-dependent regret
guarantee. This policy has the same info-graph structure as the first
one in Theorem \ref{thm:llevel-1}, but requires a higher number of
levels $L = O(\log(T/\log\log(T)))$ and larger group sizes. We will
bound its regret as a function of the gap parameter $\Delta$ even
though the construction of the policy does not depend on $\Delta$. In
particular, this regret bound outperforms the one in Theorem
\ref{thm:llevel-1} when $\Delta$ is much bigger than $T^{-1/2}$.  It
also has the desirable property that the policy does not withhold too
much information from agents---any agent $t$ observes a good fraction
of history in previous rounds.

\begin{theorem}
\label{thm:llevel-2}
There exists a $O(\log(T)/\log\log(T))$-level policy such that for
every multi-armed bandit instance with gap parameter $\Delta$, the
policy has regret $$O_K(\min(1/\Delta, T^{1/2}) \cdot \polylog(T))$$
Moreover, under this policy, each agent $t$ observes a subhistory of
size at least $\Omega( t/\polylog(T))$.
\end{theorem}

Note for constant number of arms, this result mathches the optimal
regret rate (given in \Cref{eq:model-OptRegret}) for stochastic
bandits, up to logarithmic factors.


In this section, we present the main techniques in our solution, and
the full proofs of Theorem \ref{thm:llevel-1} and Theorem
\ref{thm:llevel-2} will be deferred to Appendix
\ref{sec:llevel-details}. Similarly as Section \ref{sec:3level}, we
first prove them in the case of 2 arms (Theorem \ref{thm:llevel} and
Corollary \ref{cor:llevel}). We then extend them to the case of
constant number of arms (Theorem \ref{thm:constarm}).

A natural idea to extend the three-level policy is to insert more
levels as multiple ``check points", so the policy can incentivize the
agents to perform more adaptive exploration. However, we need to
introduce two main modifications in the info-graph two accomendate
some new challenges. We will first informally describe our techniques
for the two-arm case.

% \begin{description}
% \item[Challenge 1] \emph{Exponentially growing number of groups.} 
% \item[Challenge 2] \emph{Uncertain number of arm pulls.}  
% \end{description}


\OMIT{We will now present the main idea of extending from 3-level to
  $L$-level is that instead of using the second level as one
  ``check-point'', we use more levels to have multiple ``check
  points''.  Some new challenges appear in this process. Here we give
  an overview of the additional techniques we use to get our $L$-level
  results.}

\paragraph{Interlacing connections between levels.}A tempted approach
to generalize the three-level policy is to build an $L$-level
info-graph with the structure of a $\NG$-ary tree: for every
$l\in \{2, \ldots , L\}$, each $l$-level group observes the
sub-history from a disjoint set $\NG$ groups in level $(l-1)$. The
disjoint sub-histories observed by all the groups in level $l$ are
independent, and under the small gap regime (similar to
Lemma~\ref{3levelsmallcase}) it ensures that each arm $a$ has a
``lucky'' $l$-level group of agents that only pull $a$. This ``lucky''
property is crucial for ensuring that both arms will be explored in
level $l$.

However, in this construction, the first level will have $\NG^{L-1}$
groups, which incurs $\NG^{\Omega(L)}$ regret. The exponential
dependence in $L$ will heavily limit the adaptivity of the policy, and
prevents having the number of levels for obtaining the result in
\Cref{thm:llevel-2}. To overcome this, we will design an info-graph
structure such that the number of groups at each level stays as
$\NG^2 = \Theta(\log^2(T))$.

We will leverage the following key observation: in order to maintain
the ``lucky'' property, it suffices to have $\NG$ $l$-th level groups
that observe disjoint sub-histories that take place in level
$(l - 1)$. Moreover, as long as the group size in levels lower than
$(l-1)$ are substantially smaller than group size of level $l-1$, the
``lucky'' property does not break even if different groups in level
$l$ observe overlapping sub-history from levels $\{1, \ldots, l-2\}$.


\OMIT{In our 3-level policy, the second level has
  $\NG = \Theta(\log(T))$ groups. We do so to ensure that in the small
  gap case (Lemma \ref{3levelsmallcase}), with high probability, both
  arms are pulled enough times in the second level. The argument
  relies on the fact that agents in different second-level groups
  observe disjoint histories of the first level and the independent
  randomness in first level groups guarantees each arm $a$ to have a
  ``lucky'' second-level group such that agents in that group all pull
  arm $a$ with high probability.

  Simply generalizing this idea to an $L$-level policy would give us a
  $\NG$-ary tree like structure and the first level will have
  $\NG^{L-1}$ groups. It incurs an extra $\NG^{\Omega(L)}$ factor in
  the regret. If we use $\log\log(T)$ levels as in Theorem
  \ref{thm:llevel-1}, this factor super poly-logarithmic. If we use
  $\log(T)/\log\log(T)$ levels as in Theorem \ref{thm:llevel-2}, this
  factor goes up to polynomial in $T$.


  In order to avoid this undesirable factor, we design a slightly
  different connecting structure between levels in our $L$-level
  policy.  The key observation is that, for any level
  $l \in \{3,\ldots,L-1\}$, we do want agents in different $l$-th level
  groups to see disjoint histories of the $(l-1)$-th level to ensure
  that there is some ``lucky'' group for each arm whose mean is close
  enough to the best arm. However, it does not matter much if pulls in
  levels below $l-1$ get observed by agents in different $l$-th level
  groups because group sizes in lower levels are much smaller than
  group sizes of level $l-1$ and the independent randomness in level
  $l-1$ is sufficient.}

This motivates the following interlacing connection structure between
levels. For each level in the info-graph, there are $\NG^2$ groups for
some $\NG = \Theta(\log(T))$. The groups in the $l$-th level are
labeled as $G_{l,u,v}$ for $u,v\in[\NG]$. For any $l \in \{2,\ldots,L\}$
and $u,v,w\in [\NG]$, agents in group $G_{l,u,v}$ see the history of
agents in group $G_{l-1,v,w}$ (and by transitivity all agents in
levels below $l-1$). See Figure \ref{fig:llevel-connecting} for a
visualization of simple case with $\NG = 2$). Two observations are
in order:
% \OMIT{(i) The number of groups does not grow with $L$ and stays as
%   $\NG^2 = \Theta(\log^2(T))$ in each level. Having these groups only
%   incurs a $\Theta(\log^2(T))$ multiplicative factor to the regret.\\}
\begin{itemize}
\item[(i)] Consider level $(l - 1)$ and fix the last group index to be
  $v$, and consider the set of groups
  $\cG_{l-1, v}=\{G_{l-1,i,v}\mid i \in [\NG]\}$ (e.g. $G_{l-1,1,1}$
  and $G_{l-1,2,1}$ circled in the Figure
  \ref{fig:llevel-connecting}). The agents in any group of
  $\cG_{l-1, v}$ observe the same sub-history. If the empirical mean
  of arm $a$ is sufficiently high in their shared sub-history, then all groups in $\cG_{l-1, v}$ will become ``lucky'' groups for $a$.

\item[(ii)] Every agent in level $l$ observes the sub-history from
  $\NG$ $(l-1)$-th level groups, each of which belonging to a
  different set $\cG_{l-1, v}$. Thus, for each arm $a$, we just need
  one set of groups $\cG_{l-1, v}$ in level $l-1$ to be ``lucky'' for
  $a$ and then all agents in level $l$ will see sufficient arm $a$
  pulls.
 \OMIT{For each set of groups
    in level $l-1$, agents in level $l$ see the history of agents in
    one of these groups. Therefore, for each arm $a$, we just need one
    set of groups in level $l-1$ to get ``lucky'' and then all agents
    in level $l$ will see enough pulls of arm $a$.}
\end{itemize}

\begin{figure}[H]
\centering
\begin{tikzpicture}  
 \foreach \x in {0,3,6,9}
 {
 \filldraw[fill=purple!20!white]
 (\x+0,7.5)--(\x+2,7.5)--(\x+2,8.5)--(\x+0,8.5)--cycle;
 \filldraw[fill=green!20!white]
 (\x+0,5)--(\x+2,5)--(\x+2,6)--(\x+0,6)--cycle;
 \filldraw[fill=blue!20!white]
 (\x+0,2.5)--(\x+2,2.5)--(\x+2,3.5)--(\x+0,3.5)--cycle;
 }
\foreach \y in {2.5,5,7.5}
{
  \node at (12.5,\y+0.5){$\cdots$};
} 


\foreach \y in {3.5,6}
{
  \draw (1,\y) -- (1,\y+1.5);
  \draw (1,\y) -- (7,\y+1.5);
  
  \draw (4,\y) -- (1,\y+1.5);
  \draw (4,\y) -- (7,\y+1.5);
  
  
  \draw (7,\y) -- (4,\y+1.5);
  \draw (7,\y) -- (10,\y+1.5);
  
  
  \draw (10,\y) -- (4,\y+1.5);
  \draw (10,\y) -- (10,\y+1.5);
} 

\foreach \u in {1,2}
{
	\foreach \v in {1,2}
	{
	\pgfmathsetmacro{\x}{((\u-1)*2+(\v-1))*3};
	\pgfmathsetmacro{\xa}{((\u-1))*3};
	\pgfmathsetmacro{\xb}{(2+(\u-1))*3};
   \node at(\x+1,8){$G_{l,\u,\v}$};
   \node at(\x+1,5.5){$G_{l-1,\u,\v}$};
   \node at(\x+1,3){$G_{l-2,\u,\v}$};
	}
}


  \node at(-1.2,3){\textbf{Level $l-2$}};
  \node at(-1.2,5.5){\textbf{Level $l-1$}};
  \node at(-1.2,8){\textbf{Level $l$}};
  \draw[->] (13.3,2)--(13.3,8.5);
  \node at(13.7,5)[ rotate=90]{Time};

 \draw [rounded corners=3mm, line width=1mm, red](-0.2,4.8)--(2.2,4.8)--(2.2,6.2)--(-0.2,6.2)--cycle;
  \draw [rounded corners=3mm, line width=1mm, red](5.8,4.8)--(8.2,4.8)--(8.2,6.2)--(5.8,6.2)--cycle;
\end{tikzpicture}
\caption{Connectiing structures between levels for the $L$-level policy.}
\label{fig:llevel-connecting}
\end{figure}

\paragraph{Amplifying groups for boundary cases.} Recall in our
3-level policy, the medium gap case (Lemma \ref{3levelmedium}) is the
boundary case when the gap $|\mu_1-\mu_2|$ is
$\Omega\left(\sqrt{\frac{1}{T_1}}\right)$ and
$O\left(\sqrt{\frac{\log(T)}{T_1}}\right)$. In this case, the gap is
not large enough to conclude that with high probability agents in both
the second level and the third level all pull the best arm. The gap is
also not small enough to conclude that with high probability both arms
are explored enough times in the second level. We need to worry about
the situation in which the sub-optimal arm does not get pulled enough
times in the first two levels and some agents in the third level may
pull the sub-optimal arm. Such situation is naturally ruled out in our
3-level policy for the following reason. Agents in the third level
observe the history of the entire first level while agents in the
second level only observes the history of a single first-level
group. For each arm, even if it is not explored enough times in the
second level, its empirical mean concentrates closer to its actual
mean in the history observed by a third-level agent than the one
observed by a second-level agent. Therefore, although the gap in the
medium gap case is just not large enough for agents in the second
level to all pull the best arm, it is large enough for agents in the
third level to all pull the best arm.

When we extend the 3-level policy to an $L$-level policy, we have such boundary case for each intermediate level. Moreover, the worry mentioned above does not get naturally resolved as in some of these boundary cases, the ratios between the upper and lower bounds of the gap rise from $\Theta(\sqrt{\log(T)})$ to $\Theta(\log(T))$. The reason for such rise is that, except the first level, we don't have good enough guarantee of the number of pulls of each arm. For example, in Figure \ref{fig:llevel-connecting}, when we talk about having enough arm $a$ pulls in the history observed by agents in $G_{l,1,1}$, it could be that  only agents in group $G_{l-1,1,1}$ are pulling arm $a$ and it also could be that most agents in groups $G_{l-1,1,1},G_{l-1,1,2},...,G_{l-1,1,\NG}$ are pulling arm $a$. Therefore our estimate of the number of arm $a$ pulls can be off by an $\NG=\Theta(\log(T))$ multiplicative factor. This finally makes the boundary cases harder to deal with.

In our $L$-level policy, after understanding the boundary cases well, we resolve this problem by introducing an additional type of groups: $\Gamma$-groups. For each $l \in [L], u,v \in [\NG]$, we create a $\Gamma$-group $\Gamma_{l,u,v}$. Agents in $\Gamma_{l,u,v}$ observe the same history as the one observed by agents in $G_{l,u,v}$ and the number of agents in $\Gamma_{l,u,v}$ is $\Theta(\log(T)$ times the number of agents in $G_{l,u,v}$. The main difference between the $G$-groups and $\Gamma$-groups is that the history of $\Gamma$ groups in level $l$ is not sent to agents in level $l+1$ but agents in higher levels. When we are in the boundary case in which we don't have good guarantees about the $l+1$ level agents' pulls, the new construction makes sure that agents in levels higher than $l+1$ get enough pulls of each arm and all pull the best arm. 
%%% Local Variables:
%%% mode: latex
%%% TeX-master: "main"
%%% End:


%\bibliographystyle{plain}
%

%\acks{We would like to thank Robert Kleinberg for discussions in the early stage of this project.}
\bibliography{bib-abbrv,bib-AGT,bib-bandits,bib-ML,bib-random,bib-slivkins}
\appendix
%!TEX root = main.tex
\section{Preliminaries}
\label{sec:prelim}

We use the following  concentration and anti-concentration inequalities. \niedit{The concentration bound is the standard Chernoff bound which shows that sums of independent random variables converge to their expectation quickly.  The anti-concentration bound is the Berry-Esseen theorem which shows that the CDF of an appropriately scaled average of i.i.d.\ random variables converges to the CDF of the standard normal distribution pointwise.}

\begin{theorem}[Chernoff Bound]
Let $X_1,...,X_n$  be independent random variables such that $X_i \in [0,1]$ for all $i$. Let $\bar{X} = \frac{1}{n}\sum_{i=1}^n X_i$ denote their empirical mean. Then
\[
\Pr[ |\bar{X} - \E[\bar{X}]| > \varepsilon] \leq 2\exp(-2n\varepsilon^2).
\]
\end{theorem}

\begin{theorem}[Berry-Esseen Theorem]
Let $X_1,...,X_n$ be i.i.d. variables with $\E[ (X_1 - \E[X_1])^2] = \sigma^2 >0$ and $\E[ |X_1 - \E[X_1]|^3] =\rho <\infty$. Let $\bar{X} = \frac{1}{n} \sum_{i=1}^n X_i$. Let $F_n$ be the cumulative distribution function of $\frac{(\bar{X} - \E[\bar{X}]) \sqrt{n}}{\sigma}$ and $\Phi$ be the cumulative distribution function of the standard normal distribution. For all $x$ and $n$,
\[
|F_n(x) - \Phi(x) | \leq \frac{\rho}{2\sigma^3\sqrt{n}}.
\]
\end{theorem}


\section{Proofs from Section~\ref{sec:warmup}}
\label{app:warmup}

\begin{proof}[Proof of Lemma~\ref{lem:greedy}]
  Fix any arm $a$. Let $\GdT = (K-1) \cdot c_T + 1$ and
  $\GdP = (1/3)^{\GdT}$. \swedit{We will condition on the event that
    all the realized rewards in $\GdT$ rounds are 0, which occurs with
    probability at least $\GdP$ under Assumption~\ref{ass:embehave}.}
  In this case, we want to show that arm $a$ is pulled at least
  once. We prove this by contradiction. Suppose arm $a$ is not pulled. By
  the pigeonhole principle, we know that there is some other arm $a'$
  that is pulled at least $c_T + 1$ rounds. Let $t$ be the round in
  which arm $a'$ is pulled exactly $c_T + 1$ times. By Assumption
  \ref{ass:embehave}, we know
  \[
    \hat{\mu}_{a'}^t \leq 0 + c_m / \sqrt{c_T} < 1/3.
  \]
  \nicomment{Wait, where did we set $c_T$?} On the other hand, we have
  $\hat{\mu}_a^t \geq 1/3 > \hat{\mu}_{a'}^t$. This contradicts with
  the fact that in round $t$, arm $a'$ is pulled, instead of arm $a$.
  \swdelete{ In addition, we know that this case happens with
    probability at least $(1/3)^{\GdT} = \GdP$ as each arm's mean is
    at most $2/3$. To sum up, we know that with probability at least
    $\GdP$, \ALGG of $\GdT$ rounds pulls arm $a$ at least once.}
   \jmcomment{Agree with these changes.}
\end{proof}


\begin{proof}[Proof of Theorem~\ref{thm:2level}]
  We will set $T_1$ later in the proof, depending on whether the gap
  parameter $\Delta$ is known. For now, we just need to know we will
  make $T_1 \geq \frac{4(\GdT)^2}{(\GdP)^2}\log(T)$. Since this policy is
  agnostic to the indices of the arms, we assume w.l.o.g. that arm 1
  has the highest mean.

  The first $T_1 \cdot \GdT$ rounds will get total regret at most
  $T_1 \cdot \GdT$.  We focus on bounding the regret from the second
  level of $T - T_1 \cdot \GdT$ rounds. We consider the following two
   events. We will first bound the probability that both of them
  happen and then we will show that they together imply upper bounds
  on $|\hat{\mu}^t_a - \mu_a|$'s for any agent $t$ in the second
  level. Recall $\hat{\mu}^t_a$ is the estimated mean of arm $a$ by
  agent $t$ and agent $t$ picks the arm with the highest
  $\hat{\mu}^t_a$.

% \begin{itemize}
  \OMIT{\paragraph{Concentration of the number of arm $a$ pulls in the first
    level.}
By Lemma \ref{lem:greedy}, we know $\GdP \leq q_a \leq \GdT$.}
  Define $W_1^a$ to be the event that the number of arm $a$ pulls in
  the first level is at least $q_a T_1- \GdT \sqrt{T_1\log(T)}$.
  \swedit{As long as we set $T_1 \geq \frac{4(\GdT)^2}{(\GdP)^2}\log(T)$,
    this implies that the number of arm $a$ pulls is then at least
    $q_a T_1/2$.}
\OMIT{  By Chernoff bound,
  \[
    \Pr[W_1^a] \geq 1-\exp(-2\log(T)) \geq 1-1/T^2.
  \]
}
Define $W_1$ to be the intersection of all these events (i.e. $W_1 = \bigcap_{a}W_1^a$). By Lemma~\ref{lem:t1runs}, we have
\[
\Pr[W_1] \geq 1- \frac{K}{T^2} \geq 1 - \frac{1}{T}.
\]
\OMIT{\paragraph{Concentration of the empirical mean of arm $a$ pulls
    in the first level.}}  Next, we show that the empirical mean of
each arm $a$ is close to the true mean. To facilitate our reasoning,
let us imagine there is a tape of length $T$ for each arm $a$, with
each cell containing an independent draw of the realized reward from
the distribution $\cD_a$. Then for each arm $a$ and any $N\in [T]$, we
can think of the sequence of the first $N$ realized rewards of $a$
coming from the prefix of $N$ cells in its reward tape. Define
$W^{a,t}_2$ to be the event that the empirical mean of the first $t$
\swedit{realized rewards in the tape} of arm $a$ is at most
$\sqrt{\frac{2\log(T)}{t}}$ away from $\mu_a$. Define $W_2$ to be the
intersection of these events (i.e.  $\bigcap_{a,t} W^{a,t}_2$).  By
Chernoff bound,\swcomment{$t$ may be confusing here}
\[
\Pr[W^{a,t}_2] \geq 1 - 2\exp(-4\log(T)) \geq 1-2/T^4.
\]
By union bound,
\[
\Pr[W_2] \geq 1 - KT \cdot \frac{2}{T^4} \geq 1 - \frac{2}{T}.
\]



By union bound, we know $\Pr[W_1 \cap W_2] \geq 1 - 3/T$. For the
remainder of the analysis, we will condition on the event
$W_1 \cap W_2$.

For any arm $a$ and agent $t$ in the second level, by $W_1$ and $W_2$, we have
\[
|\bar{\mu}^t_a - \mu_a| \leq \sqrt{\frac{2\log(T)}{q_aT_1 /2}}.
\]
By $W_1$ and Assumption \ref{ass:embehave}, we have
\[
|\bar{\mu}^t_a - \hat{\mu}^t_a| \leq \frac{c_m}{\sqrt{q_aT_1/2}}.
\]
Therefore,
\[
|\hat{\mu}^t_a - \mu_a|\leq \sqrt{\frac{2\log(T)}{q_aT_1 /2}}+\frac{c_m}{\sqrt{q_aT_1/2}} \leq 3 \sqrt{\frac{\log(T)}{\GdP T_1 }}.
\]
So the second-level agents will pick an arm $a$ which has $\mu_a$ at most $6 \sqrt{\frac{\log(T)}{\GdP T_1 }}$ away from $\mu_1$. To sum up, the total expected regret is at most
\[
T_1 \cdot \GdT + T \cdot (1-\Pr[W_1 \cap W_2]) + T \cdot  6 \sqrt{\frac{\log(T)}{\GdP T_1 }}.
\]
By setting $T_1 = T^{2/3}\log(T)^{1/3}$, we get expected regret $O(T^{2/3}\log(T)^{1/3})$.
\OMIT{Notice that if we know the gap parameter is known to be larger than
$\Delta$, we can set
\[
T_1 = \max\left( \frac{4\GdT^2}{\GdP^2}\log(T), \frac{36 }{\Delta^2 \cdot p_G} \log(T) \right).
\]
In this case, since $\Delta \geq 6 \sqrt{\frac{\log(T)}{\GdP T_1 }}$, we know agents in the second level will all pull arm 1. Therefore, the total expected regret is at most
\[
T_1 \cdot \GdT + T \cdot (1- \Pr[W_1 \cap W_2]) = O(\Delta^{-2} \log(T)).
\]
This completes the proof.}
\end{proof}

%!TEX root = main.tex

\section{Missing proofs from Section~\ref{sec:3level}}
\label{sec:3level-pfs}

%\ascomment{I copied this from the 3-level section. This needs to be revised and made consistent, etc.}



%\ascomment{end of copy-and-paste}

\subsection{Events}

The following lemmas can be derived from combining Lemma~\ref{lem:t1runs} and union bound.

\begin{lemma}[Concentration of first-level number of pulls.]\label{3levelw1}
  Let $W_1$ be the event that for all groups $s\in [\NG]$ and arms
  $a\in \{1, 2\}$, the number of arm $a$ pulls in the $s$-th
  first-level group is in the range of
  $$
  \left[\fdpN  T_1- \GdT \sqrt{T_1\log(T)}, \fdpN  T_1 + \GdT \sqrt{T_1\log(T)}\right],
  $$
  where $\fdpN $ is the expected number of arm $a$ pulls in a $\ALGG$ run
  of length $\GdT$. Then $\Pr[W_1] \geq 1- \frac{4\NG}{T^2}$.
\end{lemma}

\begin{proof}[Proof of Lemma~\ref{3levelw1}]
  For the $s$-th first-level group, define $W_1^{a,s}$ to be the event
  that the number of arm $a$ pulls in the $s$-th first-level group is
  between $\fdpN T_1- \GdT \sqrt{T_1\log(T)}$ and
  $\fdpN T_1 + \GdT \sqrt{T_1\log(T)}$. By Lemma~\ref{lem:t1runs}
\[
\Pr[W_1^{a,s}] \geq 1-2\exp(-2\log(T)) \geq 1-2/T^2.
\]
Let $W_1$  be the intersection of all these events (i.e.
$W_1 = \bigcap_{a,s}W_1^{a,s}$). By union bound, we have
\[
\Pr[W_1] \geq 1- \frac{4\NG}{T^2}.
\]
\end{proof}

To state the events, it will be useful to think of a
hypothetical reward tape $\cT^1_{s, a}$ of length $T$ for each
group $s$ and arm $a$, with each cell independently sampled from
$\cD_a$.  The tape encodes rewards as follows: the $j$-th time arm $a$
is chosen by the group $s$ in the first level, its reward is taken
from the $j$-th cell in this arm's tape. The following result
characterizes the concentration of the mean rewards among all
consecutive pulls among all such tapes, which follows from Chernoff
bound and union bound.

\begin{lemma}[Concentration of empirical means in the first level]\label{3levelw2}
  For any $t_1, t_2\in [T]$ such that $t_1 < t_2$, $s\in [\NG]$, and
  $a\in \{1,2\}$, let $W_2^{s,a,t_1,t_2}$ be the event that the mean
  among the cells indexed by $t_1, (t_1+1), \ldots, t_2$ in the tape
  $\cT^1_{a, s}$ is at most $\sqrt{\frac{2\log(T)}{t_2-t_1+1}}$ away
  from $\mu_a$.  Let $W_2$ be the intersection of all these events
  (i.e.  $W_2 = \bigcap_{a,s,t_1,t_2} W_2^{s,a,t_1,t_2}$). Then
  \[
    \Pr[W_2] \geq 1- \frac{4\NG}{T^2}.
  \]
\end{lemma}


\begin{proof}[Proof of Lemma~\ref{3levelw2}]
  By Chernoff bound,
\[
\Pr[W_2^{s,a,t_1,t_2}] \geq 1 - 2\exp(-4\log(T)) \geq 1- 2/T^4.
\]
By union bound, we have
\[
\Pr[W_2] \geq 1- \frac{4\NG}{T^2}.
\]
\end{proof}

Our policy also relies on the anti-concentration of the empirical
means in the first round. We show that for each arm $a\in \{1, 2\}$,
there exists a group $s_a$ such that the empirical mean of $a$ is
slightly above $\mu_a$, while the other arm $(3 - a)$ has empirical
mean slightly below $\mu_{(3-a)}$. This event is crucial for inducing
agents in the second level to explore both arms when the their
mean rewards are indistinguishable after the first level.


\begin{lemma}[Co-occurence of high and low deviations in this first level]\label{3levelw4}
  For any group $s\in [\NG]$, any arm $a$, let $\tilde\mu_{a,s}$ be the
  empirical mean of the first $\fdpN  T_1$ cells in tape $\cT^1_{a, s}$.
  Let $W_4^{s,a,\text{high}}$ be the event
  $\tilde\mu_{a, s} \geq \mu_a + 1/\sqrt{\fdpN  T_1}$ and let
  $W_4^{s,a,\text{low}}$ be the event that
  $\tilde\mu_{a, s} \leq \mu_a - 1/\sqrt{\fdpN  T_1}$.  Let $W_4$ be the
  event that for every $a\in \{1, 2\}$, there exists a group
  $s_a\in [\NG]$ in the first level such that both $W_4^{s_a,a,\text{high}}$
  and $W_4^{s_a,3-a,\text{low}}$ occur. Then
  \[
    \Pr[W_4]\geq 1 -2 /T.
  \]
\end{lemma}



\begin{proof}[Proof of Lemma~\ref{3levelw4}]
By Berry-Esseen Theorem and
  $\mu_a \in [1/3,2/3]$, we have for any $a$,
\[
\Pr[W_4^{s,a,high}] \geq (1-\Phi(1/2)) - \frac{5}{\sqrt{\fdpN T_1}} > 1/4.
\]
The last inequality follows when $T$ is larger than some constant.
Similarly we also have
\[
\Pr[W_4^{s,a,low}] > 1/4.
\]
Since $W_4^{s,a,high}$ is independent with $W_4^{s,3-a,low}$, we have
\[
\Pr[W_4^{s,a,high} \cap W_4^{s,3-a,low}] =\Pr[W_4^{s,a,high}] \cdot  \Pr[W_4^{s,3-a,low}]>(1/4)^2 = 1/16.
\]
Notice that $(W_4^{s,a,high} \cap W_4^{s,3-a,low})$ are independent
across different $s$'s. By union bound, we have
\[
\Pr[W_4] \geq 1- 2(1-1/16)^\NG \geq 1 -2 /T.
\]
\end{proof}



Lastly, we will condition on the event that the empirical means of
both arms are concentrated around their true means in any prefix of
their pulls. This guarantees that the policy obtains an accurate
estimate of rewards for both arms after aggregating all the data in
the first two levels.
% We will again state the event using a hypothetical reward tape
% $\cT^2_a$ for each arm $a$: the $j$-th time arm $a$ is chosen in the
% two levels, its reward is taken from the $j$-th cell in $\cT^2_a$.


%\swcomment{might want to swap out $W_3$ and $W_4$} \nicomment{yes please!}
%\jmcomment{let's follow the order of levels}

\begin{lemma}[Concentration of empirical means in the first two
  levels]\label{3levelw3}
  With probability at least $1 - \frac{4}{T^3}$, the following event
  $W_3$ holds: for all $a\in \{1, 2\}$ and $m \in [N_{T, a}]$, the
  empirical means of the first $t$ \nicomment{what is $t$ here; you mean $m$? in which case please replace $m$ with $t$ in the conditioning and bound} arm $a$ pulls is at most
  $\sqrt{\frac{2\log(T)}{m}}$ away from $\mu_a$, where $N_{T, a}$ is
  the total number of arm $a$ pulls by the end of $T$ rounds.
\end{lemma}


\begin{proof}[Proof of Lemma~\ref{3levelw3}]
  For any arm $a$, let's imagine a hypothetical tape of length $T$,
  with each cell independently sampled from $\cD_a$. The tape encodes
  rewards of the first two levels as follows: the $j$-th time arm $a$
  is chosen in the first two levels, its reward is taken from the
  $j$-th cell in the tape. Define $W_3^{a,t}$ to be the event that the
  mean of the first $t$ pulls in the tape is at most
  $\sqrt{\frac{2\log(T)}{t}}$ away from $\mu_a$. By Chernoff bound,
\[
\Pr[W_3^{a,t}] \geq 1 - 2\exp(-4\log(T)) \geq 1- 2/T^4.
\]
By union bound, the intersection of all these events has probability
at least:
\[
\Pr[W_3] \geq 1- \frac{4}{T^3}.
\]
\end{proof}


Let $W = \bigcap_{i=1}^4 W_i$ be the intersection of all 4
events.  By union bound, $W$ occurs with probability $1-O(1/T)$. Note
that the regret conditioned on $W$ not occurring is at most
$O(1/T) \cdot T = O(1)$, so it suffices to bound the regret conditioned on $W$.








\subsection{Case Analysis}
Now we assume the intersection $W$ of events $W_1,\cdots,W_4$ happens. We will
first provide some helper lemmas for our case analysis.

\begin{lemma}
  For the $s$-th first-level group and arm $a$, define
  $\bar{\mu}_a^{1,s}$ to be the empirical mean of arm $a$ pulls in
  this group. If $W$ holds, then
  \[
    |\bar{\mu}_a^{1,s} - \mu_a| \leq \sqrt{\frac{4\log(T)}{\fdpN T_1}}.
  \]
\end{lemma}

\begin{proof}
  The events $W_1$ and $W_2^{a,s,1,t}$ for
  $t = \fdpN T_1- \GdT \sqrt{T_1\log(T)},...,\fdpN T_1 + \GdT
  \sqrt{T_1\log(T)}$ together imply that
\[
|\bar{\mu}_a^{1,s} - \mu_a| \leq \sqrt{\frac{2\log(T)}{\fdpN T_1- \GdT \sqrt{T_1\log(T)}}} \leq \sqrt{\frac{4\log(T)}{\fdpN T_1}}.
\]
The last inequality holds when $T$ is larger than some constant.
\end{proof}


\begin{lemma}
  For each arm $a$, define $\bar{\mu}_a$ to be the empirical mean of
  arm $a$ pulls in the first two levels. If $W$ holds, then
  \[
    |\bar{\mu}_a - \mu_a| \leq \sqrt{\frac{4\log(T)}{\NG \fdpN T_1}} .
  \]
Furthermore, if there are at least $T_2$ pulls of arm $a$ in the first two levels,
\[
|\bar{\mu}_a-\mu_a| \leq \sqrt{\frac{2\log(T)}{T_2}}.
\]
\end{lemma}

\begin{proof}
The events $W_1$ and $W_3^{a,t}$ for $t \geq  (\fdpN T_1- \GdT \sqrt{T_1\log(T)})\NG$ together imply that
  \[
    |\bar{\mu}_a - \mu_a| \leq \sqrt{\frac{2\log(T)}{\NG\left(\fdpN T_1- \GdT \sqrt{T_1\log(T)}\right)}} \leq \sqrt{\frac{4\log(T)}{\NG \fdpN T_1}} .
\]
The last inequality holds when $T$ is larger than some constant.
\end{proof}


\begin{lemma}\label{lem:luck}
  For the $s$-th first-level group and arm $a$, define
  $\bar{\mu}_a^{1,s}$ to be the empirical mean of arm $a$ pulls in
  this group. For each $a \in \{1,2\}$, there exists a group $s_a$
  such that
\[
\bar{\mu}_a ^{1,s_a} > \mu_a + \frac{1}{4\sqrt{\fdpN T_1}} \quad \mbox{and, } \quad
\bar{\mu}_{3-a} ^{1,s_a} < \mu_{3-a}   - \frac{1}{4\sqrt{\fdpN[3-a] T_1}}.
\]
\end{lemma}




\begin{proof}% [Proof of Lemma~\ref{lem:luck}]
  For each $a \in \{1,2\}$, $W_4$ implies that there exists $s_a$ such
  that both $W_4^{s_a,a,high}$ and $W_4^{s_a,3-a,low}$ happen.  The
  events $W_4^{s_a,a,high}$, $W_1$, $W_2^{s_a,a,t, \fdpN T_1}$
  for $t = \fdpN T_1- \GdT \sqrt{T_1\log(T)}+1, ...,\fdpN T_1-1$ and
  $W_2^{s_a,a,\fdpN T_1,t}$ for
  $t= \fdpN T_1,...,\fdpN T_1+ \GdT \sqrt{T_1\log(T)}$ together imply that
\begin{align*}
\bar{\mu}_a ^{1,s_a} &\geq \mu_a + \left(\fdpN T_1 \cdot \frac{1}{\sqrt{\fdpN T_1}} - \GdT \sqrt{T_1\log(T)} \cdot \sqrt{\frac{2\log(T)}{ \GdT \sqrt{T_1\log(T)}}} \right) \cdot \frac{1}{\fdpN T_1+ \GdT \sqrt{T_1\log(T)}} \\
&> \mu_a + \frac{1}{4\sqrt{\fdpN T_1}}.
\end{align*}
The second to the last inequality holds when $T$ is larger than some constant.
Similarly, we also have
\[
\bar{\mu}_{3-a} ^{1,s_a} < \mu_{3-a}   - \frac{1}{4\sqrt{\fdpN[3-a] T_1}}.
\]
This completes the proof.
\end{proof}


Now we proceed to the case analysis.


\begin{proof}[Proof of Lemma~\ref{3levelbigcase} (Large gap case)]
  Observe that for any group $s$ in the first level, the empirical
  means satisfy
\[
\bar{\mu}_1^{1,s} - \bar{\mu}_2^{1,s} \geq \mu_1 -\mu_2 - \sqrt{\frac{4\log(T)}{\fdpN[1]T_1}} - \sqrt{\frac{4\log(T)}{\fdpN[2]T_1}} \geq  \sqrt{\frac{4\log(T)}{\fdpN[1]T_1}} + \sqrt{\frac{4\log(T)}{\fdpN[2]T_1}}.
\]


For any agent $t$ in the $s$-th second-level group, by Assumption \ref{ass:embehave}, we have
\begin{align*}
\hat{\mu}_1^t - \hat{\mu}_2^t &>\bar{\mu}_1^{1,s} - \bar{\mu}_2^{1,s} - \frac{\estC}{\sqrt{\fdpN[1]T_1/2}} - \frac{\estC}{\sqrt{\fdpN[2]T_1/2}}\\
&\geq  \sqrt{\frac{4\log(T)}{\fdpN[1]T_1}} + \sqrt{\frac{4\log(T)}{\fdpN[2]T_1}}- \frac{\estC}{\sqrt{\fdpN[1]T_1/2}} - \frac{\estC}{\sqrt{\fdpN[2]T_1/2}} > 0
\end{align*}
Therefore, we know agents in the $s$-th second-level group will all pull arm 1.

Now consider the agents in the third level group. Recall $\bar{\mu}_a$
is the empirical mean of arm $a$ in the history they see. We have
\[
\bar{\mu}_1 - \bar{\mu}_2 \geq \mu_1 -\mu_2 - \sqrt{\frac{4\log(T)}{\NG\fdpN[1]T_1}} - \sqrt{\frac{4\log(T)}{\NG\fdpN[2]T_1}} \geq  \sqrt{\frac{4\log(T)}{\fdpN[1]T_1}}
+ \sqrt{\frac{4\log(T)}{\fdpN[2]T_1}}.
\]
Similarly as above, by Assumption \ref{ass:embehave}, we know
$\hat{\mu}_1^t - \hat{\mu}_2^t > 0$ for any agent $t$ in the third
level. Therefore, the agents in the third-level group will all pull
arm 1.  \OMIT{Therefore the expected regret is at most
  $\NG \GdT T_1 = O(T^{4/7} \log^{6/7}(T))$.}
\end{proof}


\begin{proof}[Proof of Lemma~\ref{3levelmedium} (Medium gap case)]
  % $2\left(\sqrt{\frac{4\log(T)}{\NG\fdpN[1]T_1}} +
  %   \sqrt{\frac{4\log(T)}{\NG\fdpN[2]T_1}}\right) \leq \mu_1-\mu_2 <
  % 2\left(\sqrt{\frac{4\log(T)}{\fdpN[1]T_1}} +
  %   \sqrt{\frac{4\log(T)}{\fdpN[2]T_1}}\right)$.
  Recall $\bar{\mu}_a$ is
  the empirical mean of arm $a$ in the first two levels. We have
\[
\bar{\mu}_1 - \bar{\mu}_2 \geq \mu_1 -\mu_2 - \sqrt{\frac{4\log(T)}{\NG\fdpN[1]T_1}} - \sqrt{\frac{4\log(T)}{\NG\fdpN[2]T_1}} \geq  \sqrt{\frac{4\log(T)}{\NG\fdpN[1]T_1}}
+ \sqrt{\frac{4\log(T)}{\NG\fdpN[2]T_1}}.
\]
For any agent $t$ in the third level, by Assumption \ref{ass:embehave}, we have
\begin{align*}
\hat{\mu}_1^t - \hat{\mu}_2^t &>\bar{\mu}_1 - \bar{\mu}_2 - \frac{\estC}{\sqrt{\NG\fdpN[1]T_1/2}} - \frac{\estC}{\sqrt{\NG\fdpN[2]T_1/2}}\\
&\geq  \sqrt{\frac{4\log(T)}{\NG\fdpN[1]T_1}} + \sqrt{\frac{4\log(T)}{\NG\fdpN[2]T_1}}- \frac{\estC}{\sqrt{\NG\fdpN[1]T_1/2}} - \frac{\estC}{\sqrt{\NG\fdpN[2]T_1/2}}\\
 &> 0.
\end{align*}
So we know agents in the third-level group will all pull arm 1. \OMIT{Therefore the expected regret is at most
\[
(\NG \GdT T_1 + \NG T_2) \cdot 2\left(\sqrt{\frac{4\log(T)}{\fdpN[1]T_1}}
+ \sqrt{\frac{4\log(T)}{\fdpN[2]T_1}}\right) = O(T^{4/7} \log^{6/7}(T))
\]
}
\end{proof}


\begin{proof}[Proof of Lemma~\ref{3levelsmallcase} (Small gap case)]
  % $ 3\sqrt{\frac{2\log(T)}{T_2}} < \mu_1-\mu_2 <
  % 2\left(\sqrt{\frac{4\log(T)}{\NG\fdpN[1]T_1}} +
  %   \sqrt{\frac{4\log(T)}{\NG\fdpN[2]T_1}}\right)$.
  In this case, we need both arms to be pulled at least $T_2$ rounds
  in the second level. For every arm $a$, consider the $s_a$-th
  second-level group, with $s_a$ given by Lemma~\ref{lem:luck}. We
  have
\begin{align*}
\bar{\mu}_a^{1,s_a} - \bar{\mu}_{3-a}^{1,s_a} &> \mu_a + \frac{1}{4\sqrt{\fdpN T_1}} -\mu_{3-a} +\frac{1}{4\sqrt{\fdpN[3-a]T_1}} \\
&> \frac{1}{4\sqrt{\fdpN[1]T_1}}+ \frac{1}{4\sqrt{\fdpN[2]T_1}} - 2\left(\sqrt{\frac{4\log(T)}{\NG\fdpN[1]T_1}}
+ \sqrt{\frac{4\log(T)}{\NG\fdpN[2]T_1}}\right) \\
&\geq \frac{1}{8\sqrt{\fdpN[1]T_1}}+ \frac{1}{8\sqrt{\fdpN[2]T_1}}.
\end{align*}
For any agent $t$ in the $s_a$-th second-level group, by Assumption \ref{ass:embehave}, we have
\begin{align*}
\hat{\mu}_a^t - \hat{\mu}_{3-a}^t &>\bar{\mu}_a^{1,s_a} - \bar{\mu}_{3-a}^{1,s_a} - \frac{\estC}{\sqrt{\fdpN[1]T_1/2}} - \frac{\estC}{\sqrt{\fdpN[2]T_1/2}}\\
&\geq   \frac{1}{8\sqrt{\fdpN[1]T_1}}+ \frac{1}{8\sqrt{\fdpN[2]T_1}}- \frac{\estC}{\sqrt{\fdpN[1]T_1/2}} - \frac{\estC}{\sqrt{\fdpN[2]T_1/2}}\\
 &> 0.
\end{align*}
So we know agents in the $s_a$-th second-level group will all pull arm $a$. Therefore in the first two levels, both arms are pulled at least $T_2$ times. Now consider the third-level. We have
\[
\bar{\mu}_1 - \bar{\mu}_2  \geq \mu_1 -\mu_2 - 2\sqrt{\frac{2\log(T)}{T_2}} \geq \sqrt{\frac{2\log(T)}{T_2}}.
\]
Similarly as above, by Assumption \ref{ass:embehave}, we know $\hat{\mu}_1^t - \hat{\mu}_2^t > 0$ for any agent $t$ in the third level. So we know agents in the third-level group will all pull arm 1.\OMIT{
Therefore the expected regret is at most
\[
(\NG \GdT T_1 + \NG T_2) \cdot 2\left(\sqrt{\frac{4\log(T)}{\NG\fdpN[1]T_1}}
+ \sqrt{\frac{4\log(T)}{\NG\fdpN[2]T_1}}\right) \leq O(T^{4/7} \log^{6/7}(T))
\]
}
\end{proof} 
%!TEX root = main.tex

\section{$L$-level Recommendation Policy}
In this section, we design an $L$-level recommendation policy for $L > 3$. By having more than 3 levels, we get even smaller regret. 

Our recommendation policy has $L$ levels and each level has $S^2$ groups for $S = \log(T)$. Label the groups in the $l$-th level as $G_{l,a,b}$ for $a,b \in [S]$. 

We set the group size as following. For $l < L$,
\[
T_l = T^{\frac{2^{L-1} + 2^{L-2} + \cdots + 2^{L-l}}{2^{L-1}+ 2^{L-2} + \cdots + 1}}.
\]
and 
\[
T_L = T/S^2 - T_1 \cdot \GdT - T_2 - \cdots - T_{l-1}. 
\]
Each first-level group ($G_{1,a,b}$ for $a,b\in [S]$) has $T_1$ \ALGG runs in parallel. For $l \geq 2$, there are $T_l$ agents in group $G_{l,a,b}$. $T_L$ is a little bit different because we want total number of agents to be $T$. 

Finally we define the information flow. Agents in the first level only observe the history defined in the \ALGG. For agents in group $G_{l,a,b}$ with $l\geq 2$, they observe all the history in the first $l-2$ levels and history in group $G_{l-1,b,c}$ for all $c \in [S]$. For agents which are not in the first level, if they are in the same group, they observe the same history.

\begin{theorem}
The $L$-level recommendation policy gets expected regret $O\left(T^{\frac{2^{L-1}}{2^L-1}} \log^2(T) L \right)$. In particular, if we pick $L = \log\log(T)$, then we get regret $O(T^{1/2} \log^2(T) \log\log(T))$.  
\end{theorem}

\begin{proof}

\end{proof}



%%!TEX root = main.tex
\section{Posterior Mean}
\jmcomment{Do we want this section?} In this section, we give an example to show that for certain natural prior, when agents use this prior to compute the posterior means and use them as the estimators, Assumption \ref{ass:embehave} holds. 

In the following example, it shows that the empirical mean and the posterior mean is really close (i.e. the difference is $O(1/n) = o(1/\sqrt{n})$). It's a special case of conjugate prior Beta distributions (i.e. this prior is just $Beta(1,1)$). Similar guarantees hold for Beta distribution with other parameters. For more details, see \cite{https://en.wikipedia.org/wiki/Conjugate_prior}.
\begin{example}
Let the prior distribution be the distribution over Bernoulli distributions $Ber(p)$ where $p$ is uniformly randomly sampled from $[0,1]$. Suppose there are $a$ 1's in $n$ samples. Then the posterior mean is $\frac{a+1}{n+2}$ and the empirical mean is $\frac{a}{n}$. Their difference is at most $\frac{1}{n+2}$. 
\end{example}

%\jmcomment{This example is against us. Should delete later.}
%\begin{example}
%Let the prior distribution be the uniform distribution over Bernoulli distributions with mean between $1/3$ and $2/3$. Suppose we have $n$ samples. With probability at least $\Omega(1/\sqrt{n})$, the difference between empirical mean and posterior mean is larger than $1/\sqrt{n}$.
%\end{example}

%jmcomment{this is not binary}
%\begin{example}
%Let the prior distribution be the sum of two independent 0-mean Gaussians. The first Gaussian is sampled once and has variance $\alpha^2$. The second Gaussian is sampled for each sample and has variance $\beta^2$. Suppose there are $n$ samples with empirical mean $m$. The posterior mean is $\frac{n \cdot m \cdot \alpha^2}{n \cdot \alpha^2 + \beta^2}$. The difference between the posterior mean and the empirical mean is $\frac{m\beta^2}{n\cdot \alpha^2 + \beta^2}$.
%\end{example}

%\jmcomment{Too many assumptions.}
%\begin{assumption}
%\label{ass:post}
%For any arm $i$ and any $m >0$, let $H_m$ be the random variables of $m$ pulls of arm $i$. Let $\emn(H_m)$ be the empirical mean and $\pmn(H_m)$ be the posterior mean. With probability at least $1 - \exp(m)$, we have $|\emn(H_m) - \pmn(H_m)| \leq \frac{1}{m}$. 
%\end{assumption}

In the above example, the samples of arm $a$ are chosen non-adaptively. Now think about the case when arms are chosen adaptively (i.e. in each round, which arm to choose is based on the history). Clearly, empirical means do not depend on whether samples are chosen adaptively. We show in the following lemma that it is also true for posterior means. It's a folklore and we include the proof for the completeness of our paper.
\begin{lemma}
\label{lem:post}
For any arbitrary strategy of pulling arms for $T$ rounds, let its history to be $G_T$. We allow this strategy to be adaptive based on history. Fix an arm $i$. Let $m_i(G_T)$ be the number of pulls of arm $i$, $\pmn_i(G_T)$ be the posterior mean of arm $i$. Let $H_{m(G_T)}$ be the sub history of arm $i$ in $G_T$. Then $\pmn_i(G_T)$ is the same as the posterior mean of seeing $H_{m_i(G_T)}$ after $m_i(G_T)$ non-adaptive pulls of arm $i$.
\end{lemma}

\begin{proof}
Let's assume there are $m^0_i(G_T)$ 0's and $m^1_i(G_T)$ 1's among of $m_i(G_T)$ pulls of arm $i$ in $G_T$. By the definition of posterior mean, we have
\[
\pmn_i(G_T) = \E[\mu_i|G_T] = \frac{\sum_{x_i} x_i \cdot \Pr[\mu_i=x_i, G_T]}{\sum_{x_i} \Pr[\mu_i=x_i, G_T]}.
\]

We also know that 
\[
\frac{\Pr[\mu_i = x_i, G_T]}{\Pr[\mu_i = x_i', G_T]} = \frac{\Pr[\mu_i = x_i] \cdot  (1-x_i)^{m^0_i(G_T)}x_i^{m^1_i(G_T)}}{\Pr[\mu_i = x_i'] \cdot (1-x_i)^{m^0_i(G_T)}(x_i')^{m^1_i(G_T)} }.
\]

Therefore ,
\[
\pmn_i(G_T) = \frac{\sum_{x_i} x_i \cdot \Pr[\mu_i = x_i] \cdot  (1-x_i)^{m^0_i(G_T)}x_i^{m^1_i(G_T)}}{\sum_{x_i}\Pr[\mu_i = x_i] \cdot  (1-x_i)^{m^0_i(G_T)}x_i^{m^1_i(G_T)}}
\]

Finally, by the definition of posterior mean, the posterior mean of seeing $H_{m_i(G_T)}$ after $m_i(G_T)$ pulls of arm $i$ is
\[
\frac{\sum_{x_i} x_i \cdot \Pr[\mu_i = x_i] \cdot  (1-x_i)^{m^0_i(G_T)}x_i^{m^1_i(G_T)}}{\sum_{x_i}\Pr[\mu_i = x_i] \cdot  (1-x_i)^{m^0_i(G_T)}x_i^{m^1_i(G_T)}} = \pmn_i(G_T).
\]
\end{proof}

%\begin{corollary}
%\label{cor:post}
%For any arbitrary strategy of pulling arms for $T$ rounds, let its history to be $G_T$. We allow this strategy to be adaptive based on history. Fix an arm $i$. Let $m(G_T)$ be the number of pulls of arm $i$, $\emn(G_T)$ be the empirical mean of arm $i$ and $\pmn(G_T)$ be the posterior mean of arm $i$. Let $p_0$ be the probability that $m(G_T) \geq m_0$. With probability $p_0 - 2\exp(m_0)$, $|\emn(G_T) - \pmn(G_T)| \leq \frac{1}{m(G_T)}$.
%\end{corollary}

%\begin{proof}
%Simply by applying Assumption \ref{ass:post}, Lemma \ref{lem:post} and union bound, we can prove the corollary.
%\end{proof}

\end{document}

%%% Local Variables:
%%% mode: latex
%%% TeX-master: t
%%% End:

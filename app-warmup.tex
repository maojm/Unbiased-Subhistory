\section{Proofs from Section~\ref{sec:warmup}}
\label{app:warmup}

\begin{proof}[Proof of Lemma~\ref{lem:greedy}]
  Fix any arm $a$. Let $\GdT = (K-1) \cdot c_T + 1$ and
  $\GdP = (1/3)^{\GdT}$. \swedit{We will condition on the event that
    all the realized rewards in $\GdT$ rounds are 0, which occurs with
    probability at least $\GdP$ under Assumption~\ref{ass:embehave}.}
  In this case, we want to show that arm $a$ is pulled at least
  once. We prove this by contradiction. Suppose arm $a$ is not pulled. By
  the pigeonhole principle, we know that there is some other arm $a'$
  that is pulled at least $c_T + 1$ rounds. Let $t$ be the round in
  which arm $a'$ is pulled exactly $c_T + 1$ times. By Assumption
  \ref{ass:embehave}, we know
  \[
    \hat{\mu}_{a'}^t \leq 0 + c_m / \sqrt{c_T} < 1/3.
  \]
  \nicomment{Wait, where did we set $c_T$?} On the other hand, we have
  $\hat{\mu}_a^t \geq 1/3 > \hat{\mu}_{a'}^t$. This contradicts with
  the fact that in round $t$, arm $a'$ is pulled, instead of arm $a$.
  \swdelete{ In addition, we know that this case happens with
    probability at least $(1/3)^{\GdT} = \GdP$ as each arm's mean is
    at most $2/3$. To sum up, we know that with probability at least
    $\GdP$, \ALGG of $\GdT$ rounds pulls arm $a$ at least once.}
   \jmcomment{Agree with these changes.}
\end{proof}


\begin{proof}[Proof of Theorem~\ref{thm:2level}]
  We will set $T_1$ later in the proof, depending on whether the gap
  parameter $\Delta$ is known. For now, we just need to know we will
  make $T_1 \geq \frac{4(\GdT)^2}{(\GdP)^2}\log(T)$. Since this policy is
  agnostic to the indices of the arms, we assume w.l.o.g. that arm 1
  has the highest mean.

  The first $T_1 \cdot \GdT$ rounds will get total regret at most
  $T_1 \cdot \GdT$.  We focus on bounding the regret from the second
  level of $T - T_1 \cdot \GdT$ rounds. We consider the following two
   events. We will first bound the probability that both of them
  happen and then we will show that they together imply upper bounds
  on $|\hat{\mu}^t_a - \mu_a|$'s for any agent $t$ in the second
  level. Recall $\hat{\mu}^t_a$ is the estimated mean of arm $a$ by
  agent $t$ and agent $t$ picks the arm with the highest
  $\hat{\mu}^t_a$.

% \begin{itemize}
  \OMIT{\paragraph{Concentration of the number of arm $a$ pulls in the first
    level.}
By Lemma \ref{lem:greedy}, we know $\GdP \leq \fdpN \leq \GdT$.}
  Define $W_1^a$ to be the event that the number of arm $a$ pulls in
  the first level is at least $\fdpN T_1- \GdT \sqrt{T_1\log(T)}$.
  \swedit{As long as we set $T_1 \geq \frac{4(\GdT)^2}{(\GdP)^2}\log(T)$,
    this implies that the number of arm $a$ pulls is then at least
    $\fdpN T_1/2$.}
\OMIT{  By Chernoff bound,
  \[
    \Pr[W_1^a] \geq 1-\exp(-2\log(T)) \geq 1-1/T^2.
  \]
}
Define $W_1$ to be the intersection of all these events (i.e. $W_1 = \bigcap_{a}W_1^a$). By Lemma~\ref{lem:t1runs}, we have
\[
\Pr[W_1] \geq 1- \frac{K}{T^2} \geq 1 - \frac{1}{T}.
\]
\OMIT{\paragraph{Concentration of the empirical mean of arm $a$ pulls
    in the first level.}}  Next, we show that the empirical mean of
each arm $a$ is close to the true mean. To facilitate our reasoning,
let us imagine there is a tape of length $T$ for each arm $a$, with
each cell containing an independent draw of the realized reward from
the distribution $\cD_a$. Then for each arm $a$ and any $N\in [T]$, we
can think of the sequence of the first $N$ realized rewards of $a$
coming from the prefix of $N$ cells in its reward tape. Define
$W^{a,t}_2$ to be the event that the empirical mean of the first $t$
\swedit{realized rewards in the tape} of arm $a$ is at most
$\sqrt{\frac{2\log(T)}{t}}$ away from $\mu_a$. Define $W_2$ to be the
intersection of these events (i.e.  $\bigcap_{a,t} W^{a,t}_2$).  By
Chernoff bound,\swcomment{$t$ may be confusing here}
\[
\Pr[W^{a,t}_2] \geq 1 - 2\exp(-4\log(T)) \geq 1-2/T^4.
\]
By union bound,
\[
\Pr[W_2] \geq 1 - KT \cdot \frac{2}{T^4} \geq 1 - \frac{2}{T}.
\]



By union bound, we know $\Pr[W_1 \cap W_2] \geq 1 - 3/T$. For the
remainder of the analysis, we will condition on the event
$W_1 \cap W_2$.

For any arm $a$ and agent $t$ in the second level, by $W_1$ and $W_2$, we have
\[
|\bar{\mu}^t_a - \mu_a| \leq \sqrt{\frac{2\log(T)}{\fdpN T_1 /2}}.
\]
By $W_1$ and Assumption \ref{ass:embehave}, we have
\[
|\bar{\mu}^t_a - \hat{\mu}^t_a| \leq \frac{c_m}{\sqrt{\fdpN T_1/2}}.
\]
Therefore,
\[
|\hat{\mu}^t_a - \mu_a|\leq \sqrt{\frac{2\log(T)}{\fdpN T_1 /2}}+\frac{c_m}{\sqrt{\fdpN T_1/2}} \leq 3 \sqrt{\frac{\log(T)}{\GdP T_1 }}.
\]
So the second-level agents will pick an arm $a$ which has $\mu_a$ at most $6 \sqrt{\frac{\log(T)}{\GdP T_1 }}$ away from $\mu_1$. To sum up, the total expected regret is at most
\[
T_1 \cdot \GdT + T \cdot (1-\Pr[W_1 \cap W_2]) + T \cdot  6 \sqrt{\frac{\log(T)}{\GdP T_1 }}.
\]
By setting $T_1 = T^{2/3}\log(T)^{1/3}$, we get expected regret $O(T^{2/3}\log(T)^{1/3})$.
\OMIT{Notice that if we know the gap parameter is known to be larger than
$\Delta$, we can set
\[
T_1 = \max\left( \frac{4\GdT^2}{\GdP^2}\log(T), \frac{36 }{\Delta^2 \cdot p_G} \log(T) \right).
\]
In this case, since $\Delta \geq 6 \sqrt{\frac{\log(T)}{\GdP T_1 }}$, we know agents in the second level will all pull arm 1. Therefore, the total expected regret is at most
\[
T_1 \cdot \GdT + T \cdot (1- \Pr[W_1 \cap W_2]) = O(\Delta^{-2} \log(T)).
\]
This completes the proof.}
\end{proof}
